\section{Пространство $L_p$}
Сегодня рассмотрим пространство $L_p(E,\mu)$, $1\le p\le \infty$. Распространим понятия, которые были для действительной функции.

Пусть $(X,\E,mu)$ "--- измеримое пространство, а $\F = \begin{cases}\R,\\\C.\end{cases}$ $E\in\E$. Функция $f\colon E\colon \F$, $u(x)=\Re f(x)$, $v(x) = \Im f(x)$, то есть $f(x) = u(x)+i v(x)$.
\begin{Def}
  $f$ "--- измеримая, если $u,v$ измеримы. $f\in L(E,\mu)$, если $u,v\in L(E,\mu)$ и $\I Ef = \I Eu + i\I Ev$.
\end{Def}

Все теоремы, где нет неравенств, верные для действительно значных функций, верны и для комплекснозначных. Некоторые свойства мы с вами докажем.
\begin{Ut}
  Если $f,g\in L(E,\mu)$ и $\lambda\in\F$, то $f+g,\ \lambda f\in L(E,\mu)$ и 
 \[
  \I E{(f+g)} = \I Ef+\I Eg,\ \I E{\lambda f} = \lambda\I Ef.
\] 
\end{Ut}
\begin{Proof}
 Например, докажем последнее свойство. Пусть $\lambda = \alpha + i\beta$, а $f = u + i v$, тогда $\lambda f = (\alpha u - \beta v) + i (\alpha v + \beta u)$. По определению интеграла комплекснозначной функции и по свойству линейности интеграла действительнозначной функции имеем
\[
 \I E{\lambda f} = \I E{(\alpha u - \beta v)} + i\I E{(\alpha v + \beta u)} = \bigg(\alpha Eu - \beta\I Ev\bigg) + i\bigg(\alpha\I Ev + \beta \I Eu\bigg) = \lambda \I Ef.
\]
\end{Proof}

\begin{Ut}
  Пусть $f\in L(E,\mu)$. Тогда $|f|\in L(E,\mu)$ и
\[
  \bigg|\I Ef\bigg|\le \I E{|f|}.
\]
\end{Ut}

\begin{Proof}
  $|f| = \sqrt{u^2+v^2}$, как обычно. Это не превосходит $|f|\le |u|+|v|\in L(E,\mu)$. Осталось доказать равенство. Представим результат интегрирования в тригонометрической форме $\I Ef = \Big|\I Ef\Big|\cdot e^{i\theta}$. Тогда
\[
  \bigg|\I Ef\bigg| = e^{-i\theta}\I Ef = \Re e^{-i\theta}\I Ef = \Re\I E{e^{-i\theta}f} = \I E{\Re(e^{-i\theta}f)}\le \I E{|f|}.
\]
\end{Proof}

\begin{Ut}
  Пусть $f_1\sim g_1$, $f_2\sim g_2$. Тогда $f_1+f_2\sim g_1+g_2$ и $\lambda f_1\sim g_1$.
\end{Ut}
Это свойство очевидно. А если $f\sim g$ и $f\in L(E,\mu)$, то $g\in L(E,\mu)$. Значит, $L(E,\mu)$ есть линейное пространство и множество классов эквивалентных функций есть линейное пространство.

Мы вводили обозначение $B(E) = \big\{f\colon E\to\F\big|f\text{ "--- ограничены на }E\big\}$.
\begin{Def}
  $L_\infty(E,\mu)$ "--- множество классов эквивалентности ограниченных функций с нормой $\|f\|_\infty = \inf\limits_{\mu(A)=0}\sup\limits_{x\in E\dd A}\big|f(x)\big|$. Оно называется множеством существенно ограниченных функций. А норма называется существенной верхней гранью.
\end{Def}

Имеем $L\supset B(E)$ "--- подпространство, $f\sim 0$. Тогда $L_\infty(E,\mu) = B(E)\dd L$. Мы будем обращаться с этими классами, как обыкновенными функциями.

Для каждого $\forall\ n\in\N\pau \exists\ A_n\in\E\colon \mu(A_n)=0,\ \forall\ x\in E\dd A_n\pau \big|f(x)\big|<\|f\|_{L_\infty} + \frac1n$. Обозначим через $A_f = \uN n1 A_n$, $\mu A_f = 0$ и $\|f\|_{L_\infty} = \sup\limits_{x\in E\dd A_f}\big|f(x)\big|$. То есть нижняя грань достигается на некотором множестве меры нуль. Такое множество может быть и не одно. Оно существует, нам этого достаточно, чтобы доказать
\begin{Ut}[Свойства нормы]
  Пусть $\|f\|_{L_\infty} = 0$. Тогда $f\sim 0$. Кроме того, $\|\lambda f\|_{L_\infty} = |\lambda|\cdot \|f\|_{L_{\infty}}$. И неравенство треугольника.
\end{Ut}
\begin{Proof}
  Как доказать неравенство треугольника. Запишем равенства
\[
  \|f\|_{L_\infty} = \sup\limits_{x\in E\dd A_f}\big|f(x)\big|,\pau \|g\|_{L_\infty} = \sup\limits_{x\in E\dd A_g} \big|g(x)\big|.
\]
Положим $A = A_f\cup A_g$. Тогда
\[
  \|f+g\|_{L_\infty}\le \sup\limits_{x\in E\dd A}\big|f(x)+g(x)\big|\le \sup\limits_{x\in E\dd A_f}\big|f(x)\big|+\sup\limits_{x\in E\dd A_g} = \|f\|_{L_\infty} + \|g\|_{L_\infty}.
\]
Вот мы и доказали все свойства нормированного пространства.
\end{Proof}

\begin{The}
  $L_\infty(E,\mu)$ "--- банахово пространство, то есть полное линейное нормированное пространство.
\end{The}
\begin{Proof}
 Рассмотрим последовательность Коши $\big\{f_n\big\}\subset L_{\infty}(E,\mu)$. Положим $A = \bigcup\limits_{n,m=1}^\infty A_{f_n-f_m}$. При этом $\mu (A) = 0$ и $\|f_n-f_m\| = \sup\limits_{x\in E\dd A}\big|f_n(x)-f_m(x)\big|$. Так как $f_n\in B(E\dd A)$ "--- последовательность Коши, то по доказанному на первой же лекции $f_n\rsh[E\dd A]nf\in B(E\dd A)$. Положим $f(x)=0$ на $A$. Тогда $f\in L_\infty(E,\mu)$ и $\|f-f_n\|_{L_\infty}\te 0$.
\end{Proof}

\begin{Def}
  $L_p(E,\mu)$, $1\le p<\infty$ "--- пространство классов эквивалентности измеримых функций $f\colon E\to \F\colon |f|^p\in L(E,\mu)$ с нормой 
\[
  \|f\|_{L_p} = \bigg(\I E{|f|^p}\bigg)^{\frac1p}.
\]
Это линейное пространство функций, суммируемых в степени $p$.
\end{Def}

Заметим, что если $f,g\in L_p(E,\mu)$, то $|f+g|^p\le 2^p (|f|^p+|g|^p)$ ну и ясно, что $\lambda f\in L_p(E,\mu)$. А чтобы доказать, что это нормированное пространство, надо доказать несколько неравенств.
\begin{Ut}[неравенство Гёльдера]
  Пусть $1<p,q<\infty$, $\frac1p+\frac1q=1$ и $f,g\colon E\to\R_+$ и измеримы. Тогда
\[
  \I E{fg}\le \bigg(\I E{f^p}\bigg)^{\frac1p}\bigg(\I E{g^q}\bigg)^{\frac1q}.
\]
Причём эти интегралы могут принимать и бесконечные значения. Суммируемость не требуется.
\end{Ut}
\begin{Proof}
  Сначала докажем неравенство Юнга для чисел $ab\le \frac{a^p}p + \frac{b^q}q$, где $a,b\ge0$. Рассматриваем функции $y = x^{p-1}$ и $x = y^{q-1}$. Легко видеть, что эти функции взаимно обратные. Значит, можно посчитать интеграл слева от кривой и снизу от кривой. А площадь прямоугольника будет меньше
\[
  ab\le \int\limits_0^a x^{p-1}\,dx + \int\limits_0^b y^{q-1}\,dy = \frac{a^p}p+\frac{a^q}q.
\]
Равенство будет только в том случае, когда $a^{p-1} = b$ или, эквивалентно $a^p = b^q$.

Чтобы доказать теперь неравенство Гёльдера, введём обозначения $A = \I E{f^p}$ и $B=\I E{g^q}$. Если одно из этих чисел равно нулю или бесконечности, то неравенство очевидно. Берём $a = \frac f{A^{\frac1p}}$ и $b = \frac q{B^{\frac1q}}$. Применяем неравенство Гёльдера и интегрируем его
\[
  \I E{ab}\le \frac1p\I E{a^p}+\frac1q\I E{b^q} = \frac1p+\frac1q=1.
\]
Отсюда вытекает уже неравенство Гёльдера. Легко видеть, что равенство будет тогда и только тогда, когда $f^p = \lambda g^q$, где $\lambda = A/B$ почти всюду на множестве $E$.
\end{Proof}

Следующее неравенство
\begin{Ut}[неравенство Минковского]
  Пусть $f,g\in L_p(E,\mu)$, $1\le p<\infty$. Тогда $\|f+g\|_{L_p}\le \|f\|_{L_p}+\|g\|_{L_p}$.
\end{Ut}
\begin{Proof}
  В случае $p=1$, это неравенство вытекает из элеметнарного неравенства для чисел $|f+g|\le |f|+|g|$. Нужно проинтегрировать это неравенство, получим неравенство треугольника для $L_1$.

Пусть $p>1$. Положим $A = \I E{|f|^p}$, $B = \I E{|g|^p}$, $C = \I E{|f+g|^p}$. Тогда 
\[
  C = \I E{|f+g|\cdot |f+g|^{p-1}}\le \I E{|f|\cdot |f+g|^{p-1}}+\I E{|g|\cdot |f+g|^{p-1}}.
\]
Найдём $q\colon \frac1p+\frac1q =1,\ (p-1)q = p$. Тогда по неравенству Гёльдера
\[
  C\le A^{\frac1p}\cdot C^{\frac1q} + B^{\frac1p}\cdot C^{\frac 1q}, \quad C^{\frac1p}\le A^{\frac1p} + B^{\frac1p}.
\]
Теперь когда достигается равенство. $|f+g| = |f|+|g|$ почти всюду на $E$ и 
\[
  \frac{|f|^p}A = \frac{|g|^p}B = \frac{|f+g|^p}C
\]
почти всюду на $E$. Из этого вытекает, что $f = h\cdot g $, для $h\ge 0$ почти всюду на $E$. Подставляя, получаем $h = \left(\frac AB\right)^p$ почти всюду на $E$ (если $g\ne 0$). Так что у нас получается, что $f=\lambda g$ и  $\lambda = \left(\frac AB\right)^{\frac1p}$. То есть равенство достигается только тогда, когда функции линейно зависимы, причём с положительным коэффициентом. Значит, $L_p$ является строго нормированным. Элемент приближения является единственным.
\end{Proof}

А вот это уже полезное неравенство.
\begin{Ut}[обобщённое неравенство Минковского]
  Пусть задано два измеримых пространства $(X,\E_x,\mu_x)$ и $(Y,\E_y,\mu_y)$, $E\in\E_x$, $F\in \E_y$ и задана измеримая функция $f\colon E\times F\to \R_+$, а $1\le o<\infty$. Тогда
\[
  \Bigg(\I E{\bigg(\I F{f_x}_y\bigg)^p}_x\Bigg)^{\frac1p}\le \I F{\bigg(\I E{f_y^p}_x\bigg)^{\frac1p}}_y.
\]
\end{Ut}
\begin{Proof}
  Нам понадобится теорема Фубини. Но это неравенство не зря называется обобщённым неравенством Минковского, так как доказывается точно так же. $g(x) = \I F{f_x}_y$ существует для почти всех $x\in E$.
\[
  \I E{g^p}_x = \I E{g\cdot g^{p-1}}_x = \I F{g^{p-1}\bigg(\I F{f_y}_x\bigg)}_y.
\]
Теперь применяем неравенство Гёльдера к произведению двух функций.
\[
 \le \I F{\bigg(\I E{ f_y^p}_x\bigg)^{\frac1p}}_y\cdot \underbrace{\bigg(\I E{g^p}_x\bigg)^{\frac1q}}.
\]
Если поделить на скобку, получится как раз обобщённое неравенство Минковского.
\end{Proof}

\begin{The}
  $L_p(E,\mu)$ "--- банахово пространство при $1\le p<\infty$.
\end{The}
\begin{Proof}
  Возьмём последовательность Коши $\{f_n\}\subset L_p(E,\mu)$. Тогда существует $\{m_k\}\colon m_1<m_2<\dots$ и $\|f_k-f_l\|_{L_p}<\frac1{2^n}\pau \forall\ k,\ge m_n$. Такую подпоследовательность можно выбрать. И рассмотрим функцию (равенство имеет смысл в почти всех точках)
\[
  g(x) = \big|f_{m-1}(x)\big|+\rY n1\big|f_{m_{n+1}}(x)-f_{m_n}(x)\big|.
\]
Если организовать частичные суммы $g_n$, то $g_n\nearrow g$ (значит, и в степени $p$ тоже монотонно возрастают), так как все члены ряда неотрицательны. Кроме того $\|g_n\|_{L_p}\le \|f_{m_1}\|+\rY n1\frac1{2^n} = \|f_{m_1}\|_{L_p}+1$, то есть норма конечная. По теореме о монотонной сходимости $g\in L_p(E,\mu)$. И отсюда $g$ конечна почти всду на $E$. Значит, ряд в определении $g(x)$ сходится почти всюду. Если снять модули, ряд будет сходиться абсолютно почти всюду
\[
  f(x) = f_{m_1}(x) + \rY n1\big(f_{m_{n+1}}(x) - f_{m_n}(x)\big)
\]
сходится абсолютно почи всюду. Тогда
\[
  f_{m_n}(x) = f_{m_1}(x) = \RY k1{n-1}\big(f_{m_{k+1}}(x) - f_{m_k}(x)\big).
\]
Из того, что $|f|^p\le |g|^p\in L(E,\mu)$ следует, что $f\in L_p(E,\mu)$. Если теперь вычесть частичную сумму, получим
\[
  f(x) - f_{m_n}(x) = \rY kn\big(f_{m_{k+1}}(x) - f_{m_k}(x)\big).
\]
Чтобы для бесконечной суммы неравенство можно было использовать, применяем теорему Фату
\[
  \|f-f_{m_n}\|_{L_p}\le \RY k1\infty\|f_{m_{k+1}}-f_{m_k}\big\|_{L_p}<\frac1{2^{n-1}}.
\]
Если имеется в метрическом пространстве последовательность Коши такую, что имеет сходящуюся подпоследовательность, то она сама сходится, что можно легко показать по неравенству треугольника. Значит, мы показали, что $f_n\to f\in L_p(E,\mu)$. Значит, мы доказали полноту.
\end{Proof}

\begin{Lem}
  Обозначим через $H(E,\mu)$ множество простых измеримых функций из $L_p(E,\mu)$, $1\le p\le \infty$. Утверждается, что $H(E,\mu)$ всюду плотно в $L_p(E,\mu)$.
\end{Lem}
\begin{Proof}
  Раскладываем в разность неотрицательнх $f = f_+- f_-$ и $f_\pm = \max\{\pm f,0\}$. Мы доказывали, что $\exists\ h_n^\pm\nearrow f_\pm$, где $h_n^\pm\in H(E,\mu)$. Так как $h_n^\pm$ интегрируемы, то и $f_\pm$ будут интегрируемы. Обозначим
\[
  h = h_n^+ - h_n^-,\pau \|f-h\|_{L_p}\le \|f_+-h_n^+\|_{L_p}+\|f_--h_n^-\|_{L_p}\te0
\]
по тереме о монотонной сходимости.
\end{Proof}

Теперь наша задача показать, что непрерывные функции всюду плотны в $L_p$. А для этого нужно вообще какую-то топологию ввести.

Пусть $(X,\rho)$ "--- метрическое пространство, $(X,\E,\mu)$ "--- измеримое пространство с регулярной мерой и все открытые множества измеримы (а значит и замкнутые и компактные).

\begin{The}
  Множество $C(X)$ непрерывных ограниченных функций (тех из них, что лежат в $L_p$) всюду плотно в $L_p(E,\mu)$ для $1\le p<\infty$ (в отличие от леммы здесь $p<\infty$).
\end{The}

\begin{Proof}
  Возьмём $f\in L_p$ и $\e>0$. По лемме $\exists\ h\in H(E,\mu)$, такая, что $\|f-h\|_{L_p}<\frac\e2$. Всякая простая функция является линейной комбинацией характеристических функций
\[
  h(x) = \RY l1m h_l\chi_{H_l}(x),\pau \chi_A(x) = \begin{cases}1,&x\in A\\0,&x\not\in A.\end{cases}
\]
Существует $\exists\ A$ "--- компактное и $\exists$ открытое $B_l$, для которых $A_l\subset H_l\subset B_l$ и $\mu(B_l\dd A_l)<\left(\frac\e{2c}\right)^p$, где $c = \RY l1m|h_k|$. У нас же функция уже фиксирована.

Напомню $\rho(x,A) = \int\limits_{y\in A}\rho(x,y)$ есть непрерывная функция, поскольку выполняется неравенство
\[
  \rho(x,A)\le \rho(y,A)+\rho(x,y) \imp \big|\rho(x,A) - \rho(y,A)\big|\le \rho(x,y).
\]
\begin{proof}
Доказательство этого неравенства простое $\rho(x,A)\le\big| \rho(x,z) - \rho(z,y)\big|+\rho(z,y)\le \rho(x,y)\pau \forall\ z\in A$.
\end{proof}

Ну теперь давайте построим функцию $g(x)\RY l1m h_lg_l(x)$, $g_l(x) = \frac{\rho(x,X\dd B_l)}{\rho(x,A_l)+\rho(x,X\dd B_l)}$. При этом $0\le g_l(x)\le 1$, $g_l(x)=1$, есил $x\in A_l$, $g_c(x) = 0$, если $x\in X\dd B_l$, то есть $x\not\in B_l$. Все эти функции непрерывны:
\[
  \|\chi_{H_l} - g_l\|_{L_p}\le \mu^{\frac1p}(B_l\dd A_l)<\frac\e{2c}.
\]
И по неравенству Минковского получаем
\[
  \|h-g\|_{L_p}<\frac\e2,\qquad \|f-g\|_{L_p}\le \|f-h\|_{L_p}+\|g-h\|_{L_p}<\e.
\]
\end{Proof}

Закончим таким следствием
\begin{Sl}
  В $L_p[0,1]$, где $1\le p<\infty$ всюду плотно множество
\begin{enumerate}
  \item $H([0,1])$ простых функций;
  \item $C[0,1]$;
  \item $\Til C[0,1]$, $f(0) = f(1)$;
  \item $S$ "--- ступенчатые функции; для некоторого разбиения $0=x_0<x_1<\dots<x_n<1\pau f(x) = \RY k1n c_k\chi_{[x_{k-1},x_k]}(x)$;
  \item $P$ "--- множество алгебраических многочленов, то есть $P(x) = \RY k1n c_k x^k$;
  \item $T$ "--- тригонометрических многочленов $T(x) = \RY k{-n}nc_k e^{2\pi i k x}$;
  \item $C^{\infty}[0,1]$.
\end{enumerate}
\end{Sl}
