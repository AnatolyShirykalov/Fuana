\section{Линейные операторы}
Пусть $E,F$ обозначают нормированные пространства над полем $\F\in\{\R,\C\}$. Норму будем в этих пространствах обозначать одинаково $\|x\|$.
\begin{Def}
  Отображение $A\colon E\to F$ называется линейным оператором, если выполнено два условия
\[
  A(x+y) = A(x)+A(y);\quad A(\lambda x) = \lambda A(x)\pau\forall\ x,y,\in E\ \forall\ \lambda\in\F.
\]

Норма линейного оператора определяется как 
\[
  \|A\| = \sup\limits_{x\in S}\big\|A(x)\big\|,\pau S:=\big\{x\in E\big|\|x\|\le 1\big\}.
\]
\end{Def}

Можно ввести эквивалентное определение для нормы
\[
  \|A\| = \sup\limits_{x\ne0}\frac{\big\|A(x)\|}{\|x\|}.
\]
\begin{Def}
  Оператор $A$ называется ограниченным, если $\forall\ M\subset F$ ограниченного множества образ $A(M) = \big\{y = A(x)\big| x\in M\big\}$ является ограниченным в $F$.
\end{Def}

\begin{Def}
 Если $M$ находится в некотором шаре, то есть $M \subset S_r(x)$, то $M$ называется ограниченным.
\end{Def}

Мы вводили сложное определение ограниченных множеств, оно здесь годится. А наше новое более простое определение не годится для произвольного метрического, только для нормированных.

Если норма оператора конечна, если и только если оператор ограничен. Мы с вами доказывали, что оператор ограничен, значит, непрерывен во всех своих точках.

Приведём пример $A\colon L_p(E,\mu)\to L_p(E,\mu)$, где $1\le p\le\infty$ и определяеся по формуле $A(f):=\phi f$, где $\phi$ "--- ограниченная измеримая функция. Этот оператор называется оператором умножения на функцию. Докажем, что
\[
  \|A\| = \|\phi\|_{L_\infty}.
\]
\begin{Proof}
Давайте вычислять норму. 
\[
  \big\|A(f)\big\|^p = \I E{|\phi\cdot f|^p}.
\]
Поскольку интеграл не зависит от изменения функции на множестве меры нуль, здесь будет такое неравенство
\[
  \big\|A(f)\big\|^p = \I E{|\phi\cdot f|^p}\le \|\phi\|^p_{L_\infty}\I E{|f|^p}.
\]

Извлекая корень, получаем такое неравенство
\[
  \|A\|\le \|\phi\|_{L_\infty}.
\]

Осталось доказать обратное неравенство. Пусть $f= \chi_A$. Тогда (по определению существенной верхней грани) $\exists\ A\subset E\colon \mu(A)>0$, такое, что 
\[
  \forall\ x\in A\pau \big|\phi(x)\big|>\|\phi\|_{L_\infty}-\e.
\]

Подставим эту функцию в оператор
\[
  \big\|A(f)\big\|^p = \I E A{|\phi|^p}>\left(\|\phi\|_{L_\infty}-\e\right)^p\mu(A) = 
    \left(\|\phi\|_{L_\infty}-\e\right)^p\I A{|f|^p}.
\]
Поскольку $\big|A(f)\big|\ge \left(\|\phi\|_{L_\infty}-\e\right)\|f\|$, мы и доказали, что $\|A\|\ge \|\phi\|_{L_\infty}$.
\end{Proof}

Пусть $\L(E,F) = \{ A\colon E\to F| A\text{ "--- линейный и ограниченный}\}$. Норма в этом пространстве есть $\|A\| = \sup\limits_{x\in S}\big\|A(x)\big\|$. Проверим свойства нормы
\begin{Proof}
Сложение и умножение определяются естественно: $(A+B)(x) = A(x)+B(x)$, $(\lambda A)(x) = \lambda\cdot A(x)$.
\begin{enumerate}
  \item Если $\|A\|=0$, то $A(x) = 0$ для всех $x\in E$. Значит, $A =\mathcal O$.
  \item $\|A+B\| = \sup\limits_{x\in S}[\big\|A(x)+B(x)\big\|\le \sup\limits_{x\in S}\big\|A(x)\big\| + \sup\limits_{x\in S}\big\|B(x)\big\|$. Значит, $\|A+B\|\le \|A\|+\|B\|$, а $\|\lambda A\| = |\lambda| A$.
\end{enumerate}
\end{Proof}

\begin{The}
  Если $F$ "--- банахово пространство, то $\L(E,F)$ "--- банахово пространство.
\end{The}

\begin{Proof}
  Пусть $\{A_n\}\in \L(E,F)$ последовательность Коши, то есть
\[
  \forall\ \e>0\pau \exists\ N\in\N\colon \forall\ n,m\ge <\pau \|A_n-A_m\|<\e.
\]
Тогда $\big\|A_n(x)-A_m(x)\big\|<\e \|x\|\pau \forall\ x\in E,\ \forall\ n,m\ge N$.Значит, последовательность $\big\{A_n(x)\big\}\subset F$ является последовательностью Коши в $F$. Значит,
\[
  \exists A(x) = \yo n\infty A_n(x)
\]
и это линейный оператор $A\colon E\to F$. У нас есть его сходимость в каждой точке. Покажем сходимость по норме. Устремим $m\to\infty$ в неравенстве
\[
  \big\|A_n(x)-A(x)\big\|\le \e\|x\|\pau\forall\ x\in E,\ \forall\ n\in N.
\]
\end{Proof}

\begin{The}[Банаха"--~Штейнгауза]
  Пусть $E$ "--- банахово пространство, и задано множество линейных операторов $\{A_i\}_{i\in I}\subset \L(E,F)$, и выполнено условие
\[
  \forall\ x\in E\pau \sup\limits_{i\in I}\big\|A_i(x)\big\|.
\]
Тогда отсюда вытекает, что $\sup\limits_{i\in I}\|A_i\|<\infty$.
\end{The}
То есть из поточечной сходимости следует сходимость по норме.
\begin{Proof}
  Все принципы равностепенной непрерывности здесь выполнены. Мы запишем условия равностепенной непрерывности в точке ноль.
\[
  \forall\ \e>0\pau \exists\ \delta>0\colon\ \forall\ \|x\|<\delta, \forall\ i\in I\pau \big\|A_i(x)\big\|<\e
\]
в силу линейности оператора. Поделим неравенство на $\delta$.
\[
\forall\ \left\|\frac x\delta\right\|<1,\ \forall\ i\in I\pau   \left\|A_i\left(\frac x\delta\right)\right\|<\frac\e\delta
\]
Отсюда вытекает, что $\|A_i\|\le \frac\e\delta$.
\end{Proof}

\begin{Sl}
  Пусть $E$ "--- банахово пространство. И задана последовательно линейных операторов $\{A_n\}\in\L(E,F)$, сходящаяся в каждой точке, то есть $\forall\ x\in E\pau A_n(x)\te A(x)$\footnote{А всякая сходящаяся последовательность ограничена по норме, и можно применить терему.}. Тогда $\sup\limits{n}\|A_n\|<\infty$.
\end{Sl}

Это теорему очень интенсивно будем применять в следующий раз. А сейчас мы докажем очень знаменитую теорему. Для начала введём некоторые понятия.
\begin{Def}
  Пусть $X$ "--- множество. Оно называется упорядоченным, если в нём задано отношение порядка $\le$, то есть
\begin{enumerate}
  \item $x\le x$; \item $x\le y$ и  $y\le z\imp x\le z$; \item $x\le y$ и $y\le x\imp x=y$.
\end{enumerate}  
\end{Def}
\begin{Def}
  Множество $A\subset X$, где $X$ упорядочено, называется цепью, если $\forall\ x,y\in A$ $x\le y$  или $y\le x$. Цепь $A$ называется ограниченной, если $\exists\ y\in X\colon\forall\ x\in A\pau x\le y$.
\end{Def}
\begin{Def}
  Элемент $x\in X$, где $X$ упорядоченно, называется максимальным, если из того, что $x\le y$, следует, что $x=y$.
\end{Def}

Следующая лемма является аксиомой, хотя все её называют леммой. Для нас она будет аксиомой, но вообще она эквивалентна одной из аксиом теории множеств.
\begin{Lem}
  Если всякая цепь $A$ ограничена, то в $X$ существует максимальный элемент.
\end{Lem}

Эту аксиому мы и будем применять для доказательства теоремы.

Пусть $E$ "--- линейное пространство, $f\colon E\to\F$ "--- линейный функционал (он является линейным оператором, только действует в поле).
\begin{enumerate}
  \item $f(x+y) = f(x)+f(y)$; \item $f(\lambda x) = \lambda f(x)$
\end{enumerate}
$\forall\ x,y\in E,\pau \forall\ \lambda\in \F$.

Если $E$ "--- нормированное пространство, то $\|f\| = \sup\limits_{x\in S}\big|f(x)\big|$. 
\begin{Def}
 Пространство $E^* = \{f\colon E\to \F\mid f\text{ "--- линейный и ограниченный}\}$ называется сопряжённым. Ограниченность  $f$ значит, что $\|f\|<\infty$.
\end{Def}
Это банахово пространство.

Будем рассматривать подпространства $L\subset E$ и линейный функционал $f\colon L\to\F$. Введём отношение порядка $f\le g$, где $f\colon L\to\F$, $g\colon M\to \F$, если
\begin{enumerate}
  \item $L\subset M$; \item $\forall\ x\in L\pau g(x) = f(x)$.
\end{enumerate}
Говорят, что $g$ является расширением $f$ на $M$.

Напомню определение полунорм.
\begin{Def}
  $p\colon E\to \R_+$ называется полунормой, если
\begin{enumerate}
  \item $\forall x\in E\pau p(\lambda x) = |\lambda|p(x)$;
  \item $\forall\ x,y\in E\pau p(x+y)\le p(x)+p(y)$.
\end{enumerate}
Пара $(E,p)$ называется полунормированным пространством.
\end{Def}
\begin{The}[Хана"--~Банаха]
  Пусть $(E,p)$ "--- полунормированное пространство и $f\colon L\to \F$ "--- линейный функционал, $L\subset E$ (линейное подпространство) и выполнено условие
\[
  \forall\ x\in L\pau \big|f(x)\big| \le p(x).
\]
Тогда $\exists\ g\colon E\to\F$ линейный функционал на всём $E$, такой, что
\[
  g\big|_L = f\text{ и }\forall\ x\in E\pau \big|g(x)\big|\le p(x).
\]
То есть $g$ является продолжением $f$ с сохранением неравенства.
\end{The}

\begin{Proof}
  Нам для заданного функционала $f$ нужно построить продолжение на всё пространство, причём такое, чтобы выполнялось условие ограниченности. Сначала построим продолжение для линейной оболочки. Пусть $e_1\not\in L$ и $L_1:=\mathop{sp}\{e_1,L\}$. Давайте попытаемся применить лемму Цорна или аксиому Цорна.

Вначале рассмотрим действительный случай, то есть $\F = \R$.
\[
  \forall\ x,y\in L\pau f(x+y)\le p(x+y)\le p(x- e_1) + p(y+ e_1).
\]
Для всех $x$ и $y$ получаем неравенство
\[
  f(x) - p(x-e_1)\le p(y+e_1)-f(y).
\]
Слева функция от $x$, справа "--- функция $y$. Значит, 
\[
  \exists\ c_1\in \R\colon   f(x) - p(x-e_1)\le c_1\le p(y+e_1)-f(y)
\]
Если для некоторого  $\lambda>0$ заменить $x,y$ на $\frac x\lambda, \frac y\lambda$, получаем
\[
  \forall\ x\in L,\forall\ \lambda>0\pau f(x) \pm \lambda c_1\le p(x\pm \lambda e_2).
\]
Тогда мы можем определить линейный функционал на оболочке по формуле
\[
  \forall\ x\in L,\forall\ \lambda\in\R\pau f_1(x+\lambda e_1): = f(x) + \lambda e_1.
\]
Аргумент, обозначим $z = x+\lambda e_1$ принадлежит именно линейной оболочке. Выполнено два условия.
\begin{roItems}
  \item $\forall\ x\in L\pau f_1(x) = f(x)$.
  \item $\forall\ z\in L_1\pau f_1(x)\le p(z)$, при этом $p(-z) = p(z)$, значит, $\big|f_1(x)\big|\le p(z)$.
\end{roItems}
Далее можем определить $L_2\mathop{sp}\{e_2,L_1\}$, где $e_2\not\in L_1$.

Если бы пространство имело счётную размерность, мы бы всё уже доказали.

Надо рассмотреть множество всех продолжений. Применим лемму Цорна. Существует максимальное продолжение на всё пространство.

Теперь перейдём от случая действительных чисел к комплексным числам. Пусть $\F=\C,\ f = u+iv$, причё $u,v$ "--- линейные функционалы над полем действительных чисел. Посчитаем
\[
  u(ix) + i v(ix) = f(ix) = i f(x) = i u(x) - v(x).
\]
Следовательно, $v(x) = -u(ix) $, и функционал записывается в виде
\[
  f(x) = u(x) - i u(ix).
\]
Таким образом, функционал зависит только от своей действительной части. Построим функционал. $\exists\ h\colon E\to\R$ линейный функционал, такой, что $h\big|_{L} = u$ и $\forall\ x\in E\pau \big|h(x)\big|\le p(x)$. Определяем
\[
  \forall\ x\in E\pau g(x):= h(x) - i(h(ix).
\]
Условие $g\big|_{L}=f$ очевидно. Докажем, что функционал линейный над полем комплексных чисел. Достаточно доказать для $ix$
\[
  g(ix) = h(ix)  - i h(x) = i\big(h(x) - i h(x)\big) = i g(x).
\]
То, что он аддитивный, тоже очевидно. Ведь $h$ аддитивный. Покажем ограниченность. Расмотрим тригонометрическое представление $g(x) = e^{i\theta}\big|g(x)\big|$. В силу линейности действительное число
\[
  \big|g(x)\big| = e^{-i\theta} g(x) = g(e^{-i\theta}x) = h(e^{-i\theta}x)\le p(e^{-i\theta}x) = p(x).
\]
\end{Proof}

Саму теорему применяют редко. Важно следствие.
\begin{Sl}
  Пусть $E$ "--- нормированное пространство, а $L\subset E$ "--- линейное подпространство. И пусть задан линейный ограниченный функционал $f\colon L\to\F$. Тогда $\exists\ g\colon E\to \F$ линейный ограниченный функционал, удовлетворяющий условиям: $f\big|_{L} = f$ и $\|g\| = \|f\|$, то есть существует продолжение функционала на всё пространство с сохранением его нормы.
\end{Sl}

\begin{Proof}
  Берём $p(x) =\|f\|_{L}\cdot \|x\|$. Это норма, но она будет и полунормой. Тогда 
\[
  \exists\ g\colon E\to\F\colon g\big|_L = f,\ \big|g(x)\big|\le \|f\|_L\cdot\|x\|\big|g(x)\big|\imp \|g\|\le \|f\|.
\]
Тогда, поскольку $\|f\|_L = \sup\limits_{x\in S\cap L}\big|f(x)\big|$, а норма $g$ считается, как $\sup$ по большему множеству, $\|g\| = \|f\|$.
\end{Proof}

\begin{The}[Рисса]
  Если функционал $\alpha\in C^*[a,b]$, то есть функционал является линейным и ограниченным, определённым на пространстве $C[a,b]$\footnote{Для тех, кто не знает, что такое $C[a,b]$ "--- это пространство непрерывных функций $f \colon [a,b]\to \F$ и $\|f\| = \sup\limits_{x\in[a,b]}\big|f(x)\big|$.}, то $\exists\ F\in \BV[a,b]$, такая, что
\begin{enumerate}
  \item $\forall\ f\in C[a,b]\pau \alpha(f) = \int\limits_a^b f\,dF$;
  \item $\|\alpha\| = \Var(F)$.
\end{enumerate}
\end{The}

\begin{Proof}
  Так как $C[a,b]\subset B[a,b]$, существует продолжение $\alpha\in B^*[a,b]$ по следствию из теоремы Хана"--~Банаха. (Не будем вводить новую букву, пусть тоже $\alpha$.) Рассмотрим $F(t) = \alpha\left(\underbrace{\chi_{[a,t)}}_{u_t}\right)$\footnote{Полуинтервал для непрерывности слева. Доказывать не буду, доказательство очень кропотливое.}, $t\in[a,b]$, $F(a) = 0$.

Рассмотрим разбиение $a=x_0<x_1<\dots<x_n = b$. Тогда
\[
  \RY k1n\big|F(x_k-F(x_{k-1})\big) = \RY k1n e^{-i\theta_k } \big(F(x_k)-F(x_{k-1})\big) = 
\]
Подставим определение функции $F$
\[
 = \RY k1n e^{-i\theta_k}\big(\alpha(u_{x_k}) - \alpha(u_{x_{k-1}})\big) = \alpha\bigg(\RY k1n e^{-i\theta_k}\big(u_{x_k} - u_{x_{k-1}}\big)\bigg).
\]
Заметим, что
\[
  \bigg|\RY k1n e^{-i\theta_k}\big(u_{x_k} - u_{x_{k-1}}\big)\bigg|\le 1.
\]
Значит, 
\[
  \alpha\bigg(\RY k1n e^{-i\theta_k}\big(u_{x_k} - u_{x_{k-1}}\big)\bigg)\le \|\alpha\|.
\]
Таким образом, доказано $\Var\limits_a^b(F)\le \|\alpha\|$.

Возьмём $f\in C[a,b]$, для разбивения $\tau$ возьмём также $f_\tau(x) = \RY k1n f(\xi_k)\big(u_{x_k} - u_{x_{k-1}}\big)$, где $\xi_k\in[x_{k-1},x_k]$. Так как функция непрерывна, $f_\tau\rsH[]{d(\tau)\to0}f$.
\[
  \alpha(f) = \yo{d(\tau)}0\alpha(f_\tau) = \yo{d(\tau)}0 \RY k1n f(\xi_k)\big(F(x_k) - F(x_{k-1})\big) = \int\limits_a^b f\,dF.
\] 
Это и есть интеграл Римана"--~Стилтьеса. Ну а модуль оценивается
\[
  \bigg|\int\limits_a^b f\,dF\bigg|\le \yo{d(\tau)}0\RY j1n\big|f(\xi_k)\big|\big|F(x_k)-F(x_{k-1})\big|.
\]
Значит, $\|\alpha\|\le \Var\limits_a^b (F)\le \|f\|_C\Var\limits_a^b(F)$.
\end{Proof}

Теперь сформулируют ещё одну теорему без доказательства.
\begin{The}[Рисса]
  Если $\alpha\in L^*_p(E,\mu)$, где $1\le p<\infty$, то $\exists\ g\in L_q(E,\mu)$, $\frac1p+\frac1q =1$, такая, что
\begin{enumerate}
  \item $\forall\ f\in L_p(E,\mu)\pau \alpha(f) = \I E{fg}$;
  \item $\|\alpha\| = \|g\|_{L_q}$.
\end{enumerate}
\end{The}

\begin{Sl}
  $L^*_p(E,\mu) = L_1(E,\mu)$ для $1\le p<\infty,\ \frac1p+\frac1q=1$.
\end{Sl}

А для непрерывных имеем $C^*[a,b] = V_0[a,b]$, причём
\begin{enumerate}
  \item $F\in BV[a,b]$;
  \item $\forall\ t\in (a,b)\pau F(t-0) = F(t)$;
  \item $F(a) =0$.
\end{enumerate}
