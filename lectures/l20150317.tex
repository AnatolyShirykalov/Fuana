\section{Сопряжённые операторы}
Сначала несколько дополнительных сведений, которые вам известны уже.

Пусть $E$ "--- нормированное пространство над полем $\F\in\{\R,\C\}$, $E^*$ "--- сопряжённое пространство, оно состоит из всех линейных ограниченных функционалов. Мы говорили, что такое ограниченный. В данном случае это означает, что норма $\|f\| = \sup\limits_{x\in S}\big|f(x)\big|$, где $S = \big\{x\in E\bigm|\|x\|\le 1\big\}$ ограничена.

\begin{Def}
  Система $\{e_k\}_{k=1}^n\subset E$ называется линейно независимой, если из $\RY k1n\l_k\,e_k=0$ следует $\l_k=0$ для $k=1,\dots,n$.

Система функционалов $\{f_k\}_{k=1}^n\subset E^*$ называется линейно независимой, если из $\forall\ x\in E\pau \RY k1n f_k(x)=0$ следует, что $\l_k=0$ для $k=1,\dots,n$.
\end{Def}

\begin{Def}
Системы $\ar ek1n\subset E$ и $\ar fk1n\subset E^*$ называют биортогональными, если для $l\ne k\pau f_k(e_k)=1$ и $f_l(e_k)$.
\end{Def}

Бесконечная система линейно независима, если линейна независима любая конечная подсистема. По лемме Цорна можно выбрать максимальную линейно независимую систему. Её называют базисом Гамеля. Она может быть несчётной.

\begin{The}
  Система функционалов $\ar fk1n\subset E^*$. Тогда существует $\ar ek1n$ биортогональная с $f_k$, если и только если $\ar fk1n$ линейно независима.
\end{The}
\begin{Proof}
 Необходимость. Пусть $\forall\ x\in E\pau  \RY k1n\l_k\,f_k(x)$. Берём $x = e_k$, откуда $\l_k=0$ для $k=1,\dots,n$.

Достаточность по индукции. Для $n=1$ пусть $f_1\ne 0$. Тогда $\exists\ e_1\in E\colon f_1(e_1)=1$. Пусть верно для $n-1$, то есть $\exists\ \ar xk1{n-1}\colon f_l(x_k) = \delta_{lk}$. Рассмотрим $y = x - \RY k1{n-1} f_k(x)\,x_k$ для $x\in E$. Отсюда вытекает, что $f_k(y)=0$ для $k=1,\dots,n-1$ в силу линейности. Если применить к нашему равенству функционал, будет только одно слагаемое. Поэтому это равно нулю для любого $x$.
Если  $\forall\ x\in E\pau f_n(y)=0$ ($y$ зависит от $x$), то отсюда следует, если приметь функционал, что $\forall\ x\in E\pau f_n(x) =\RY k1{n-1}f_k(x)\,x_k$, что невозможно. Значит, существует такой элемент $x\in E\colon f_n(y)\ne0$. Обозначим через $e_n=\frac{y}{f_n(y)}$, а через $e_k$ обозначим элементы $e_k = x_k - f_n(x_k)e_n$ для $k=1,\dots,n-1$. Теперь легко проверить, что эти элементы будут биортогональны нашим функционалам.
\end{Proof}
\begin{Sl}
Пусть задана система элементов $\ar ek1n\subset E$. Тогда существует биортогональная система функционалов $\ar fk1n\subset E^*$, если и только если $\ar ek1n$ линейно независима.
\end{Sl}
\begin{Proof}
Надо вспомнить, как пространство вложено в своё второе сопряжённое $J\colon E\to E^{**}$. $\forall\ x\in E\pau J(x) = \delta_x\in E^{**}$, где $f_x(f) = f(x)$ для $f\in E^*$. Здесь $\big\{\delta_{e_k}\big\}_{k=1}^n$ линейно независимы если и только если система $\ar ek1n$ линейно независима. Отсюда существует $\ar fk1n\subset E^*$ биортогональна системе $\big\{\d_{e_k}\big\}_{k=1}^n$.
\end{Proof}

Пусть $E,F$ "--- нормированные пространства над полем $\F\in\{\R,\C\}$. Через $\L(E,F)$ мы обозначали пространство линейных ограниченных операторов $A\colon E\to F$ с нормой $\|A\| = \sup\limits_{\|x\|\le1}\big\|A(x)\big\|<\infty$.

С каждым линейным оператором (не обязательно ограниченным) связано понятие ядра $\ker(A) = \big\{x\in E\bigm |A(x)=0\big\}$ и понятие образа $\Im(A) = \big\{y\in F\bigm|\exists\ x\in E\colon y=f(x)\big\}$.

\begin{Ut}
 $ker(A)\subset E$ замкнутое линейное подпространство, $\Im(A)\subset F$ линейное подаространство.
\end{Ut}
\begin{Proof}
 Если $x,y\in \ker(A)$, то $A(x+y) = A(x) + A(y)=0$. Значит, $x+y \in ker(A)$. С умножением на число аналогично. Если $x = \lim x_n$, где $x_n\in \ker(A)$, то $A(x) = \lim A(x_n)=0$, то есть $x\in \ker(A)$.

Для образа аналогично доказательность того, что оно подпространство.
\end{Proof}
\begin{Ut}
  Линейный оператор $A\colon E\to F$ (не обязательно ограниченный) является биективным, если и только если его ядро нулевое, а образ всё $F$, то есть
\[
  \ker(A) = 0,\ \Im(A)=F.
\]
\end{Ut}
\begin{Proof}
Если $A$ биективен, то ядро $\ker(A) = A^{-1}(0) = \{0\}$.  и в силу биективности $\Im(A)=F$. Обратно, если $A(x) = A(y)$, то отсюда $A(x-y)=0$, ну а следовательно, так как ядро ноль, $x-y=0$. Значит, отображение $A$ является взаимнооднозначным.
\end{Proof}
\begin{Def}
  Пусть есть два оператора $A\colon E\to F$ и $C\colon F\to F$. Тогда $BA\colon E\to G$ определяется по правилу $\forall\ x\in E\pau BA(x) = A\big(A(x)\big)$ и называется произведением операторов.
\end{Def}
\begin{Def}
Оператор $B\colon F\to E$ называется обратным к оператору $A\colon E\to F$, если $BA = I_E$ и $AB = I_F$ (тождественный оператор).
\end{Def}
\begin{Ut}
  Если $A\in \L(E,F)$ и $B\in\L(F,G)$, то $\|BA\|\le \|A\|\|B\|$.
\end{Ut}
\begin{Proof}
  Имеем $\forall\ x\in S\pau \big\|BA(x)\big\|\le\|B\|\big\|A(x)\big\|\le \|B\|\|A\|$.
\end{Proof}
\begin{Ut}
  Линейный оператор (не обязательно даже ограниченный) $A\colon E\to F$ является биективным, если и только если $A^{-1}F\to E$ является линейным и биективным.
\end{Ut}
\begin{Proof}
Биективность понятна. Первое условие определения даёт, что ядро ноль, а второе условие "--- образ есть $F$. 

Нужно доказать линейность. Пусть $u,v\in F$ и $A^{-1}(u) = x$, $A^{-1}(v)=y$. Тогда
\[
  A^{-1}(u+v) = A^{-1}\big(A(x) + A(y)\big) = A^{-1}A(x+y) = x+y = A^{-1}(u) + A^{-1}(v).
\]
Пусть $\l\in\F$. Тогда $A^{-1}(\l\,u) = A^{-1}\big(\l\,A(x)\big) = A^{-1}A(\l\,x) = \l\, x = \l\,A^{-1}(x)$.
\end{Proof}

\begin{Def}
 Оператор $A^*\colon F^*\to E^*$ называется сопряжённым оператором к оператору $A\in \L(E,F)$, если 
\[
  \forall\ f\in F^*\pau A^*(f) = g\colon \forall\ x\in E\pau g(x):=f(Ax).
\]
\end{Def}
Видно, что так как оператор и функционал линейны и ограничены, то и сопряжённый будет линейный и ограниченный. Только что с вами доказывали, что нормы не более чем перемножатся.

Если ввести обозначения, как мы с вами рассматривали в теории обобщённых функций, $f(x) = \la f,x\ra$, то можно записать следующее равенство.
\[
  \forall\ x\in E,\ \forall\ f\in F^*\pau \la A^*f,x\ra = \la f,Ax\ra.
\]

\begin{Ut}
  Если $A\in\L(E,F)$, то $A^*(F^*,E^*)$ и $\|A^*\|=\|A\|$.
\end{Ut}
\begin{Proof}
  В начале надо доказать линейность. Пусть $f,g\in F^*$ и $A^*(f+g)=h$. Тогда по определению 
\[
h(x) = (f+g)(Ax) = f(Ax) + g(Ax);\imp A^*(f+g) = A^*f+A^*g.
\]
Пусть $\l\in\F$. Тогда
\[
  A^*(\l f) = h,\pau h(x) = (\l f)(Ax) = \l f(Ax);\imp A^*(\l f) = \l A^*(f).
\]

Осталось доказать равенство норм. $\big|(A^*f)(x)\big| = \big|f(Ax)\big|\le \|f\|\|Ax\|\le \|f\|\|A\|\|x\|$. Следовательно,
\[
  \|A^*f\|\le \|f\|\|A\|\imp \|A^*\|\le \|A\|.
\]

По теореме Хана"--~Банаха для фиксированного $x\in E$ существует функционал $f\in F^*$, для которого $f(Ax) = \|Ax\|$ и $\|f\|=1$. Отсюда вытекает, что 
\[
\|Ax\| = f(Ax)\le A^*f(x)\le \|A^*f\|\cdot\|x\|\le \|A^*\|\|x\|,
\]
так как $\|f\|=1$. Ну а отсюда вытекает обратное неравенство $\|A\|\le \|A^*\|$.
\end{Proof}
\begin{Ut}
  Пусть $A\in \L(E,F)$, $B\in \L(F,G)$. Тогда $(BA)^* = A^*B^*$. В частности $(A^{-1})^* = (A^*)^{-1}$, если $A$ биективен.
\end{Ut}
\begin{Proof}
Докажем первое равенство.
\[
  \big((BA)^*g\big)(x) =g\Big(B\big(A(x)\big)\Big) = (B^*g)(Ax) = A^*B^*g(x).
\]
Второе утверждение следует из того, что $A^{-1}A = I_F$ и $AA^{-1} = I_E$. С помощью только что доказанного равенства имеем
\[
  A^*(A^{-1})^* = I_{F^*};\quad (A^{-1})^*A^* = I_{E^*}.
\]
\end{Proof}

То что обратный оператор ограничен следует из теоремы Банаха, которую мы докажем позже.
\begin{Def}
Пусть $V\subset E$ и $W\subset E^*$ "--- некоторые множества. Тогда аннулятором${}^*$ (аналогично ортогональному дополнению) $V$ называетсся $V^\perp = \big\{f\in E^*\bigm|\forall\ x\in V\pau f(x)=0\big\}$. Аналогично $W_\perp = \big\{x\in E\bigm|\forall\ f\in W\colon f(x)=0\big\}$ "--- аннулятор $W$.
\end{Def}
\begin{Lem}
  $V^\perp\subset E^*$, $W_\perp\subset E$ являются замкнутыми подпространствами.
\end{Lem}
\begin{Proof}
  Вытекает из равенств $W_\perp = \bigcap\limits_{f\in W}\ker(f)$, $V^\perp = \bigcap\limits_{x\in V}\ker(\delta_x)$.
\end{Proof}
\begin{The}
  Если $A\in \L(E,F)$, то
\begin{roItems}
\item  $\ker(A) = (\Im A^*)_\perp$;
\item $\ker(A^*) = (\Im A)^\perp$;
\item $\Im(A)\subset \big(\ker(A^*)\big)_\perp$;
\item $\Im(A^*)\subset \big(\ker(A)\big)^\perp$.
\end{roItems}
\end{The}
\begin{Proof}
  Самое сложное равенство первое. Нужно применять теорему Хана"--~Банаха. Имеем
\[
  x\in\ker(A)\iff Ax=0\iff \forall\ f\in F^*\pau f(Ax) = 0
  \iff \forall\ f\in F^*\pau (A^*f)(x) = 0\iff x\in (\Im A^*)_\perp.
\]
Доказательство второго.
\[
  f\in\ker(A^*)\iff \forall\ x\in E\pau f(Ax) = 0 \iff
  f\in (\Im A)^\perp.
\]
Третье. Если $y\in\Im(A)$, то $\exists\ x\in\E\colon y=Ax$. Значит, 
\[
  \forall\ f\in ker(A^*)\pau f(y) = f(Ax) = (A^*f)(x) = 0 \iff y\in \big(\ker(A^*)\big)_\perp.
\]
И последнее. Если $g\in \Im(A^*)$, то $\exists\ f\in F^*\colon \forall\ x\in E\pau g(x) = f(Ax)$. А это как раз и означает, что $g\in \big(\ker(A)\big)^\perp$.
\end{Proof}

Теперь нам осталось привести два примера и закончим.
\begin{Exa}
  $A\colon \F^n\to\F^m$ задаётся матрицей $A = \{a_{kl}\}_{k,l=1}^{m,n}$. Это означает, что если $Ax=y$, $x = (x_1,\dots,x_n)$, $y = (y_1,\dots,y_m)$ и $y_k = \RY l1n a_{kl}x_l$ для $k=1,\dots,m$. Найдём сопряжённый. Пусть $A^*u=v$. Тогда мы имеем
\[
  \la u,Ax\ra = \RY k1m u_ky_k=\RY k1m u_k\bigg(\RY l1n a_{kl}x_l\bigg) = 
  \RY l1n\bigg(\RY k1m a_{kl} u_k\bigg)x_l = \la A^*u,x\ra.
\]
То есть как раз $A^* = \{a_{kl}^*\}$, где $A^*_{kl} = a_{lk}$.
\end{Exa}
\begin{Exa}
  Возьмём такой интегральный оператор $Af(x) = \int\limits_0^1K(x,y)\,f(y)\,dy$. Функция $K(x,y)$ называется ядром интегрального оператора $A$. Предполагается, что
\begin{roItems}
\item $K(x,y)$ измерима на $[0,1]^2$;
\item $\|K\|_{\L_{pq}} = \bigg(\int\limits_0^1\Big(\int\limits_0^1\big|K(x,y)\big|^p\,dx\Big)^{\frac qp}\,dy\bigg)^{\frac 1q}<\infty$ для $q = \frac p{p-1}$, $1<p<\infty$ или $(p,q) = (1,\infty)$.
\end{roItems}
\end{Exa}
\begin{Proof}
  Применяем обобщённое неравенство Миньковского, а затем неравенство Гёльдера.
\[
  \bigg(\int\limits_0^1\big|A f(x)\big|^p\,dx\bigg)^{\frac 1p}\le 
  \int\limits_0^1\bigg(\int\limits_0^1\big|K(x,y)\big|^p\,dx\bigg)^{\frac1p}\cdot \big|f(y)\big|\,dy\le \|K\|_{\L_{pq}}\cdot \|f\|_{\L_p}.
\]
Отсюда $\|A\|\le \|K\|_{\L_{pq}}<\infty$.

Построим теперь сопряжённый. Мы знаем, что $\L_p^*[0,1] = \L_q[0,1]$. Возьмём от-туда $g\in\L_q[0,1]$. Тогда 
\[
  \la g,Ag\ra = \int\limits_0^1 g(x)\bigg( \int\limits_0^1 K(x,y)\,f(y)\,dy\bigg)\,dx = 
  \int\limits_0^1\bigg(\int\limits_0^1 K(x,y)g(x)\,dx\bigg)f(y)\,dy = \la A^*g,\ra.
\]
Значит, $A^*g(y) = \int\limits_0^1 K(x,y) g(x)\,dx$, пишут $K^*(x,y)  = K^*(y,x)$.
\end{Proof}
