\section{Теорема о спектральном радиусе}
Сначала мы разберём примеры.

Найдём спектр преобразования Фурье $\mathcal F\colon \L_2(\R)^n\to\L_2(\R^n)$. Как известно определяется
\[
  \mathcal F(f):=\hat f(x).
\]
Я напомню, что оператор преобразования Фурье сохраняет норму
\[
  \big\|\mathcal F(f)\big\|_{\L_2} = \|f\|_{\L_2}.
\]
При этом $\Im(\mathcal F) = \L_2(\R)$. Значит, $\mathcal F$ "--- изометрия $\L_2(\R)$.

Рассмотрим $h_n(x) = H_n(x)e^{-\frac{x^2}2}$ "--- функции Эрмита. Имеем $\mathcal F(h_k) = (-1)^n h_n(x)$, $n=0,1,2,\dots,$, то есть это собственные функции оператора Фурье. Ктоме того $\{h_n\}$ "--- полная ортонормированная система в $\L_2(\R)$. Таким образом $\sigma(\mathcal F) = \{\pm 1,\pm i\}$, $i=\sqrt{-1}$. То есть у оператора Фурье спектр чисто точечный $\sigma(\mathcal F) = \sigma_p(\mathcal F)$.

Вычислим резольвенту оператора Фурье.
\[
  \L_2(\R) = H_1\oplus H_{-1}\oplus H_i \oplus H_{-i},\pau
  H_\lambda = \ker(\mathcal F_\lambda),\pau \mathcal F_\lambda = \lambda I - \mathcal F.
\]
Здесь будет даже ортогональная прямая сумма. В каждом слагаемом можно выделить базис из функций Эрмита. И таким образом, тождественный оператор можно представить в виде суммы проекторов
\[
  I = P_1 + P_{-1} + P_i + P_{-i},\pau P_\lambda\colon \L_2(\R)\to H_\lambda\text{ "--- ортогональный проектор}.
\]
 Применим преобразование Фурье
\[
  \mathcal F = P_1 - P_{-1} + i P_{i} - i P_{-i}.
\]
Тогда $\mathcal F_\lambda = \lambda I - \mathcal F = (\lambda-1) P_1 + (\lambda+1) P_{-1} + (\lambda-i)P_i +(\lambda+i)P_{-i}$.
 Чтобы найти резольвенту, нужно взять обратный к этому оператор
\[
  \mathcal R_\lambda = \mathcal F^{-1}_\lambda = \mathcal F_\lambda  = (\lambda-1)^{-1} P_1 + (\lambda+1)^{-1} P_{-1} + (\lambda-i)^{-1}P_i +(\lambda+i)^{-1}P_{-i}.
\]

\begin{Def}
 Пусть $E,F$ "--- банаховы пространства, $A\in\L(E)$,  $B\in\L(F)$. Говорят, что $A$ и $B$ эквивалентны или подобны (обозначают $A\sim B$), если существует биектривный оператор $T\in\L(E,F)$, для которого $BT = TA$.
\[
  \begin{diagram}
\node{E} \arrow{e,t}{A} \arrow{s,l}{T} 
\node[1]{E}\arrow{s,r}{T}\\
\node{F}\arrow{e,b}{B}
\node[1]{F}
\end{diagram}
\]

$A\approx B$, если $T$ изометрия.
\end{Def}

Вот такие свойства
\begin{Ut}
  Если $A\sim B$, то $\sigma(A) = \sigma(B)$ и $\sigma_*(A) = \sigma_*(B)$, где $* = p,c,r,l,d$.
\end{Ut}
\begin{Proof}
  Пусть $A_\lambda = \lambda I - A$, $B_\lambda = \lambda I - B$. Тогда $B_\lambda T = T A_\lambda$. Соответственно $A_\lambda$ обратим, если и только если $B_\lambda$ обратим.

$\ker B_\lambda = T(\ker A_\lambda)$ и $\Im B_\lambda = I(\Im A_\lambda)$. Таким образом, $R_\lambda(B) = T R_\lambda(A) T^{-1}$, $\lambda\in\rho(A)$.
\end{Proof}

\begin{Ut}
  Пусть $A\approx B$. Тогда $\|A\|=\|B\|$.
\end{Ut}
\begin{Proof}
  Пусть $B = TAT^{-1}$. Тогда $\|B\|\le \|T\| \|A\| \|T^{-1}\| = \|A\|$, так как $T$ "--- изометрия и $\|T\| = \|T^{-1}\| = 1$. Аналогично доказывается $\|A\|\le \|B\|$. 
\end{Proof}

Теперь мы готовы обсудить следующий пример. Пусть $K\in \L_1(\R)$. Рассмотрим оператор свёртки
\[
  \forall\ f\in\L_2(\R)\pau A f(x) = \Gint{\R} K(x-y) f(y)\,dy;\quad A\colon \L_2(\R)\to\L_2(\R).
\]

Тогда $\widehat{A f}(x) = 2\pi \hat K(x)\hat f(x)$ почти всюду на $\R$. Положим $\phi(x) = 2\pi\hat K(x)$.

Рассмотрим умножение на функцию $\phi$, то есть
\[
  \forall\ g\in\L_2(\R)\pau Bg(x) = \phi(x)g(x).
\]
Ясно, что $B\approx A$. Имеем
\[
  B_\lambda g(x) = \big(\lambda- \phi(x)\big)g(x);\quad
  R_\lambda g(x) = \frac{g(x)}{\lambda-\phi(x)}
\]

Докажем пару утверждений, хотя они уже у нас доказаны.
\begin{azItems}
\item $\|A\| = \|B\| = \max\limits_{x\in\R}\big|\phi(x)\big|$.
\item $\sigma(A) = \sigma(B) = \phi(\ol\R) = \big\{\lambda = \phi(x)\bigm|x\in\ol\R\big\}$, $\ol\R = \R\sqcup\{\pm\infty\}$.
\item $\sigma_p(A) = \sigma_p(B) = \Big\{\lambda\in\phi(\ol\R)\Bigm|\mu\big(\phi^{-1}(A)\big)>0\Big\}$.
\item $\sigma_c(A) = \sigma_c(B) = 
  \Big\{ \lambda\in\phi(\ol\R)\Bigm| \mu\big(\phi^{-1}(\lambda)\big)=0\Big\}$.
\item $\sigma(A) = \sigma_e(A) = \sigma(A)\sqcup \sigma_c(A)$.
\end{azItems}
Доказывать достаточно просто, нужно работать с нашим оператором. Если $\mu\big(\phi^{-1}(A)\big)>0$, то в качестве собственной функции можно взять $e(x) = \chi_{\phi^{-1}(\lambda)}(x)$.

Если $\mu\big(\phi^{-1}(x)\big)=0$. То берёз окрестностьи $O_\delta(\lambda) = \big\{z\bigm| |z-\lambda|<\delta\big\}$ и функции для $g\in\L_2(\R)$, а именно
\[
  g_\delta(x) = 
\begin{cases}
  \frac{g(x)}{\lambda-\phi(x)},&x\not\in \phi^{-1}\big(O_\delta\big);\\
  0,& x \in\phi^{-1}(O_\delta).
\end{cases}
\]
Тогда  $g_\delta\in\L_2(\R)$ и 
\[
  \|g - B_\lambda g_\delta\|_{\L_2} <\e
\]

\begin{The}[о спектральном радиусе]
  Пусть $A\in \L(E)$, $E$ "--- банахово пространство. Тогда $\yo n\infty \sqrt[n]{\|A^n\|} = r(A)$ (спектральному радиусу), где 
\[
  r(A) := \sup\limits_{\lambda\in\sigma(A)}|\lambda|.
\]
\end{The}
Спектральный радиус "--- это радиус наименьшего круга, содерждащего спектр. Мы доказывали, что спектр "--- замкнутое и ограниченное множество в комплексной плоскости.

\begin{Proof}
Пусть $\lambda\in\sigma(A)$. Запишем такое равенство
\[
 \lambda^n - z^n = (\lambda-z) p_{n-1}(z).
\]
Здесь $p_{n-1}(z)$ "--- многочлен степени $n-1$ от $z$. Если теперь подставить оператор $A$ вместо $z$, то мы получим такое равенство
\[
  \lambda^n I - A^n = (\lambda I- A)p_{n-1}(A) = p_{n-1}(A) (\lambda I - A).
\]
Если оператор $\lambda^n I - A^n$ обратим, то и $\lambda I - A$ тоже обратим. Это не возможно, так как $\lambda\in\sigma(A)$. Значит, и $\lambda^n\in \sigma(A^n)$. Отсуда
\[
  |\lambda|\le \|A^n\|
\]
Это мы в прошлый раз доказывали. Отсюда следует, что спектральный радиус будет меньше
\[
  r(A) = \varliminf \sqrt[n]{\|A\|^n}.
\]

Теперь давайте рассмотрим резольвенту.
\[
  R_\lambda = \rY n0 \frac1{\lambda^{n+1}} A^n,\pau |\lambda|>\|A\|.
\]
Возьмём функционал $f\in \L^*(E)$, рассмотрим функцию $F(\lambda) = f(R_\lambda)$, последня  является голоморфной в $\rho(A)$. Значит ряд сходится для
\[
  \forall\ |\lambda|>r(A)\pau 
  F(\lambda) = \rY n0\frac1{\lambda^{n+1}} f(A^n).
\]
Раз ряз сходится, значит есть ограничение
\[
  \forall\ n\in \Z_+\pau \left|f(A^n)/\lambda^{n+1}\right|\le c_f.
\]

По теореме Банаха"--~Штенгауза имеем
\[
  \|A^n/\lambda^{n+1}\|\le c.
\]
Отсюда имеем
\[
  \varlimsup\sqrt[n]{\|A^n\|}\le |\lambda|.
\]
Значит, $\varlimsup\sqrt[n]{\|A^n\|}\le r(A)\le \varliminf\sqrt[n]{\|A^n\|}$.
\end{Proof}

\begin{Def}
  Линейный оператор $A\colon E\to F$, где $E,F$ "--- банаховы пространства, называется компактным, если $\forall\ A\ M\subset E$ ограниченного множества образ $A(M)\subset E$ является предкомпактным. 
\end{Def}
Давайте обозначим $\mathcal K(E,F)$ "--- множество всех компакных операторов.

\begin{Ut}
  $\mathcal K(E,F)\subset \L(E,F)$. Ведь предкомпактное множество явлется ограниченным. Значит, каждое ограниченное множество компактный оператор переводит в ограниченное.
\end{Ut}

\begin{Ut}
  $A\in\mathcal K(E,F)\iff A(S)\subset F$ является предкомпактным, где $S = \big\{x\in E\bigm\|x\|\le1\big\}$.
\end{Ut}
Шар разиуса $S_r = r S$. Так что это утверждение очевидно.
\begin{Ut}
Если $A\in\L(E,F)$ и $\dim E<\infty$ или $\dim F<\infty$, то $A$ компактный.
\end{Ut}
Потому в конечномерном пространстве всякое ограниченное множество предкомпактно. Так что это утверждение тоже очевидно.

\begin{Ut}
  Если $A,B\in\mathcal K(E,F)$, то их сумма $A+B\in\mathcal K(E,F)$. 
\end{Ut}
\begin{Proof}
В самом деле. Берём последовательность $\{x_n\}\subset S$ с единичного шара. Существует последовательность индексов $\{n_k\}$, для которой $A_{x_{n_k}} \to y$. Ну и существует ещё одна последовательность $\{n_{k_j}\}$, для которой $B_{x_{n_{k_j}}}\to z$. Таким образом
\[
  (A+B)x_{n_{k_j}}\to A x_{n_{k_j}} + B_{x_{n_{k_j}}}\to y + z.
\]
И согласно второму свойству, сумма является компактным оператором.
\end{Proof}

\begin{Ut}
  Если $A\in\mathcal K(E,F)$, $B\in\L(F,G)$, то $BA\in\mathcal K(E,G)$.
\end{Ut}
\begin{Ut}
  Если $A\in\L(E,F)$, $B\in\mathcal K(F,G)$, то $BA\in\mathcal K(E,G)$.
\end{Ut}
Эти утверждения очевидны. В первом: $B$ непрерывны, а значит, предкомпактные переводит в предкомпактные.

\begin{Ut}
  Если $A\in\mathcal K(E,F)$ и $\dim E = \infty$, то $A$ необратимый оператор.
\end{Ut}
Тоже почти очевидное свойство.
\begin{Proof}
Если $A$ обратимый, то $A^{-1}$ будет ограниченный, то есть из $\L(F,E)$ по теореме Банаха об обратном операторе. Поэтому $A^{-1}A = I$ в $E$. Ну и согласно свойству 5, тождественный оператор будет компактным. Но тождественный оператор не является компактным в бесконечномерном пространстве по теореме Рисса. Единичный шар является компакнтым тогда и только тогда, пространство имеет конечную размерность.
\end{Proof}

\begin{Lem}
  Если $A_n\mathcal K(E,F)$ и $\|A_n-A\|\to 0$ (пространство $\L(E,F)$ является банаховым, поэтому $A$ будет как минимум ограниченным), то $A$ компактный оператор.
\end{Lem}
\begin{Proof}
 $\forall\ \e>0\pau \exists\ n\colon \|A_n-A\|<\frac\e2$.
 Пусть у нас $\{y_k\}_{k=1}^m\subset A_n(S)$ "--- $\e/2$-сеть в $A_n(S)$.
По неравенству треугольника имеем
\[
  \|y_k-Ax\|\le \|y_k-A_nx\| + \|A_nx - Ax\|.
\]
Имеет местро следующее $\forall\ x\in S\pau \exists k\colon \|y_k - A_n x\|<\frac\e2$. Значит
\[
  \|y_k-Ax\|<\e.
\]
Значит $\{y_k\}_{k=1}^m$ является $\e$-сеть в $A(S)$.
\end{Proof}
\begin{The}[Шаудера]
  Пусть $E,F$ "--- банаховы пространства, $A\in\L(E,F)$. Тогда $A\in\mathcal K(E,F)$, если и только если $A^*\in\mathcal K(F^*,E^*)$.
\end{The}
\begin{Proof}
 Необходимость.  Пусть у нас $K = \ol{A(S)}$. Это "--- компакт (замыкание предкомпактного). Для каждой функции $f$ из единичного шара сопряжённого пространства построим функцию на компакте $K$.
\[
  \forall\ f\in S^*\subset F^*\pau
  g(y) := f(y),\ \forall\ y\in K\imp g\in C(K).
\]
Так как функционал непрерывный, то и функция будет непрерывной на компакте $K$.

Проверим, что семейство таких функций будет равномерно ограниченным и равностепенно непрерывным. Дейтсвительно
\[
  \sup\limits_{y\in K} \big|g(y)\big| = \sup\limits_{x\in S} \big|f(Ax)\big|\le \|A\|.
\]
И равностепенная непрерывность следует из
\[
  \big|g(y_1)-g(y_2)\big| = \big|f(y_1-y_2)\big|\le \|y_1-y_2\|.
\]
Поэтому вот это $V = \big\{g\in C(K)\bigm| \forall\ y\in K,\ f\in S^*\pau g(y) = f(y)\big\}$ является предкомпактным. Кроме того $M$ изометрично $A^*(S^*)$.
В самом деле $\|A^*f\| = \sup\limits_{x\in S}\big\|A^* f(x)\big\| = \sup\limits_{x\in S} \big|f(Ax)\big| = \|g\|_{C(K)}$. Ну раз они изометричны и одно из них предкомпактно, то и второе предкомпактно.

Теперь достаточность.
Пусть $A^*\in\mathcal K(F^*,E^*)$. Отсюда следует, что второй сопряжённый будет компактным, то есть $A^{**}\in K(E^{**},F^{**})$ по доказательству необходимости. 
Тогда рассмотрим изометрические вложения  
$J_1\colon E \to E^{**}$, $J_2\colon F\to F^{**}$. Тогда $A^{**} J_1 = J_2A$. Следовательно, $J_2\big(A(S)\big) = A^{**}(J_1S)\subset A^{**}(S)$.
\end{Proof}

Осталось нам пример рассмотреть. Мы рассмотрим интегральный оператор. $Af(x) = \int\limits_0^1 K(x,y) f(y)\,dy$. $K(x,y)$ называется ядром оператора
\[
  A\colon \L_p[0,1]\to \L_p[0,1],\pau 1\le p\le \infty.
\]
Не обязательно для всех таких ядер оператор будет определён.
\begin{roItems}
\item $k(x,y)\in C[0,1]^2$. Докажем, что $A\in\mathcal K(\L_p,\L_p)$. Воспользуемся теоремой Арцелла"--~Асколли. Ограниченность вытекает из неравенства (неравенство Гёльдера и мажорирование нормы из $\L_q$)
\[
  \|Af\|_{\L_p}\le \|k\|_C\cdot \|f\|_{\L_p}.
\]
А значит, что $A(S)$ "--- ограниченное множество в $\L_p$.

Теперь для $\forall\ \e>0$ найдём такое $\delta>0\colon \forall\ |x_1-x_2|<\delta,\ \forall\ y\in[0,1]\pau \big|k(x_1,y) - k(x_2,y)\big|<\e$. Тогда
\[
  \big|Af(x_1) - Af(x_2)\big|<\e\|f\|_{\L_p}.
\]
Значит, $A(S)$ "--- равностепенно непрерывна. Если множество предкомпактно $\L_p$, то оно предкомпактно в $C[0,1]$.
\item Пусть $k\in\L_r[0,1]^2$, где $r = \max\{p,q\}$, где $\frac1p +\frac1q = 1$. Ну и будем считать, что $1<p<\infty$.
$  \|f\|_{\L_p}\le \|f\|_{\L_q}$, если $p<q$. Неравенство Гёльдера один раз применить. Нужно взять здесь $s = \frac qp$. Наисать $\frac1s + \frac1{s_1}=1$. И получится требуемое неравенство.

Из этого неравенства вытекает, что
\[
  \|A f\|_{\L_p}\|\le \|k\|_{\L_r}\cdot\|f\|_{\L_p},\pau A\colon \L_p\to\L_p.
\]
Теперь лемму применим. Существует последвоательность $K_n\in C[0,1]^2$, для которой $\|k-k_n\|_{\L_r}<\frac1n$, так как множество непрерывных всюду плотно в $\L_p$. Применим оператор
\[
  A_n f(x) = \int\limits_0^1 k_n(x,y) f(y)\,dy.
\]
У нас из старого неравенства вытекает следующее
\[
  \|A f - A_nf\le \| k-k_n\|_{\L_p}\|f\|_{\L_p}.
\]
Отсюда $\|A-A_n\|<\frac1n$ и $A_n\te A$ по норме. Отсюда $A$ компактный оператор.
\end{roItems}
