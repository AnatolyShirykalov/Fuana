\section{Мера множеств}
Пусть $X$ "--- множество. Тогда $2^X$ "--- совокупность всех его подмножеств, а $S\subset 2^X$ называется системой множеств.

Положим по определению $E = \bigcup\limits{A\in S}A$. Это называется единицей системы $S$.

\begin{Def}
  Система $S$ называется кольцом, если $\forall\ A,B\in S\pau A\cup B,A\dd B\in S$, то есть кольцо замкнуто относительно конечного числа объединений и разностей. Если кольцо $S\supset E$, оно называется алгеброй.
\end{Def}

Пусть $S\subset 2^X$. Тогда $\Rim(S)$ "--- наименьшее кольцо, содержащее систему $S$, а $\Aim(S)$ "--- наименьшая алгебра, содержащая $S$, то есть $\Rim(S)$ пересечение всех колец, содержащих $S$, $\Aim(S)$ "--- пересечение всех алгебр, содержащих $S$.

\begin{Ut}
  $S$ "--- кольцо, если и только если $\forall\ A,B\in S\pau A\cap B\in S$ и $A\sdd B\in S$.
\end{Ut}

\begin{Proof}
  Это доказывается с помощью таких равенств
\[
  A\cap B = A\dd (A\dd B);\quad A\sdd B = (A\dd B)\cup (B\dd A);\quad
  A\cup B = (A\sdd B)\sdd(A\cap B);\quad A\dd B = A\sdd (A\cap B).
\]
\end{Proof}

\begin{Def}
  Кольцо (алгебра) $S$ называется $\sigma$-кольцом (-алгеброй), если
\[
  \forall\ A_n\in S\pau \uN 1n A_n\in S.
\]
\end{Def}

\begin{Def}
  Кольцао (алгебра) $S$ называется $\delta$-кольцом (-алгеброй), если
\[
  \forall\ A_n\in S\pau \caP 1n A_n\in S.
\]
\end{Def}

\begin{Ut} 
  Условия для $\sigma$ и $\delta$ алгебры совпадают.
\end{Ut}
\begin{Proof}
  Запишем формулы двойственности.
\[
  E\dd \bigg( \uN n1 A_n\bigg) = \caP n1(E\dd A_n),\quad
  E\dd \bigg( \caP n1 A_n\bigg) = \uN n1(E\dd A_n).
\]
Утверждение, очевидно, доказано.
\end{Proof}

$\Rim_\sigma(S)$ "--- это наименьшее $\sigma$-кольцо, содержащее $S$, $\Aim_\sigma(S)$ "--- это наименьшая $\sigma$-алгебра, содержащая $S$.
\begin{Def}
  Пусть $(X,\rho)$ "--- метрическое пространство, $\tau$ "--- топология. Тогда $\Aim_\sigma(\tau) =: \B(X)$ называется борелевской $\sigma$-алгеброй метрического пространства $X$.
\end{Def}

\begin{Def}
  $S$ называется полукольцом, если $\forall\ A,B\in S\pau A\cap B\in S$ и $A\dd B = \DUN i1n C_i$, где $C_i\in S$.
\end{Def}

\begin{Ut}
  Если $S$ "--- полукольцо, то $\forall A, B_i\in S\pau A\dd \UN i1n B_i = \CAP j1n C_j$, где $C_j\in S$.
\end{Ut}

\begin{Proof}
  По индукции. Для $n=1$ верно. Пусть верно для $n$, докажем для $n+1$.
\[
  A\dd \UN i1{n+1} B_i = A\dd \UN i1nB_i\dd B_{n+1},
\]
что есть $\DUN j1m (C_j\dd B_{n+1}) = \DUN j1m\DUN i1n C_{ij}$, где $C_{ij}\in S$, что и требовалось доказать.
\end{Proof}

\begin{Lem}
  Пусть $S$ "--- полукольцо. Тогда $A\in \Rim(S)$ если и только если $A = \DUN i1n A_i$, где $A_i\in S$.
\end{Lem}

\begin{Proof}
  Положим $R = \{ A = \DUN i1n A_i\mid n\in\N,\ A_i\in S\}$. Отметим, что $R\subset \Rim(S)$. Покажем, что $R$ "--- кольцо.
\[
  A = \DUN i1n A_i,\quad B=\DUN j1mB_j,\quad A_i,B_j\in S.
\]
$A\dd B = \DUN i1n (A_i\dd B)$. В силу доказанного выше утверждения это является $\DUN i1n\DUN j1mC_{ij} C_{ij}$, где $C_{ij}\in S$. Следовательно, $A\dd B\in R$.

$A\cup B = A\dd B\sqcup B\in R$. Следовательно $R$ "--- кольцо. И, следовательно, $R = \Rim(S)$.
\end{Proof}

Пусть $X$ "--- множество. Опять же $S\subset 2^X$. И функция $\phi\colon S\to \F$, где $\F\in\{\R,\C\}$.
\begin{Def}
 Функция $\phi$ называется аддитивной, если $\phi (A\sqcup B) = \phi(A)+\phi(B)\pau \forall\ A,B,A\sqcup B\in S$. $\phi$ называется конечно аддитивной, если $\phi \bigg(\DUN i1n A_i\bigg)  = \RY i1n\phi(A_i)\pau A_i,\DUN i1n A_i\in S\pau \forall\ n\in \N$.
\end{Def}
\begin{Def}
  $\phi$ называется $\sigma$-аддитивной, если $\phi \bigg(\duN i1 A_i\bigg)  = \rY i1\phi(A_i)\pau A_i,\duN i1 A_i\in S\pau \forall\ n\in \N$.
\end{Def}

Так как $\duN i1$ не зависит от порядка множеств, то ряд сходится абсолютно.
\begin{Def}
  Функция $m\colon S\to\R_+$ называется конечно-аддитивной мерой ($\sigma$-аддитивной мерой), если
\begin{enumerate}
  \item $S$ "--- это полукольцо;
  \item $m$ конечно (или $\sigma$-) аддитивна.
\end{enumerate}
\end{Def}

\begin{Def}
  Мера $m_1\colon S_1\colon R_+$ называется называется продолжением меры $m\colon S\to \R_+$, если $S\subset S_1$ и ограничение $m_1\big|_{S_1} = m$.
\end{Def}

\begin{The}
  Для любой меры $m\colon S\to \R_+\pau\exists!\ m_1\colon S_1\to \R_+$ продолжение, где $S_1\in \Rim(S)$.
\end{The}
\begin{Proof}
  Определим $m_1(A) = \RY i1n m(A_i)$, где $A = \DUN i1n A_i$, $A_i\in S$. Пусть $A = \DUN i1n A_i = \DUN j1m B_j$. Тогда одновременно выполнится $A = \DUN i1n\DUN j1m (A_i\cap B_j)$ и $m_1(A) = \RY i1n\RY j1m m(A_i\cap B_j)$ не зависит от разложения $A$.

 Пусть $A = \DUN i1n A_i$, $A_i\in\Rim(S)$. В свою очередь $A_i = \DUN j1{m_i} A_{ij}$, где $A_{ij}\in S$. Соответственно, 
\[
  A = \DUN i1n\DUN j1{m_i}A_{ij},\quad m_1(A) =\RY i1n\RY j1{m_i} m(A_{ij}) = \RY i1nm_1(A_i).
\]
Таким образом доказана конечная аддитивность. Устремив $n\to\infty$  в предыдущих рассуждениях, докажем $\sigma$-аддитивность.
\end{Proof} 

\subsection{Свойства $\sigma$-аддитивной меры}
Пусть $m\colon S\to \R_+$ "--- $\sigma$-аддитивная мера. Тогда
\begin{Ut}
  $m(\q) = m(\q\sqcup \q) = 2m(\q)\imp m(\q) = 0$.
\end{Ut}
\begin{Ut}[монотонность]
  Если $\duN i1 A_i\subset A$, причём $A_i,A\in S$, то $\rY i1m(A_i)\le m(A)$.
\end{Ut}
\begin{Proof}
  Возьмём фиксированное $n\in\N$. Тогда $\DUN i1n A_i\subset A$ и $A = \DUN i1n A_i \sqcup \DUN j1m B_j$, $A_i,B_j\in S$. Тогда
\[
  m(A) = \RY i1n m(A_i)\RY j1n m(B_j) \ge \RY i1n m(A_i).
\]
Устремим $n\to\infty$ и получим требуемое.
\end{Proof}

\begin{Ut}[полуаддитивность]
  Пусть $A\subset \uN i1A_i$, где $A,A_i,\uN 1iA_i=: B\in S$. Тогда $m(A)\le \rY i1m(A_i)$.
\end{Ut}
\begin{Proof}
  Берём $B_1 = A_1$, $B_k = A_k\dd \bigg(\UN i1{k-1} A_i\bigg)$, где $k=2,3,\dots$ $B_k\in \Rim(S)$. Считаем, что $m$ определена для $B_k$, как продолжение меры. $B = \duN k1B_k$ и $m(B) = \rY k1B_k$. Так как $A\subset B$, $m(A)\le m(B)\le \rY k1B_k\le \rY k1A_k$.
\end{Proof}
\begin{Ut}[непрерывность снизу]
  Если $A_i\uparrow A$, $A,A_i\in S$, то $\yo i\to \infty m(A_i) = m(A)$.
\end{Ut}
\begin{Proof}
  Что значит стрелочка вверх: $A_1\subset A_2\subset \dots$ и $\uN i1A_i = A$. Пусть $A_0 = \q$, $B_i = A_i\dd A_{i-1}$. Тогда
\[
  A = \DUN i1\infty B_i,\pau B_i\in\Rim(S).
\]
Считаем меру $m$ продолженной на $\Rim(S)$. Тогда $m(A) = \RY i1\infty m(B_i) = \rY i1(A_i\dd A_{i-1}) = \yo i\infty m(A_i)$.
\end{Proof}
Сформулируем обратное утверждение.
\begin{Ut}
  Если конечно аддитивная мера непрерывна снизу, то она $\sigma$-аддитивна.
\end{Ut}

\begin{Proof}
  Путьс $A = \duN i1A_i$, $A_i,A\in S$. Положим, $B_n = \DUN i1n A_i$. Тогда $B_n\uparrow A$ и $m(A) = \yo n\infty m(B_n) = \rY i1m(A_i)$.
\end{Proof}

\begin{Ut}[непрерывность сверху]
  Если $A_i\downarrow A$, $A_i,A\in S$, то $\yo i\to \infty m(A_i) = m(A)$.
\end{Ut}
\begin{Proof}
  $A_1\supset A_2\supset \dots$ и $\caP i1A_i = A$. Обозначим $B = A_1\dd A$, $B_i := A_1\dd A_i$, $i=1,2,\dots$ Тогда $B_i\uparrow B$ и $m(B) = \yo i\infty m(B_i)$.  $m(A_i)- m(A) = m(A_i) - \yo i\infty m(A_i)$, следовательно, $m(A) = \yo i\infty m(A_i)$.
\end{Proof}
\begin{Ut}
  Если конечно аддитивная мера непрерывна сверху, то она $\sigma$-аддитивна.
\end{Ut}
\begin{Proof}
  $A = \duN i1A_i$, $A,A_i\in S$, $B_n = \DUN i1n A_i$, $B_n\downarrow \q$. $\yo n\infty B_n= 0$. Тогда
\[
  m(A) - \yo n\infty \RY i1nm(A_i) = 0.
\]
Что и требовалось доказать.
\end{Proof}
\begin{Def}
  Пусть $(X,\rho)$ "--- метрическое пространство, $S$ "--- полукольцо в $X$. Мера $m\colon S\to \R_+$ называется регулярной, если
\[
  \forall\ \e>0,\ \forall\ A\in S\pau \exists\ B,C\in S\colon \ol B\text{ компактно},\ \ol B\subset A\subset C^0, m(C\dd B)<\e.
\]
\end{Def}
\begin{The}
  Каждая регулярная мера $m\colon S\to\R_+$ является $\sigma$-аддитивной.
\end{The}
\begin{Proof}
  Пусть $A = \duN i1A_i$, $A_i,A\in S$. $m(A) \ge \rY i1m(A_i)$. Существуют $B,C,B_i,C_i\in S\colon \ol B,\ol B_i$ "--- компакты, $\ol B\subset A\subset C^0, \ol B_i\subset A_i\subset C^0_i$ и $m(C\dd B)<\frac\e2$, $m(C_i\dd B_i)<\frac\e{2^{i+1}}$.

$\ol B\subset \uN i1 C_i^0$. Из компактности следует, что $\ol B\subset \UN i1nC_i^0$. Следовательно, $m(B)\le \RY i1nm(C_i)$.
\[
  m(A)\le m(C)\le m(B)+\frac\e2\le \RY i1nm(C_i)+\frac\e2\le \RY i1n m(B_i) + \RY i1n\frac\e{2^{i+1}}+\frac\e2\le \rY i1m(A_i)+\e.
\]
Так как $\e$ "--- произвольная постоянная, получаем требуемое.
\end{Proof}
\subsection{Мера Стилтьеса в $\R$}
Пусть $S = \big\{[a,b)\big| a,b\in\R,\ a\le b\big\}$. Это полукольцо. Пусть $\alpha(x)$ "--- неубывающая функция на $\R$.
\begin{Def}
  $m_\alpha\big([a,b)\big) = \alpha(b)-\alpha(a)$. $\alpha$ называется функцией распределения, а $m_\alpha$ "--- конечно-аддитивная мера.
\end{Def}
\begin{The}
  Мера $m_\alpha$ является $\sigma$-аддитивной, если и только если $\alpha(x)$ непрерывна слева.
\end{The}
\begin{Proof}
 Необходимость. Пусть $x_n\uparrow x$. Тогда полуинтервал $[x_n,x)\downarrow\q$. Следовательно, существует предел $\yo n\infty m_\alpha\big([x_n,x)\big) = 0$. Следовательно, $\alpha(x) = \yo n\infty \alpha(x_n)$, то есть $\alpha$ непрерывна слева.

Достаточноть. Пусть $\forall\ x\in \R\pau \alpha(x-0) = \alpha(x)$. Полуинтервал $[a,b-\delta)\subset [a,b)\subset (a-\delta,b)\pau \forall\ \delta>0$.
\[
  m_\alpha\big([a-\delta,b)\dd [a,b-\delta)\big) = m_\alpha\big([a-\delta,a)\big) + m_\alpha\big([b-\delta,b)\big) = 
  \alpha(a) - \alpha(a-\delta) + \alpha(b)- \alpha(b-\delta)\le \frac\e2+\frac\e2 = \e.
\]
Мера Стилтьеса регулярна, следовательно, $\sigma$-аддитивна.
\end{Proof}
