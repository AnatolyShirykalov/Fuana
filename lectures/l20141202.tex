\section{Сильная и слабая сходимости линейных операторов и линейных функционалов}
Пусть $E,F$ "--- нормированные пространства, $\L(E,F)$ "--- пространство ограниченных операторов.
\begin{Def}
  Последовательность операторов $\{A_n\}\in\L(E,F)$ сходится сильно $A_n\to A$ к оператору $A$, если
\[
  \forall\ x\in E\pau \yo n\infty A_n(x) = A(x).
\]
Сходится равномерно, если $\|A_n-A\|\te0$, то есть сходится по норме.
\end{Def}
Сильная сходимость "--- это поточечная сходимость, а равномерная значит по норме.

Соответственно вводится понятия сильной и равномерной ограниченности.
\begin{Def}
  Подмножество $M\subset \L(E,F)$ равномерно ограничено, если 
\[
  \exists C>0\colon\forall\ A\in M\pau  \|A\|\le C.
\]
\end{Def}
\begin{Def}
  $M\subset \L(E,F)$ сильно ограничено, если 
\[
  \forall\ x\in E\exists\ C_x>0\colon \forall\ A\in M\pau \big\|A(x)\big\|\le C_x.
\]
\end{Def}
Давайте свойства обсудим.
\begin{Ut}
  Если $A_n\te A$ равномерно, то схоится сильно.
\end{Ut}

\begin{Proof}
  Доказательство почти очевидно.
\[
  \|A_n-A\|\te0,\pau \big\|A_n(x) - A(x)\big\|\le \|A_n-A\|\cdot\|x\|\te0,\pau \forall\ x\in E.
\]
\end{Proof}
Надо понимать, что сходимость здесь везде по норме. Просто в сильной сходимости сходимость по норме $F$.

Нужно сделать следующее замечание.
Если $\dim E<\infty$, то верно и обратное утверждение. Я его доказывать не буду, это можно сделать, пользуясь теоремой об эквивалентности норм в конечномерном пространстве.

\begin{Ut}
  Если $A_n\te A$ сильно, то $\{A_n\}$ сильно ограничена, и выполнено вот такое неравенство
\[
  \|A\|\le\varliminf\|A_n\|.
\]
\end{Ut}
 Неравенство очень похоже на лемму Фату.
\begin{Proof}
  Первое свойство почти очевидно. Если сходится для каждого $x$, то ограничена по норме $F$. Отсюда и следует сильная ограниченность.

Докажем неравенство.
\[
  \exists\ \{n_k\}\colon \yo k\infty\|A_{n_k}\| = \varliminf\|A_n\|
\]
по определению нижнего предела. Так как норма является непрерывной функцией, а последовательность сходится в каждой точке, имеем
\[
  \big\|A(x)\big\| = \yo k\infty\big\|A_{n_k}(x)\big\|\le 
  \yo k\infty \|A_{n_k}\|\cdot \|x\| = 
\]
последнее по определению нормы оператора. Отсюда получаем
\[
  = \varliminf\|A_n\|\cdot \|x\| \imp \|A\|\le \varliminf \|A_n\|.
\] 
\end{Proof}

Давайте приведём один пример, когда последовательность сходится сильно, но не сходится равномерно. В конечномерном пространстве вы такой пример не приведёте. Примеров всё же много. Мы рассмотрим пространство $L_p(E,\mu)$, $1\le p <\infty$ и рассмотрим последовательность 
\[
  E_1\subset E_2\subset \dots,\pau \uN n1E_n  E,\pau 0<\mu(E\dd E_n)\to0.
\]
Например, можно взять $E = [0,1]$ и $E_n = [1/n,1]$. Рассмотрим оператор
\[
  A_nf = \phi \cdot f,\pau \phi_n = \chi_{E_n} = \begin{cases}1,&x\in E;\\0,&x\not\in E.\end{cases}
\]
Мы даже считали норму такого оператора уже.
\[
  \|A_nf - f\|^p = \I{E\dd E_n}{|f|^p}.
\]
Но так как $E\dd E_n\te 0$, то по абсолютной непрерывности интеграла Лебега, сам интеграл стремится к нулю. Таким образом, $A_n\te I$ cсходится сильно к тождественному оператору. Но
\[
  \|A_n-I\| = \|\chi_{(E\dd E_n)}\|_{L_\infty} = 1.
\]
Просто потому, что она будет равняться верхней грани функции на множестве. Значит, не сходится равномерно, причём вообще ни к какому оператору.

\begin{Ut}
  Пусть $E$ "--- банахово пространство. Тогда $M\subset \L(E,F)$ сильно ограничено, если и только если $M$ равномерно ограничено.
\end{Ut}
\begin{Proof}
  Необходимость по теореме Банаха"--~Штенгауза, а достаточность из неравенства
\[
  \big\|A(x)\|\le \underbrace{\|A\|}_{\le C}\cdot \|x\|.
\]
Справа же стоят нормы, равномерно ограниченные.
\end{Proof}
\begin{Lem}
 Пусть $E$ "--- банахово пространство, $\{A_n\}\subset\L(E,F)$ и $A_n\to A$ сильно. Тогда оператор $A$ тоже является ограниченным, то есть $A\in\L(E,F)$.
\end{Lem}
\begin{Proof}
 По теореме Банаха"--~Штенгауза верхняя грань $\sup\limits_n \|A_n\| \le C$, то есть конечна. А значит оператор будет ограничен, поскольку $A(x) = \yo n\infty A_n(x),\ \forall\ x\in E$ и $\|A\| = \varliminf \|A_n\|$.
\end{Proof}

\begin{Def}
  Пусть $K\subset E$ "--- система элементов. Через $M$ обозначаем $M = \ol\sP(K)$ "--- замкнутую линейную оболочку.
\[
\sP(K = \bigg\{ y = \RY i1n\lambda_i x_i\bigg|\lambda_i\in \F,\ x_i\in K\bigg\},
\]
$K\subset E$ называется полной, если $\ol\sP(K)=E$.
\end{Def}

\begin{The}[критерий сильной сходимости]
  Пусть $E,F$ "--- банаховы пространства, $\{A_n\}\in \L(E,F)$. Тогда $A_n\te A$ сильно, если и только если
\begin{roItems}
  \item $\sup\limits_n\|A_n\|<\infty$;
  \item $\exists\ \yo n\infty A_n(x) = A(x)$, $\forall\ x\in K$, где $K\subset E$ "--- некоторая полная система.
\end{roItems}
\end{The}
\begin{Proof}
Необходимость очевидна. Первое условие вытекает из следствия теоремы Банаха"--~Штенгауза. А второе условие прямо из определения вытекает. Так что нужно доказать достаточность.

Обозначим $L = \sP(K)$. Тогда из условия два вытекает $\forall\ x\in K\pau \exists\yo n\infty A_n(x) = A(x)$. Значит, $L$ всюду плотно в $E$. Значит, $\forall\ \e>0$
\[
  \forall\ x\in E\pau \exists\ y\in L\colon \|x-y\|\le \frac\e{4C},
\]
где $C > \sup\limits_n\|A_n\|$ (можно было равно написать, но с делением на ноль надо было бы быть аккуратнее).
Далее 
\[
  \exists N\in\N\colon \forall\ n,m\ge N\pau \big\|A_n(y) - A_m(y)\big\|<\frac\e2.
\]
И запишем неравенство треугольника
\[
  \big\|A_n(x) - A_m(x)\big\|\le \big\|A_n(x) - A_n(y)\big\| + \big\|A_n(y) - A_m(y)\big\| + \big\|A_m(y) - A_m(x)\big\|.
\]
Второе слагаемое $<\frac\e2$, а для первого и третьего слагаемых нужно ещё написать такое неравенство
\[
  \forall\ n,m\ge N\pau \big\|A_n(x) - A_n(y)\big\|\le \|A_n\|\cdot \|x-y\|.
\]
И всё будет меньше $\e$. В силу полноты $F$ $A_n$ будет сходиться и по лемме оператор будет ограничен.
\end{Proof}

\subsection{Функционалы}
С операторами мы закончили. Переходим к функционалам.
\begin{Def}
  $\{f_n\}\subset E^*$ слабо${}^*$ сходится, если 
\[
  \forall\ x\in E\pau \exists\ \yo n\infty f_n(x) = f(x)
\].
 Сходится слабо, если сходится в каждой точке.

Множество функционалов $M\subset E^*$ называется слабо${}^*$ органиченным, если
\[
  \forall\ x\in E\exists\ C_X>0\colon \pau \forall\ f\in M\pau \big|f(x)\big|\le C_X.
\]
\end{Def}
Далее свойства легко переносятся из того, что мы только делали для операторов. И я передоказывать не буду.
\begin{Ut}
  Если $f_n\to f$ по норме, то $f_n\to f$ сходится${}^*$ слабо.
\end{Ut}
\begin{Ut}
  Если $f_n\to f$ сходится слабо${}^*$, то $\{f_n\}$ слабо${}^*$ ограничено.
\end{Ut}
\begin{Ut}
  Пусть $E$ "--- банахово пространство. Тогда $M\subset E^*$ слабо${}^*$ ограничено, если и только если $M$ ограничено по норме.
\end{Ut}
Ну и давайте запишем критерий.
\begin{The}[критерий слабой${}^*$ сходимости]
  Пусть $E$ "--- банахово пространство. Тогда $\{f_n\}\subset E^*$ сходится слабо к $f$, если и только если
\begin{roItems}
  \item $\sup\limits_n\|f_n\|<\infty$;
  \item $\exists\ \yo n\infty f_n(x) = f(x)\pau\forall\ x\in K$, где $K$ "--- некоторая полная система элементов.
\end{roItems}
\end{The}

Пример. Пусть $(X,\rho$ "--- метрическое пространство. $C(X)$ "--- пространство ограниченных непрерывных функций и $\sup$-нормой. Возьмём последовательность $x_n\in X$, $x\in X\colon x\te x$. Для каждой точки рассмотрим функционал Дирака
\[
  \delta_{x_n}(f) : = f(x_n).
\]
Имеем $\forall\ f\in C(X)\pau \delta_{x_n}(f) = f(x_n)\te f(x) - \delta_x(f)$, то есть $\delta_{x_n}\te \delta_x$ сходится слабо${}^*$. При этом $\|\delta_n-\delta\| = 2$ для отрезка  и не стремится к нулю. Значит, последовательность не сходится по норме.
\begin{The}
  Отображение $J\colon E\to E^{**}$, определённое по формуле $J(x) = \delta_x$, где $\delta_x(f) := f(x)\pau\forall\ f\in E^*$ "--- функционал Дирака из второго сопряжённого пространства (если докажем, что он ограничен). Тогда $J$ является изометричным отображением, то есть
\[
  \forall\ x\in E\pau \big\|J(x)\big\| = \|x\|.
\]
\end{The}
Считается, что пространство является подпространством своего второго сопряжённого. Введём перед доказательством определение.
\begin{Def}
  Если $J(E) = E^{**}$, $E$ называется рефлексивным.
\end{Def}
Простанство $L_p(E,\mu)$ рефлексивно, если $1<p<\infty$. Это вытекает из теоремы, которую мы не доказывали, об общем виде функционалов в $L^*_p$.
\begin{Proof}
  Пусть $S^*\subset E^*$ "--- единичный шар, то есть
\[
  S^* = \big\{f\in E^*\big|\|f\|\le 1\big\}.
\]
Тогда $\forall\ x\in E,\forall\ f\in S^*\pau \big|f(x)\big|\le \|f\|\cdot \|x\|\le \|x\|$. Отсюда вытекает, что норма функционала Дирака оценивается $\|\delta_x\|\le 1$. Таким образом, так как $J(x) = \delta_x$, $\big\|J(x)\big\|\le \|x\|$.

Осталось доказать обратное неравенство. Для этого применим теорему Хана"--~Банаха. Берём линейную оболочку фиксированного элемента $x$ и определим функционал
\[
  L = \sP\{x\},\pau f(\lambda x ): = \lambda\|x\|,\pau \|f\|_L=1.
\]
По теореме Хана"--~Банаха существует функционал $g\in E^*$, у которого норма $\|g\| = 1$ и $g(y) = f(y)\pau \forall\ y\in L$. Тогда $\delta_x(g) = g(x) = \|x\|$. Значит, $\|\delta_x\|= \|x\|$. Значит, имеет место нужное равенство.
\end{Proof}

\begin{Def}
  Последовательность элементов $\{x_n\}\subset E$ нормированного пространства сходится слабо (уже без звёздочки, так как это для элементов, а не для функционалов), если 
\[
  \forall\ f\in E^*\pau \exists\ \lim f(x_n) = f(x).
\]

Множество $M\subset E$ слабо ограничено, если
\[
  \forall\ f\in E^*\pau \exists\ C_f>0\colon \forall\ x\in E\pau \big|f(x)\big|\le C_f.
\]
\end{Def}

Одни и те же объекты можно интерпретировать как элементы и как функционалы на пространствах.
\begin{Ut}
  Если $x_n\te x$ по норме, то $x\te x$ слабо.
\end{Ut}
\begin{Ut}
  Если $x_n\te x$ слабо, то $\{x_n\}$ слабо ограничена и $\|x\|\le\varliminf\limits_{n\to\infty} \|x_n\|$.
\end{Ut}
\begin{Ut}
  Множество $M\subset E$ слабо ограничено, если и только если ограничено по норме\footnote{Условие банаховости не нужно, так как функционал Дирака рассматривается на сопряжённом пространстве, а оно всегда банахово.}.
\end{Ut}
\begin{The}[критерий слабой сходимости]
  $\{x_n\}\subset E$ слабо сходится $x_n\te x\in E$, если и только если выполнено два условия
\begin{roItems}
  \item Нормы равномерно  ограничены, то есть $\sup\limits_n\|x_n\|<\infty$;
  \item $\forall\ f\in K\subset E^*\pau \exists\ \yo n\infty f(x_n) = f(x)$, где $K$ "--- некоторая полная система элементов.
\end{roItems}
\end{The}

Давайте ещё один примерчик. Слабая сходимость в $C[a,b]$. Утверждается, что последовательность функций $\{f_n\}\subset C[a,b]$ $f_n\to f\in C[a,b]$ слабо, если и только если
\begin{roItems}
  \item $\sup\limits_n\|f_n\|<\infty$;
  \item $\forall\ x\in [a,b]\pau \exists\ \yo n\infty f_n(x) = f(x)$.
\end{roItems}
\begin{Proof}
  Необходимость вытекает из критерия. Для второго условия нужно в качестве $x_n$ взять функционалы Дирака.

 А достаточность вот как. Мы знаем, что всякий $\alpha\in C^*[a,b]$ является интегралом Римана"--~Стилтьеса $\alpha(f) = \int\limits_a^b f\,dF$. Интеграл Римана"--~Стилтьеса совпадает с интегралом Лебега"--~Стилтьеса, для которого есть теорема о предельном переходе.
\end{Proof}
\begin{Def}
  Нормированное пространство $E$  называется сепарабельным, если в $E$ существует счётная полная система элементов $K = \{x_n\}$.
\end{Def}

Построим метрику в сопряжённом пространстве.
\begin{equation}\label{turbonorma}
  \rho(f,g) = \rY n1\frac1{2^n}\frac{\big|f(x_n) - g(x_n)\big|}{1+\big|f(x_n) - g(x_n)\big|},\pau f,g\in E^*.
\end{equation}
Вот такую обычно пишут в учебниках. Можно и по-другому. А какие свойства выполнены?
\begin{roItems}
  \item $\rho(f,g) = \rho(g,f)$;
  \item $\rho(f,g)\le \rho(f,h)+\rho(h,g)$;
  \item $\rho(f,g) = 0\imp f(x_n) = g(x_n)\pau\forall\ n$. Значит, $\forall\ y\in L\pau f(y) = h(y)$, то есть $f=g$.
\end{roItems}
\begin{Proof}
  Неравенство треугольника.  Берём функцию $\phi(t) = \frac t{1+t}$ для $t\ge0$. Очевидно
\[
  \phi(a+b)\le \phi(a)+\phi(b).
\]
В определение \eqref{turbonorma} в модуле прибавляем и вычетаем $\pm h$ и раскрываем по неравенству треульника для чисел.
\end{Proof}
То есть мы получаем метрическое пространство. Мы будем эту метрику рассматривать лишь на единичном шаре.
\begin{Lem}
  Последовательность $\{f_n\}\subset S^*$ сходится слабо${}^*$ $f_n\te f$, если и только если $f_n\te f$ в $(S^*,\rho)$. Иными словами, слабая сходимость равносильна сходимости по метрике.
\end{Lem}
\begin{Proof}
  \textbf{Необходимость.} Зафиксируем произвольное $\e>0$. $\exists\ m\in \N\colon \frac1{2^m}<\frac\e2$.
\[
  \exists\ N\in\N\colon \forall\ n\ge N,\forall\ k\in\{1,\dots,m\}\pau \big|f_n(x_k) - f(x_k)\big|<\frac\e{2^m}.
\]
Следовательно, в метрике знаменаль отбрасываю и $2^m$ тоже отбрасываю и будет неравенство.
\[
  \rho(f_n,f)\le \RY k1m\big|f_n(x_k) - f(x_k)\big| + \rY k{m+1}\frac1{2^k}<\e.
\]
Значит, доказали необходимость.

\textbf{Достаточность.} Пусть $\forall\ n\ge N\pau \rho(f_n,f)<\e$. Тогда каждое слагаемое в сумме $<\e$, то есть
\[
  \frac{\big|f_n(x_k) - f(x_k)\big|}{1+\big|f_n(x_k) - f(x_k)\big|}<2^k\e.
\]
Значит, $\big|f_n(x_k)-f(x_k)\big|<\frac{2^k\e}{1-2^k\e}$. И $0<\e<\frac1{2^k}<\e$.
\end{Proof}

Ну и осталось только теорему доказать.
\begin{The}
  Пусть $E$ "--- сепарабельное пространство.
  Тогда шар $(S^*,\rho)$ является слабо${}^*$ компактным метрическим пространством (всякая последовательность имеет слабо сходящуюся подпоследовательность).
\end{The}
\begin{Proof}
  Рассмотрим $\{f_n\}\subset S^*$. Докажем, что эта последовательность сходится на множестве $k = \{x_n\}$ "--- счётной и полной системе в $E$.
Берём последовательность $\big\{f_n(x_1)\big\}_{n=1}^\infty$.
Это ограниченная последовательность чисел.
Она имеет сходящуюся подпоследовательность $\big\{f^{(1)}_n(x_1)\big\}$.

Далее берём $\big\{f^{(1)}_n(x_2)\big\}_{n=1}^\infty$.
Это ограниченная последовательность чисел.
Она имеет сходящуюся подпоследовательность $\big\{f^{(2)}_n(x_2)\big\}$.

И так далее.

Берём диагональную последовательность $f_{m_n} = f_n^{(n)}\subset \{f_n\}$. Поскольку последовательность диагональная, она будет сходиться в каждой точке $x_n\in K$. Ну всё, значит мы имеем ограниченную подпоследовательность, сходяющуся в каждой точке полной системы элементов. Очевидно, она сходится слабо $f_{m_n}\to f\in S^*$. И теорема доказана.
\end{Proof}
