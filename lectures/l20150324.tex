\section{Теорема о замкнутом графике}
В начале мы рассмотрим произведение нормированных пространств. Пусть $E,F$ "--- нормированные пространства. Тогда прямое произведение есть 
\[
E\times F = \big\{(x,y)\bigm| x\in E,\ y\in F\big\}.
\]
Тогда на этом прямом произведении заданы операции
\begin{azItems}
\item $(x_1,y_2)+(x_2,y_2) =: (x_1+x_2,y_1+y_2)$;
\item $\lambda (x,y) = (\lambda x,\lambda y)$;
\item и норму введём $\big\|(x,y)\big\| = \sqrt{\|x^2\|\|y\|^2}$.
\end{azItems}
Легко понять, что это именно нормированное пространство.

\begin{Lem}
  Если $E,F$ "--- банаховы пространства, то $E\times F$ "--- банахово пространство.
\end{Lem}
\begin{Proof}
  Пусть у нас задана последовательность Коши $\big\{(x_n,y_n)\big\}$ в прямом произведении $E\times F$. Тогда из определения нормы вытекает, что $\{x_n\}$ является последовательностью Коши в $E$ и $\{y_n\}$ является последовательностью Коши в $F$. И, следовательно, существуют пределы $x = \lim x_n$ и $y = \lim y_n$. А так как
\[
  \big\|(x,y) - (x_n,y)\big\|\le \|x-x_n\| + \|y-y_n\|\te0,
\]
то существует предел $\lim(x_m,y_n)$.
\end{Proof}

Вот такая простая лемма. Вы могли её сами доказать. Это как утверждения о том, что так как $\R$  полно, то и $\R^2$ полно. Доказательство точно такое же.

Дальше у нас будет всюду $E,F$ "--- банаховы пространства, а $L\subset E$ "--- линейное подпространство (не обязательно замкнутое).

Рассмотрим оператор $A\colon L\to F$ (мы будем рассматривать линейные операторы, но не обязательно ограниченные). У этого оператора область определения есть $\dom (A) = L$.
Определим и обозначим график этого оператора
\[
  \gr(A) = \big\{(x,y)\in E\times \bigm| x\in L,\ y\in Ax\big\}.
\]
График является линейным подпространством в $E\times F$.
\begin{Def}
  Линейный оператор $A\colon L\to F$ называется замкнутым, если его график $\gr(A)$ замкнут в $E\times F$.
\end{Def}

  $\gr(A)$  замкнут, если и только если
\[
  \forall\ x_n\to x,\ \forall y_n\to y\pau x\in L,\ Ax\in=y.
\]

\begin{The}[критерий замкнутости ограниченного оператора]
  $A\in \L(L,F)$ является замкнутым, если и только если его область определения $\dom(A) = L$ замкнута в $E$.
\end{The}
\begin{Proof}
  Необходимость. Пусть оператор $A$ замкнут. Пусть $x_n\to x$ и $x_n\in L$. Нам нужно доказать, что $x\in L$ тоже.
  В силу непрерывности оператора $A$ следует, что $y_n = Ax_n\to y\in F$, так как $F$ полно. А в силу того, что оператор $A$ замкнут имеем $x\in L$ и $Ax = y$. Так мы и доказали, что $\ol L=L$.

Обратно. Пусть теперь $\ol L=L$ и $x_n\to x$, $y_n=Ax_n\to y$. Нужно доказать, что $x\in L$ и $Ax = y$. Действительно, так как $A$ является непрерывным, то $A x_n\to Ax$; в силу единственности предела $A x_n\to y$ и $x\in L$ в силу замкнутости $L$.
\end{Proof}

В обе стороны использовали непрерывность оператора $A$.

Но бывают примеры неограниченных замкнутых операторов. Я хочу один такой пример привести.
Пусть мы работаем с пространством Соболева $L = W_1^1[0,1]$, а оператор имеет вид $D\phi(x) = \phi'(x)$ почти всюду на $[0,1]$. Тогда $D\colon L\to \L_1[0,1]$. Докажем, что этот оператор неограниченный. Будем рассматривать $L\subset \L_1[0,1]$ с~соответствующей нормой. Для доказательства неограниченности такого оператора возьмём, например, функции $\phi_n(x)= e^{i\,n\,x},\ n=1,2,\dots$ Тогда $\phi_n'(x) = i\,n\,e^{i\,n\,x}$, а норма $\|phi'_n\| = n\te\infty$ при $\|\varphi_n\|=1$.

Это мы доказали, что оператор $D$ неограничен. Теперь покажем, что он замкнут.  Пусть у нас $\varphi_n\te\varphi$ и $\varphi'_n\te\psi$ в $\L_1[0,1]$. При этом $\phi_n\in L = W_1^1[0,1]$, её можно изменить на множестве меры нуль так, чтобы эквивалентная ей была абсолютно непрерывной.
Будем считать, что $\phi_n$ абсолютно непрерывны. Значит, $\phi_n(x) = \phi_n(0) + \int\limits_0^x\phi'_n(t)\,dt$ почти всюду на $[0,1]$. Так как $\big|\phi_n(0)\big|<c$, можем перейти к пределу под знаком равенства
\[
  \phi(x) = c + \int\limits_0^x\psi(t)\,dt
\]
почти всюду на $[0,1]$.

Это может быть и очень сложный пример. Можно было взять подпространство непрерывных функций. Но полезно рассмотреть именно этот пример, в приложениях часто встречается именно пространство Соболева.

Дальше у нас $S_r(x)$ будет обозначать шар
\[
  S_r(x) = \big\{y\in E\bigm| \|x-y\|\le r\big\},
\]
а $S_r:=S_r(0)$, $S:=S_1(0)$.

\begin{Lem}
 Пусть $E$ "--- банахово пространство, $A\colon E\to F$ "--- линейный оператор; множество функций задано следующим образом $M_c = \big\{x\in E\bigm|\|Ax\|\le c\big\}$ для $c>0$. Тогда
\[
  \forall\ c>0\pau \exists\ r>0\colon S_r\subset \ol M_c.
\]
\end{Lem}
Серьёзная лемма. Мы не предполагаем непрерывность оператора. Из-за этого утверждение становится нетривиальным.
\begin{Proof}
  Воспользуемся теоремой Бэра о категориях: представим $E$ в виде
\[
  E = \uN k1 M_{kc}
\]
для достаточно большого $c$. Но $E$ является полным метрическим пространством, поэтому существует хоть одно $M_{kc}$, которое содержит некоторых шар вокруг некоторой точки, то есть
\[
  \exists\ k\in\N,\ \exists\ r>0,\ \exists\ x\in E\colon S_r(x)\subset \ol M_{kc}.
\]
Множество $M_c$ симметрично, то есть вместе с $x$ лежит всегда $-x$. $-M_{kc}=M_{kc}$, значит, и $-\ol M_{kc} = \ol M_{kc}$. Отсюда следует, что и $S_r(-x)\subset \ol M_{kc}$.

Пусть $y\in S_r$. Тогда $y\pm x\in S_r(\pm x)$ и существуют такие последовательности $x_n^\pm\in M_{kc}$, для которых $x_n^\pm\te y\pm x$. Отсюда $x_n = \frac{x_n^+ + x_n^-}{2}\in M_{kc}$ и $x_n\to y$, значит, $y\in \ol M_{kc}$. Отсюда $S_r\subset \ol M_{kc}$. Ну и в силу однородности можем записать, что $S_{r/k}\subset \ol M_c$.
\end{Proof}
\begin{The}[о замкнутом графике]
  Пусть $E,F$ "--- банаховы пространства и $A\colon E\to F$ "--- замкнутый оператор (замкнутость предполагает линейность оператора). Тогда $A\in\L(E,F)$. 
\end{The}
\begin{Proof}
  По лемме $\forall\ c>0\pau \exists\ r>0\colon S_r\subset \ol M_c$. Пусть $c_n = \frac{c}{2^n}$, $r_n=\frac{r}{2^n}$. Тогда $S_{r_n}\subset\ol M_{c_n}$.

Пусть $x\in S_r$. 
Так как $S_r\in\ol{M}_c$, то $\exists\ x_0\in M_c\colon \|x-x_0\|<r_1$.
Так как $S_{r_1}\in\ol{M}_{c_1}$, то $\exists\ x_1\in M_{c_1}\colon \|x-x_0-x_1\|<r_2$. И так далее по индукции
Так как $S_{r_k}\in\ol{M}_{c_k}$, то $\exists\ x_k\in M_{c_k}\colon \bigg\|x-\RY k1nx_k\bigg\|<r_{k+1}$.

Значит, ряд сходится $x = \rY k0x_k$. Положим $y_k = Ax_k$. Тогда
\[
  \|y_m-y_n\| = \bigg\|\RY k{n+1}mAx_k\bigg\|\le \RY k{n+1}m\|Ax_k\|\le
  \RY k{n+1}mc_k<\frac{c}{2^n}\te0 (m>n).
\]
Значит, $\{y_n\}$ последовательность Коши в $F$, а оно банахово, занчит, $\exists\ y = \lim y_n$. Так как $A$ замкнут, то $Ax = y$.
Следовательно, 
\[
\|Ax\| = \|y\| = \yo n\infty\|y_n\|\le \rY k0\|Ax_k\|\le
  \rY k0 c_k = 2c.
 \]
Таким образом, мы имеем $\forall\ x\in S_r\pau \|A_x\|\le 2c$. Отсюда следует, что $\|A\|\le\frac{2c}{r}$ в силу однородности. Отсюда имеем $A\in \L(E,F)$.
\end{Proof}

\subsection{Эрмитовы сопряжённые операторы}
Операторы в гильбертовых пространствах. Пусть $H$ "--- гильбертово пространство, $\la x,y\ra $ "--- скалярное произведение, $\|x\| = \sqrt{ \la x,y\ra }$ "--- евклидова норма. Пусть $L\subset H$ "--- линейное подпространство. Рассмотрим линейный оператор $A\colon L\to H$, определённый на этом подпространстве.
\begin{Def}
 Оператор $A'\colon M\to H$ называется эрмитово сопряжённым к оператору $A$, если $A'y = z$, где $z$ определяется из уравнения $\forall\ x\in L\pau \la z,x\ra = \la y, Ax\ra$.
\end{Def}
Область определения такого уравнения есть $M = \dom(A') = \big\{y\in H\bigm|\exists\ z\in H\colon \forall\ x\in L\pau \la z,x\ra = \la y,Ax\ra\big\}$.

Если бы оператор $A$ был ограничен, можно было бы определить значительно проще.
\begin{Ut}
  Оператор $A'$ существует, если и только если $\ol L = H$.
\end{Ut}
\begin{Proof}
  Что означает существование сопряжённого оператора? Это тот факт, что уравнение $\la z,x\ra = \la y,Ax\ra$ решается однозначно, то есть из равенства $\la z_1,x\ra = \la z_2,x\ra = \la y,Ax\ra$ для всех $x\in L$ следует $z_1=z_2=0$. Из равенства слeдует $\forall\ z\in L\pau \la z_1-z_2,x\ra=0$. Чтобы это гарантировало равенство $z_1=z_2$, нужно как раз, чтобы $\ol L$ было всюду плотным.
\end{Proof}
\begin{Ut}
$A'$ "--- линейный оператор.
\end{Ut}
\begin{Proof}
Пусть $\forall\ x\in L \la z_1,x\ra = \la y_1,x\ra,\ \la z_2,x\ra = \la y_2,Ax\ra$. Тогда и 
\[
  \forall\ x\in L\pau \la z_1+z_2,x\ra = \la y_1+y_2,Ax\ra.
\]
Значит, $A'(y_1+y_2) = z_1+z_2$.

Если $\forall\ x\in L,\ \forall\ \lambda\in \F\pau \la z,x\ra = \la y,Ax\ra$, то и 
$\la \lambda z,x\ra = \la \lambda y,Ax\ra$. То есть $A'(\lambda y) = \lambda z$.
\end{Proof}
\begin{Ut}
  $A'$ "--- замкнутый оператор.
\end{Ut}
\begin{Proof}
  Пусть $y_n\te y$, $Ay_n = z_n\to z$ и $y_n\in M$.

$\forall\ n\in\N,\ \forall\ x\in L\pau \la z_n,x\ra = \la y_n,Ax\ra$, перейдём к пределу $\la z,x\ra =\la y,Ax\ra$. Отсюда $y\in M$ и $A'y=z$.
\end{Proof}
\begin{The}[достаточное условие ограниченности $A'$]
  Если $A\in\L(L,H)$ и $\ol L = H$, то $A'\in \L(H,H)$ и $\|A'\|=\|A\|$.
\end{The}
\begin{Proof}
  Рассмотрим линейный функционал $f(x) = \la Ax,y\ra$ на пространстве $L$. Тогда по неравенству Коши"--~Буняковского
\[
  \big|f(x)\big|\le \|Ax\|\cdot \|y|\le \|A\|\cdot \|x\|\cdot \|y\|,\pau x\in L,\ y\in H.
\]
Видим, что функционал ограничен. Можно его продолжить по теореме Хана"--~Банаха, то есть считаем, что $f\in H^*$. По теореме Рисса $\exists\ z\in H\colon f(x) = \la x,z\ra$.
 Тогда $\forall\ x\in L\pau \la Ax,y\ra=\la x,z\ra$ влечёт $\forall\ x\in L\pau \la z,x\ra = \la y,Ax\ra$. Тогда
\[
  \forall\ y\in H\pau A'y = z,\ \|A'y\| = \|z\|=\|f\|\le \|A\|\cdot \|y\|.
\]
Отсюда следует, что $\|A'\|\le \|A\|$.

Легко доказать, что $A''=A$ на подпространстве $L$. Отсюда получаем равенство $\|A\|=\|A'\|$.
\end{Proof}

\begin{Def}
  Линейный оператор $A\colon L\to H$, где $L\subset H$ подпространство, называется самосопряжённым, если $A'=A$, то есть $\dim(A') = \dim(A)$ и 
\[
  \forall\ x,y\in L\pau   \la A'y,x\ra = \la y,Ax\ra
\]
\end{Def}
\begin{Def}
  Линейный оператор $A\colon H\to H$ называется эрмитовым, если является самосопряжённым, то есть 
\[
  \forall\ x,y\in H\pau \la Ay,x\ra = \la y,Ax\ra.
\]
\end{Def}
\begin{The}[Хемингера"--~Теплица]
  Если $A\colon H\to H$ является эрмитовым, то $A\in \L(H,H)$.
\end{The}
\begin{Proof}
  Так как $A'=A$, то оператор $A$ является замкнутым. По теореме о замкнутом графике получаем требуемое.
\end{Proof}

Осталось нам примеры привести. Пусть $A\colon L\to l_2$, где $L\subset l_2$ по формуле
\[
  (Ax)_n = \lambda_nx_n,\pau n=1,2,\dots,\ x = \{x_n\};\quad \|x\|_{l_2} = \bigg(\rY n1|x_n|^2\bigg)^{\frac{1}{2}}.
\]
Такой оператор называют диагональным. Область определение этого операторе
\[
  L:=\big\{ z\in l_2\bigm| Ax\in l_2 \big\}.
\]
\begin{Exa}
  Если $\sup |\lambda_n|<\infty$, то $L\subset l_2$ и $A$ ограниченный оператор.
\end{Exa}
\begin{Exa}
  Если $\ol\lambda_n = \lambda_n$, то $A$ "--- самосопряжённый оператор.
\end{Exa}
\begin{Exa}
  Если $\ol\lambda = \lambda_n$ и $\sup|\lambda_n|<\infty$, то $A$ "--- эрмитов оператор.
\end{Exa}
\begin{Exa}
  Если $\sup|\lambda_n|<\infty$ или $\inf|\lambda_n|>0$, то $A$ "--- замкнутый оператор.
\end{Exa}
Последнее вытекает из первого и из критерия замкнутости ограниченного оператора. В случае $\inf$ нужно рассмотреть обратный оператор.
