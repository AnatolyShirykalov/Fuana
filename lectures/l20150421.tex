\section{Фредгольмовы операторы}
Вначале докажем теорему Рисса"--~Шауде. Рассмотрим спектр компактного оператора.

Пусть $\K(E)\subset \L(E)$, где $E$ "--- банахово пространство, причём
\[
  \K(E):=\big\{A\colon E\to E\bigm| A\text{ "--- компактный оператор}.
\]

Тогда верно следующее
\begin{Ut}
  Пусть $A\in\K(E)$. Тогда $\sigma_l(A)\subset \sigma_p(A)\cup\{0\}$.
\end{Ut}
\begin{Proof}
Возьмём $\lambda\in\sigma_l(A)\dd\{0\}$. Найдём последовательность $\{x_n\}\subset E$ таких элементов, что $\|x_n\|=1$ и $A_\lambda x_n\to0$, где $A_\lambda = \lambda I - A$.

Существует сходящая подпоследовательность $A_{x_{n_k}}\to e\in E$. Можем записать
\[
  x_{n_k} = \frac{(A_\lambda x_{n_k} + A x_{n_k})}{\lambda}\to \frac e\lambda.
\]
Ну и затем, поскольку у нас оператор $A$ непрерывный, получается, что применяя оператор $A$ к последнему соотношению, получаем
\[
  A x_{n_k}\to A\frac e\lambda = e\imp A_e = \lambda e.
\]
То есть $\lambda$ является собственным значенем, если $e\ne0$. Но $e$ не равно нулю, так как $\|x_n\|=1$.
\end{Proof}

\begin{Ut}
  Если $A\in\K(E)$, то множество 
\[
  \Lambda_\e = \big\{\lambda\in \sigma_p(A)\bigm| |\lambda|\ge\e\big\}
\]
конечно или пусто.
\end{Ut}
\begin{Proof}
От противного. Пусть существует целая последовательность различных  $\{\lambda_n\}\subset \Lambda_\e$ точек этого множества. По определению этого множества все эти точки являются собственными значениям. Значит, существуют собственными векторы $Ae_n = \lambda_n e_n$, $e_n\ne0$. Из курса линейной алгебры известно, что $e_n$ линейно независимы.

Рассмотрим последовательность линейных оболочек $L_n = \sp\{e_1,\dots,e_n\}\subset L_{n+1}$.
По лемме Рисса о почти перпендикуляре $\exists\ y_n\in L_n\colon \|y_n\|=1$ и $\forall\ x\in L_{n-1}\pau \|y_n-x\|\ge \frac12$. Можно представить в виде комбинации
\[
  y_n = c_n e_n + x_n,\pau x_n\in L_{n-1}.
\]
Тогда у нас $A y_n = \lambda_n c_n e_n + A x_n = 
  \lambda_n y_n - \underbrace{(\lambda_n x_n - A x_n)}_{\in L_{n-1}}$. Теперь давайте мы рассмотрим для $n>k$ вот такую разность
\[
  A y_n - A y_k = \lambda_n y_n - \underbrace{(\lambda_n x_n - A x_n + A x_k)}_{L_{n-1}}.
\]
 Следовательно, если мы теперь возьмём норму, получим
\[
  \forall\ n > k\pau \| A y_n - A y_k\| = \big\|\lambda_n y_n - \underbrace{(\cdots)}_{\in L_{n-1}}\big\|\ge
  \frac{\lambda_n}{2}\ge \frac\e2.
\]
Следовательно, последовательность $\{A x_n\}$ не имеет сходящейся подпоследовательности, что противоречит компактности оператора $A$.
\end{Proof}

\begin{The}[Рисса"--~Шаудера]
  Пусть $E$ "--- банахово пространство, $\dim E = \infty$, $A\in\K(E)$. Тогда
\[
  \sigma(A) = \sigma_p(A)\cup\{0\};\quad
  \forall\ \lambda\in\sigma_p(A)\dd\{0\}\pau \dim E_\lambda<\infty,
\]
где $E_\lambda = \ker A_\lambda$.
\end{The}
\begin{Proof}
  По теореме о границе спектра, граница спектра содержится в предельном спектре. А по свойству один предельный спектр содержится в точечном спектре плюс ноль.
\[
  \dl \sigma(A) \subset \sigma_l(A)\subset \sigma_p(A)\cup\{0\}
\]

 А по свойству два вне всякого круга радиуса $\e$ с центром в нуле на компексной плоскости имеется только конечное число граничных точек.
Легко понять, что такое множество является дискретным, все точки изолированные. Чтобы это понять, возьмём $\lambda\in\sigma(A)\dd \sigma_p(A)$.
% рисунок
% точка ноль с подписью
% точка бесконечность
% чуть кривая кривая, не от нуля, но чуть правее и идёт направо в бесконечность.
% где-то на ней есть точка $\lambda_1$.

Получаем $\sigma(A) = \dl\sigma(A) = \sigma_l(A) = \sigma_p(A)\cup\{0\}$.
Первое мы доказали.


Теперь второе. Пусть $\lambda\in\sigma_p(A)$. Тогда $A\colon E_\lambda\to E_\lambda$ компактный оператор и кратен тождественному, то есть $A|_{E_\lambda} = \lambda I_{E_\lambda}$. А тождественный оператор обратим. А мы доказывали, что компактный оператор в бесконечномерном банаховом пространстве необратим. Значит, $\dim E_\lambda<\infty$.
\end{Proof}

\begin{Def}
  Пусть $E,F$ "--- банаховы пространства, $A\in\L(E,F)$. Оператор $A$ называется фредгольмовым, если выполнены следующие два свойства
\begin{roItems}
\item $\dim\ker A<\infty$,
\item $\codim(\Im A)<\infty$.
\end{roItems}
Здесь обозначения $\Im A\subset F$, $\hat F = E/\Im A$, $\codim (\Im A):=\dim(\hat F)$.

Пространство таких операторов обозначают $\FF(E,F)$.

\end{Def}
\begin{Lem}
Если $A\in\FF(E,F)$, то образ $\Im A\subset F$ замкнут.
\end{Lem}
\begin{Proof}
  Рассмотрим базис $\hat F = \sp\{\hat y_1,\dots,\hat y_m\}$. Берём $y$, берём его смежный класс, пишем линейную комбинацию
\[
  \forall\ y\in F\pau \hat y = \RY k1m c_k\hat y_k.
\]
Отсюда сам $y$ можно представить в виде
\[
  y = \RY k1m c_k y_k + z,\pau z\in Im A.
\]
Отсюда следует, что пространство $F$ представляется в виде прямой суммы $F = \Im A \oplus M$, где
\[
  M = \sp\{y_1,\dots,y_m\}.
\]

Рассмотрим оператор $B\colon E\times M\to F$, заданный по формуле $B(x,y) = Ax + y$. Этот оператор является отображением на, то есть сюрьективным, а так как $A$ непрерывный, то и $B$ будет непрерывным.
По теореме о гомеоморфизме образ любого открытого будет открыт, а образ замкнутого будет замкнут. Таким образом, $B(E\times 0) = \Im A$ замкнуто.
\end{Proof}
\begin{Def}
  Пусть $E$ "--- банахово пространство, $A\in \K(E)$. $B = I-A$ называется классическим оператором Фредгольмана.
\end{Def}
\begin{The}
Если $A\in\K(E)$ и $E$ "--- банахово пространство, то $B\in\FF(E)$.
\end{The}
\begin{Proof}
По теореме Рисса"--~Шаудера имеем $\dim(\ker B)<\infty$. Поскольку оператор $B$ можем рассматривать, как оператор $A_1$ (то есть где $\lambda=1$). Поэтому можем выбрать дополнительное подпространство к ядру (ядро-то замкнуто).
\[
  E = \ker B\oplus M,
\]
$M\subset E$ "--- замкнутое подпространство.

Мы сначала докажем, что образ замкнут, то есть $\Im B$ замкнут. Рассмотри $C = B|_M$, $\ker C=0$ и $\Im C = \Im B$.

По свойству один выполняется такое неравенство
\[
  \|C_x\|\ge c\|x\|,
\]
где $c>0$.


Осталось доказать, что образ $C$ замкнут. Рассмотрим $y\in\ol{\Im C}$. Тогда $\exists\ \{x_n\}\subset E$, для которой $C x_k\to y$. Тогда
\[
  \|x_{k} - x_s\| \le \frac{\big(\|C x_k - y\| + \|y - C x_s\|\big)}c\to 0\pau(k,s\to\infty).
\]
Применили только что полученное неравенство. 

Значит, $\{x_k\}$ "--- последовательность. Следовательно, $\exists\ \lim x_k = x\in M$, так как $M$ замкнутое подпространство. Ну и следовательно, $Cx = y\in Im C$. И мы доказали замкнутость образа $C$.

Осталось доказать, что образ $\Im B$ имеет конечную размерность. Мы уже доказали, что
\[
  \Im B = (\ker B^*)_\perp.
\]
По теореме Рисса"--~Шаудера имеем конечную размерность
\[
  \dim(\ker B^*)<\infty.
\]
Выберем базис $\Im B = \CAP k1m \ker f_k$, где $\{f_k\}$ "--- базис в $\ker B^*$. А отсюда давайте мы возмём теперь $\{e_k\}_{k=1}^m$ "--- биортогональная система $\{f_k\}_{k=1}^n$. Тогда можно рассмотреть подпространство
\[
  L = \sp\{e_1,\dots,e_m\}.
\]
Каждый $x\in E$ можно записать в виде
\[
 z = \RY k1m f_k(x) e_k,\pau y = x - z.
\]
То есть всякий элемент $x$ представляется в виде суммы $x = y+z$. А $y\in \Im (B)$, поскольку если к $y$ применить $f_k$, получим ноль, то есть $y$ принадлежит пересечению ядер. Отсюда следует, что 
\[
 \codim (\Im B) = m<\infty.
\]
\end{Proof}
\subsection{Теоремы Фредгольма}
Из линейной алгебры известны такие альтернативы. Либо система имеет решение для любой правой части, либо имеет ненулевое решение однородной системы. Это "--- альтернатива Фредгольма.

Пусть $E$ "--- банахово пространство, $A$ "--- компакнтый оператор, $B = I- A$ "--- классический оператор Фредгольма, $B^*=  I - A^*$ (он тоже будет фредгольмовым по доказанной теореме).

Обозначим $\dim(\ker B) = n$, $\dim (\ker B^*) = m$.

\begin{The}[Первая теорема Фредгольма]
  Пусть $\{f_k\}_{k=1}^m$ "--- базис решений однородного уравнения $B^*f=0$ (то есть базис ядра $B^*$). Тогда $\exists\ x\in E\colon Bx=y\iff \forall\ k\in\{1,\dots,m\}\pau f_k(y)0$.
\end{The}
\begin{Proof}
  У нас была доказана формула 
\[\Im B = (\ker B^*)_\perp.\]
Ну и поскольку $f_k$ составляют базис ядра сопряжённого оператора
\[\Im B = (\ker B^*)_\perp= \CAP k1m \ker f_k.\]
Таким образом, $y\in \Im B\iff \forall\ k\in\{1,\dots, m\}\pau f_k(y) = 0$.
\end{Proof}

\begin{The}[Вторая теорема Фредгольма]
  Пусть $\{x_k\}_{k=1}^n$ базис решений однородного уравнения $Bx=0$. Тогда $\exists\ f\in E^*\colon B^*f = g\iff \forall\ k=1,\dots,n\pau g(x_k)=0$.
\end{The}
\begin{Proof}
Доказательство этой теоремы, как и первой, сводится к формуле (поскольку $B^*$ фредгольмов оператор, по первой теореме) $\Im B^* = (\ker B)^\perp = \CAP k1n\ker \delta_{x_k}$, где 
$\delta_{x_k}(f) = f(x_k)$, $f\in E^*$.

Из этой формулы вытекает, что $g\in \Im B^*\iff\forall\ k\in\{1,\dots,m\}\pau  \delta_{x_k}(g) = g(x_k) = 0$.
\end{Proof}

\begin{The}[Альтернатива Фредгольма]
  Уравнение $B x = y$ имеет решения для всех $y\in E$, если и только если уравнение $Bx=0$ имеет  только нулевое решение.

Можно сформулировать как альтернативу: либо имеет решение для любого $y$, либо имеет ненулевое решение однородной.
\end{The}
\begin{Proof}
  Необходимость. Пусть у нас неоднородное уравнение имеет решение для любого $y$. Но предположим, что есть ненулевое решение $x_0\ne 0\colon Bx_0=0$. Рассмотрим такие подпространства $L_k = \ker B^k$. Ясно, что $L_1\subset L_2\subset\dots$

Покажем, что включения являются строгими. В самом деле, построим $x_{k+1}$ из $L_{k+1}$, для которых $B x_{k+1} = x_k$ при $k=0,1,\dots$

Если $B^k x_{k+1} = x_0\ne0$, то $x_{k+1}\not\in L_k$.
\[
  \exists\ y_k\in L_k\colon \|y_k\|=1\text{ и } \forall\ x\in L_{k-1}\pau \|y_k-x\|>\frac12.
\]

Если $B=I-A$, то $Ay_k = y_k - By_k$. Тогда
\[
  Ay_k - Ay_s = y_k - \underbrace{(B y_k + y_s - B y_s)}_{\in L_{k-1}}.
\]
Имеем $\|A y_k - A y_s\|>\frac12$ для $k>s$. Стало быть $\{A_{y_k}\}$ не имеет сходящихся подпоследовательностей. А это противоречит компактности оператора $A$. Получили противоречие, необходимость доказана.

Достаточность. Пусть $Bx=0\iff x=0$. Тогда в силу второй теоремы сопряжённое однородное уравнение $B^* f = y$ разрешимо для любого $y\in E^*$. Ну теперь в силу доказанной необходимости этой теоремы (берём в качестве $B$ оператор $B^*$) имеет место $B^*f=0\iff f=0$. Ну а следовательно, по первой теореме $Bx=y$ имеет решение для любого $y\in E$.
\end{Proof}

\begin{The}[Четвёртая теорема Фредгольма]
  Уравнения $Bx=0$ и $B^*f=0$ имеют одинаковое число линейно независимых решений, то есть размерность $\dim(\ker B) = \dim\ker(B^*)$.
\end{The}
\begin{Proof}
Обозначим $\dim\ker B = n$ и $\dim\ker B = m$.

Пусть $\{x_i\}_{i=1}^n$ "--- базис $\ker B$, 
      $\{f_i\}_{i=1}^n$ "--- биортогональная система функционалов. 
Берём $\{g_j\}_{j=1}^m$ "--- базис $\ker B^*$ и 
      $\{y_j\}_{j=1}^m$ "--- соответствующая биортогональная система.
Мы с вами доказывали, что для каждой конечной системы элементов (функционалов) существует биортогональная система функционалов (элементов).

Рассмотрим два случая.
\begin{roItems}
\item Пусть $n<m$. Положим $Cx = Ax + \RY i1n f_i(x)y_i$ и оператор $D = I-C$.
Докажем, что $\ker D= 0$. Пусть $x\in \ker D$. Тогда
\[
  B_x = \RY i1n f_i(x)y_i.
\]
Так как $B^*g_j=0$ ($g_j$ "--- это элементы ядра $B^*$) для $j\in\{1,\dots, m\}$, то получаем определению
\[
  \forall\ j\in\{1,\dots, n\}\pau 0=  B^*g_j(x) = g_j(Bx) = f_j(x).
\]
Значит, $Bx=0$. Ну а поскольку $x\in\ker B$, то он записывается в виде линейной комбинации
\[
  x = \RY i1n f_i(x) x_i  = 0,
\]
так как все $f_i=0$. Вот мы и доказали, что ядро равно нулю.

Ну а теперь заметим, что $C\in \K(E)$, поскольку $A$ компактный, к которому добавляем конечномерный. И следовательно, согласно третьей теореме существует $z\in E$, для которого $Dz = y_m$  (нужно вспомнить, что такое $y_m$; это элементы биортогональной системы к базису ядра сопряжённого оператора). Поэтому $D^*g_m(z) = g_m(Dz) = g_m(y_m) = 1$ в силу биортогональности (как раз она определяется по этому равенству и $g_i(y_j) = \delta_{ij}$).

Если сравнить $Dz$ и $Bz$, получается такое равенство
\[
  g_m(Dz) = g_m(Bz) = B^*g_m(z)=0.
\]

\item Случай $n>m$ разбирается точно так же. Мы должны взять $C^*f = A^*f + \RY j1m f(y_j) f_j$, взять $D^* = I - C^*$. Ну и подобными рассуждениями докажеваем, что $\ker D^*=0$. А затем замечаем, что $C^*$ является сопряжённым к такому оператору: $C x = Ax + \RY j1n f_j(x) y_j$ и $D = I-C$. Если на экзамене вы разберёте первый случай, второй скажете, что «аналогично».
\end{roItems}
\end{Proof}

Теперь ещё пример разберём. Рассмотрим оператор $(Ax)_n = \lambda_n x_n$, здесь $A\colon l_p\to l_p$ для $1\le p\le \infty$. Рассмотрим $N = \big\{n\in\N\bigm|\lambda_n=0\big\}$ и $M = \N\dd N$. Тогда само пространство $l_p$ будет раскладываться в прямую сумму
\[
  l_p = l_p(N) \oplus l_p(M).
\]

Если $A\in\FF(l_p)$, то $\ker A = l_p(N)$ имеет конечную размерность и $\Im A = A\big(l_p(M)\big)$ замкнут. Следовательно, $N$ "--- конечное множество и $A\colon l_p(M)\to \Im (A)$. По теореме Банаха об обратном операторе $A$ имеет ограниченный обратный. Из второго вытекает, что $\sup\limits_{n\in M} |\lambda_n|^{-1}<\infty>$. Ну и поскольку множество $N$ конечно, отсюда следует, что $0$ не является предельной точкой последовательности $\{\lambda_n\}$.
Таким образом, если оператор Фредгольмовым, то $0$ не является предельной точкой последовательности $\{\lambda_n\}$. На самом деле это критерий, но мы не будет это обсуждать.
