\section{Гильбертовы пространства}
Начнём с определений. Сначала определим евклидово бесконечномерное пространство. Обычно математики считают, что евклидово пространство обязательно конечномерное, но нам будет удобно определить иначе.

\begin{Def}
  Пусть $E$ "--- линейное пространство на полем $\F\in\{\R,\C\}$. Для каждой пары элементов $\forall\ x,y\in\ E$ определено скалярное произведение $\la x,y\ra $, если выполнены следующие свойства.
\begin{enumerate}
  \item $\forall\ x,y\in E\pau \la x,y\ra  = \ol{\la y,x\ra }$.

  \item Этот функционал является линейным по первому аргументу
  \[
     \forall\ \lambda_1,\lambda_2\in\F,\ x,y\in E\pau \la \lambda_1 x_1 + \lambda_2 x_2,y\ra  = \lambda_1 \la x_1,y\ra  + \lambda_2 \la x_2,y\ra .
  \]
  \item Скалярный квадрат положительно определён как квадратичная форма, то есть $\la x,x\ra \ge 0$ и $\la x,x\ra =0\iff x=0$.
\end{enumerate}
Пространство вместе со скалярным произведением называется евклидовым пространством. На нём вводится евклидова норма $\|x\| = \sqrt{\la x,y\ra }$ и евклидова метрика $\rho(x,y) = \|x-y\|$.
\end{Def}

Давайте проверим, что нами действительно введена норма.

Следущее неравенство Коши доказал для последовательностей, а Буняковский доказал для интегралов. Иногда ещё называют неравенством Шварца.
\begin{Ut}[Неравенство Коши"--~Буняковского]
  $\forall\ x,y\in E\pau \big|\la x,y\ra |\le \|x\|\cdot \|y\|$.
\end{Ut}
\begin{Proof}
 Берём $z = tx + \lambda y$, $t\in \R$, $\lambda = \frac{\la x,y\ra }{\big|\la x,y\ra \big|}$, $|\lambda|=1$ и раскрываем скалярный квадрат
\[
  \la z,z\ra  = t^2\la x,x\ra  + 1 t \Re\big(\ol \lambda\la x,y\ra \big) + |\lambda|^2\la y,y\ra  =
  t^2\|x\|^2 + 2t\big|\la x,y\ra \big|+\|y\|^2\ge 0.
\]
Это выполнено для любого $t$, значит, есть условие на неотрицательность дискриминанта:
\[
  \big|\la x,y\ra \big|^2\le \|x\|^2\cdot \|y\|^2.
\]
причём отсюда же вытекает, что в случае равенства $z = tx+\lambda y=0$, то есть $x$ и $y$ линейно зависимы.
\end{Proof}

\begin{Ut}
  $\forall\ x,y\in E\pau \|x+y\|\le \|x\| + \|y\|$.
\end{Ut}
\begin{Proof}
  Берём скалярный квадрат и раскрываем по свойствам скалярного произведения.
  \[
    \|x+y\|^2 = \la x+y,x+y\ra  = \la x,x\ra  + 2\Re\la x,y\ra  + \la y,y\ra \le \|x\|^2 + 2\|x\|\|y\|+ \|y\|^2 = \big(\|x\|+\|y\|\big)^2.
  \]
Это неравенство верно тогда, когда верно неравенство Коши"--~Буняковского. Если же $\Re\la x,y\ra =\|x\|\cdot\|y\|$, то $x = \lambda y$, где $\lambda\in\F$ и $\Re\lambda =|\lambda|\ge 0$.
\end{Proof}

Таким образом евклидово пространство является строго нормированным. Это нам пригодиться, когда будем говорить об элементе наилучшего приближения.

\begin{Ut}[Равенство параллелограмма]
  $\|x+y\|^2 + \|x-y\|^2 = 2\big(\|x\|^2+\|y\|^2\big)$.
\end{Ut}
\begin{Proof}
  Доказательство очень простое
\[
  \|x\pm y\|^2 = \|x\|^2\pm\Re\la x,y\ra  + \|y\|^2.
\]
\end{Proof}
Оказывается, это равенство является характеристическим свойством евклидова пространства.
\begin{Ut}
  Если нормированное пространство таково, что выполняется равенство параллелограмма, то пространство евклидово, то есть существует скалярное произведение, порождающее заданную норму.
\end{Ut}

Доказательство этого утверждения можно прочитать в учебнике «Колмогоров"--~Фомин».

$B(X)$ не является евклидовым пространством. Пусть $X = A\sqcup B$, $f(x) = \chi_A(x)$, $g(x) = \chi_B(x)$, Нормы $\|f\|_B:=\sup\limits_x\in X\big|f(x)\big|$. Значит
\[
  \|f\| = \|g\| = \|f+g\| = \|f-g\|=1.
\]
И неравенство параллелограмма не выполняется.

\begin{Ut}
  Непрерывность скалярного произведения $\la x,y\ra $ для $x,y\in E$.
\end{Ut}
\begin{Proof}
  Достаточно доказать, что оно пепрерывно в точке $x_0,y_0$. Применяем неравенство треугольника для модуля.
\[
  \big|\la x,y\ra  - \la x_0,y_0\ra \big|\le \big|\la x-x_0,y_0\big|+\big|\la x_0,y-y_0\big| + \big|\la x-x_0,y-y_0\big|\le
\]
Если снять модули, неравенство превращается в равенство. Это так, отступление.

Теперь применяем неравенство Коши"--~Буняковского
\[
  \le \|x-x_0\|\cdot\|y_0\| + \|x_0\|\cdot \|y-y_0\| + \|x-x_0\|\cdot \|y-y_0\|
\]
Зафиксируем $\e\ra 0$. Возьмём $C\ra \max\big(\|x_0\|,\|y_0\|\big)$ Берём $0\la \max\left\{\delta\la \frac\e{3C},C\right\}$. Тогда для $\|x-x_0\|\la \delta$ и $\|y-y_0\|\la \delta$ имеем
\[
  \le \|x-x_0\|\cdot\|y_0\| + \|x_0\|\cdot \|y-y_0\| + \|x-x_0\|\cdot \|y-y_0\| \la \e.
\]
\end{Proof}

\begin{Ut}[Неравенство Беппо"--~Леви]
  Пусть $L\subset E$ линейное подпространство, $x\in E\dd L$, $d = \rho(x,L) = \int\limits_{y\in L}\|x-y\|$. Утверждается, что
\[
  \|y-z\|\le \sqrt{\|x-y\|^2 - d^2} + \sqrt{\|x-z\|^2-d^2}.
\]
\end{Ut}
Мы это неравенство докажем, используя только свойства скалярного произведения, то есть без геометрических соображений.

\begin{Proof}
  Пусть $u = \frac{ty+z}{t+1}\in L$, $\|x-u\|\ge d$. Рассмотрим скалярный квадрат следующего вида
\[
  \big\|t(x-y)+x-z)\big\|^2 = \big\|(t+1)(x-u)\big\|^2 = (t+1)^2\|x-u\|^2\ge (t+1)^2d^2.
\]
Теперь сам скалярный квадрат раскроем. Я ещё кое-что сразу перенесу из правой части неравенство в левую.
\[
  t^2\big(\|x-y\|^2-d^2\big)+2t\big(\Re\la x-y,x-z\ra -d^2\big) + \big(\|x-z\|^2-d^2\big)\ge0.
\]
Опять получили, как в доказательстве неравенства Коши"--~Буняковского, квадратный трёхчлен. Условие на дискриминант принимает вид
\[
  \big(\Re\la x-y,x-z\ra -d^2\big)^2\le \big(\|x-y\|^2-d^2\big)\big(\|x-z\|^2-d^2\big).
\]
Мы теперь будем использовать это неравенство.
\begin{multline*}
  \|y-z\| = \big\|(x-z)-(x-y)\big\|^2 = \|x-z\|^2 - 2\Re\la x-z,x-y\ra +\|x-y\|^2 =\\
   = \big(\|x-z\|^2-d^2\big) - 2\big(\Re\la x-z,x-y\ra -d^2\big) + \big(\|x-y\|^2-d^2\big)\le\\
  \le \big(\|x-z\|^2-d^2+2\sqrt{\big(\|x-z\|^2-d^2\big)\big(\|x-y\|^2-d^2\big)} + \big(\|x-y\|^2-d^2\big).
\end{multline*}
А это равносильно доказываемому неравенству.
\end{Proof}

\begin{Def}
  Элементы $x,y\in E$ называются ортогональными $x\perp y$, если $\la x,y\ra =$.

  $x\perp L$, если $\forall\ y\in L\pau \la x,y\ra =0$.

  $M\perp L$, если $\forall x\in M,\ \forall\ y\in L\pau \la x,y\ra =0$.
\end{Def}

\begin{Lem}
  Пусть $L\subset E$ линейное подпространство, $x\in E$. Тогда
\[
  \rho(x,L) = \|x-y\|,\ y\in L\iff x-y\perp L.
\]
\end{Lem}
\begin{Proof}
  Пусть $\exists\ z\in L\colon \la x-y,z\ra \ne0$. Рассмотрим такой элемент $u = y+\lambda z$, где $\lambda = \frac{\la x-y,z\ra }{\la z,z\ra }$. Тогда по свойствам скалярного произведения.
\[
  \|x-u\|^2 = \big\|(x-y)-\lambda z\big\|^2 = \|x-y\|^2 - 2\Re\big(\ol \lambda\la x-y,z\ra \big)+|\lambda|^2\la z,z\ra  =
  \|x-y\|^2-\underbrace{|\lambda|^2\la z,z\ra }_{\ne0}.
\]
Отсюда видно, что необходимость доказана.

Достаточность. Пусть $\forall\ z\in L\pau \la x,y,z\ra  = 0$.
Так как $z$ ортогонален, могу заменить $\la x-y,x-y\ra  = \la x-y,x-z\ra $. Тогда
\[
  \|x-y\|^2 = \la x-y,x-y\ra  = \la x-y,x-z\ra \le \|x-y\|\cdot \|x-z\|.
\]
Можно сократить, получим $\forall\ z\in L\pau \|x-y\|\le \|x-z\|$.
\end{Proof}

\begin{The}
  Пусть $L = \sP\{x_1,\dots,x_n\}$, где $x_1,\dots,x_n\in E$ линейно независимы. Пусть также $x\in E\dd L$. Утверждается, что расстояние выражается через определители
\[
  \rho(x,L) = \sqrt{\frac{D(x_1,\dots,x_n,x)}{D(x_1,\dots,x_n)}},\quad
  D(x_1,\dots,x_n) = \det\begin{pmatrix}
   \la x_1,x_1\ra &\dots & \la x_n,x_1\ra\\
   \vdots & \ddots & \vdots
   \la x_1,x_n\ra &\dots & \la x_n,x_n\ra\\
\end{pmatrix}
\]
Этот определитель, составленный из скалярных произведений, называется определителем Грама.
\end{The}

\begin{Proof}
  В строго нормированном пространстве элемент наилучшего приближения единственный, то есть $\exists!\ y\in L\colon\rho(x,L) = \|x-y\|=d$. Запишем такой скалярный квадрат.
\[
  \la x-y,x-y\ra = \la x-y,x\ra = d^2;\quad \la y,x\ra = \la x,x\ra - d^2,\quad \ra y,x_k\ra \la x,x_k\ra.
\]
Отсюда если мы запишем $y$ в виде линейной комбинации $y = \RY k1n\l_kx_k\in L$, $l_k\in\F$, $k=1,\dots,n$, то получим систему уравнений
\[
\begin{cases}
\l_1\la x_1,x_1\ra +\dots + \l_n\la x_n,x_1\ra = \la x,x_1\ra;\\
\dotfill
\l_1\la x_1,x_n\ra +\dots + \l_n\la x_n,x_n\ra = \la x,x_n\ra;\\
\l_1\la x_1,x\ra +\dots + \l_n\la x_n,x\ra = \la x,x\ra;\\
\end{cases}
\]
Так как элемент единственный, система имеет единственное решение, значит, ранг расширенной матрицы системы равен рангу матрицы системы. Определитель расширенной матрицы будет равен нулю. Последний столбец можно представить в виде суммы двух столбцов. Таким образом,
\[
  D(x_1,\dots,x_n)- d^2D(x_1,\dots,x_n) = 0.
\]
Чтобы доказать, что второе слагаемое не равно нулю, нужно применить метод индукции. При $n=1$ верно. Дальше по индуции доказывам, что определитель Грама не равен нулю, когда элементы линейно независимы. Хотя вы можете это помнить из линейной алгебры.
\end{Proof}

\subsection{Гильбертовы пространства}
\begin{Def}
  Полное евклидово пространство $H$ называется гильбертовым пространством.
\end{Def}
Пример: $\L_2(E,\mu)$ является гильбертовым пространством. Можем ввести скалярное произведение по формуле
\[
  \forall\ f,g\in \L_2(E,\mu)\pau \la f,g\ra :=\I E{f\ol g},\quad
                                     \|f\|_{\L_2} = \bigg(\Gint E{|f|^2}\bigg)^{\frac12} = \sqrt{\la f,f\ra}.
\]

Другой пример. Частный случай $\L_2$, а именно $l_2$. Оно тоже является гильбертовым и часто его используют для примеров. Напомню, что это последовательности элементов поля $x = \pos x$, где $x_n\in \F$, для который $\rY n1|x_n|^2<\infty$. Скалярное произведение определяется как
\[
  \la x,y\ra = \rY n1 x_n\ol y_n,\quad \|x\| = \bigg(\rY n1|x_n|^2\bigg)^{\frac12}.
\]
Это частный случай $\L_2$, а именно когда $E=\N$, а мера $\forall\ n\in\N\pau \mu\big(\{n\}\big) = 1$, которую можно продолжить.

\begin{The}[о наилучшем приближении]
  Пусть $H$ "--- гильбертово пространство, $L\subset H$ "--- замкнутое подпространство. Тогда
\[
  \forall\ x\in H\pau \exists!\ y\in L\colon \rho(x,L) = \|x-y\|.
\]
\end{The}
\begin{Proof}
 Главное доказать существование, единственность очевидна.
 Пусть $d= \rho(x,y) = \|x-y\|$. Тогда
\[
 \forall\ n\in\N\pau \exists\ y_n\in L\colon \|x-y_n\|^2 < d^2+\frac1{n^2}.
\]
Теперь применяем неравенство Беппо"--~Леви.
\[
  \forall\ n,m\in\N\pau \|y_n-y_m\|\le \sqrt{\|x-y_n\|^2-d^2} + \sqrt{\|x-y_m\|^2-d^2} = \frac1n+\frac1m.
\]
Пусть $\pos y$ "--- последовательность Коши в $H$. Тогда $\exists\ y = \yo n\infty\in L$ и 
\[
  \|x-y\| = \yo n\infty\|x-y_n\| \le d.
\]
Это неравенство получено из предыдущего и лемме о переходе к пределу в неравенствах из курса мат анализа. Ну а меньше быть не может, значит, равевняется.
\end{Proof}

\begin{The}[об ортогональном ражложении]
  Пусть $H$ "--- гильбертово пространство, $L\subset H$ "--- замкнутое подпространство. Тогда
\[
  L^{\perp} := \big\{x\in H\big|\forall\ y\in L\pau \la x,y\ra =0\}.
\]
Тогда $H = L\oplus L^{\perp}$.
\end{The}
Мы здесь ещё утверждаем, что прямое произведение топологий совпадает с топологией на $H$, это мы доказывать не будем, хотя это совсем просто.
\begin{Proof}
  По теореме о наилучшем приближении 
\[
 \forall\ x\in H\pau \exists!\ y\in L\colon \rho(x,L) = \|x-y\|.
\]
Определим ортогональную проекцию $P(x) = y\in L$, $P\colon H\to L$. Мы можем ещё рассмотреть элемент $z = x-y\perp L$ по доказанной лемме. Поэтому $z\in L^{\perp}$. Следовательно, $x = y+z$, где $y\in L$, а $z\in L^{\perp}$.

Осталось доказать, что подпространства не пересекаются. Для этого нужно доказать единственность разложения. Пусть у нас есть два разложения $x = y_1+z+1 = y_2+ z_2$, Тогда $y_1-y_2=z_2-z_1\in L\cap L^{\perp}$. Значит, эти элементы-разности ортогональны самим себе, то есть равны нулю. Таким образом, $L\cap L^{\perp} = \{0\}$.
\end{Proof}
\begin{Sl}
   Пусть $H$ "--- гильбертово пространство, $L\subset H$ "--- линейное подпространство. Тогда $L$ всюду плотно в $H$, если и только если $L^\perp =0$.
\end{Sl}
\begin{Proof}
  Докажем необходимость. Если $L$ всюду плотно в $H$, то по определению $\ol L=H$. Значит, всякий элемент из $H$ является пределом последовательности элементов из $L$, то есть 
\[
  \forall\ x\in H\pau \exists\ x_n\in L\colon x_n\te x.
\]
Тогда в силу непрерывности скалярного произведения 
\[
  \forall\ y\in L^\perp\pau \la x,y\ra = \yo n\infty \la x_n,\underbrace{y}_{\ne0}\ra = 0.
\]
Поэтому отсюда вытекает, что $L^\perp\subset H^\perp$, ну а $H^{\perp} = \{0\}$. И необходимость доказана.

Достаточность. Пусть $L^\perp = \{0\}$. Ну а $(\ol L)^\perp\subset L^\perp = 0\imp (\ol L)^\perp=0$. Значит, $H = \ol L\oplus {(\ol L)^\perp} = \ol L$.
\end{Proof}

В необходимости достаточно евклидовости пространства, а в достаточности существенна полнота гильбертова пространства.
Пример на случай, когда для евклидова пространства эта достаточность не верна. Рассмотрим $C[0,1]\subset \L_2[0,1]$ (здесь, конечно, берётся мера Лебега на отрезке $[0,1]$). $E = C[0,1]$ евлидово, если рассматривать скалярное произведение и норму из $\L_2$. Теперь рассмотрим множество многочленов 
\[
  M = \Big\{P(x) = \RY k1na_kx^k\bigg| P\perp\chi_{[0,1/2]}\bigg\}.
\]
Ясно, что $M\subset C[0,1]$.
И его ортогональное дополнение $M\subset C[0,1]$ и $M^\perp=0$ в $C[0,1]$, так как в $\L_2$ ортогональным дополнением будет прямая, натянутая на $\chi_{[0,1/2]}$. $M$ не является всюду плотным в $C[0,1]$, если бы являлось, то и в $\L_2$ тоже, а это не верно.
