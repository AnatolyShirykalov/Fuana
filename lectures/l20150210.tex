\section{Пространства сходимости}
Для того, чтобы нам теорию обобщённых функций рассмотреть, сегодняшняя лекция будет о некоторых специальных спространствах.

Пространства будут определяться с помощью определения того, что значит последовательность сходится. Введём понятие абстрактной сходимости.

Пусть $E$ "--- линейное пространство над $\F\in\{\R,\C\}$. $\zeta$ "--- множество всех сходящихся последовательностей. Предполагается, что задано какое-то множество последовательностей, которые мы называем сходящимися.
\begin{Def}
Пара $(E,\zeta)$ называется пространством сходимости, если выполнены следующие аксиомы:
\begin{enumerate}
\item $\forall\ \{x_n\}\in\zeta\pau\exists!\ x = \lim x_n$;
\item если $\exists\ n_0\in\N\colon \forall\ n\ge n_0\pau x_n=x$, то $\{x_n\}\in\zeta$ и $\lim x_n = x$;
\item если $\{x_n\}\in\zeta$, то $\forall\ \{x_{n_k}\}\in\zeta$ и $\lim x_{n_k} = \lim x_n$;
\item если $\{x_n\}$, $\{y_n\}\in\zeta$, то $\{x_n+y_n\}\in\zeta$ и $\lim(x_n+y_n) = \lim x_n + \lim y_n$;
\item если $\{x_n\}\in\zeta$ и $\{\lambda_n\}\in \zeta_\F$, то $\{\lambda_n\,x_n\}\in\zeta$ и $\lim\lambda_n\,x_n = \lim\lambda_n\cdot\lim x_n$.
\end{enumerate}
\end{Def}
Эти аксиомы естественные. Они выполняются в нормированных и метрических линейных пространствах.
\begin{Def}
  Отображение $f\colon E\to F$ одного пространства сходимости в другое называется непрерывным (секвенциально), если 
\[
  \forall\ \{x_n\}\in\zeta_E\pau \big\{f(x_n)\big\}\in\zeta_F\text{ и }\lim f(x_n) = f(\lim x_n).
\]
\end{Def}
Естественное определение непрерывности по последовательностям.

Ну давайте ещё одно определение дам.
\begin{Def}
Пространство сходимости $(E,\zeta)$ называется регулярным, если для всякой двойной последовательности $\{x_{nk}\big\}$, для которой существует предел $\exists\ \lim\limits_kx_{nk} = x_n$ и $\exists\ \lim\limits_n x_n = x$, существует $\exists\ k_n\to \infty$ такая, что $\yo n\infty x_{n\,k_n} = x$.
\end{Def}

\begin{Lem}\label{MetrReg}
  Метрическое линейное пространство $(E,\rho)$ является регулярным пространством сходимости.
\end{Lem}
В частности, это верно и для нормированных.
\begin{Proof} Сходимость там уже задана.
Нужно доказать регулярность. По условию задана двойная последовательность, у которой есть пределы по строкам и существует предел этих пределов.

Обозначим квазинорму $\|x\| = \rho(x,0)$ "--- расстояние от $x$ до нуля. Хотя мне квазинорма не нужна.

Запишем наше условие:
\[
  \lim\limits_{k}\rho(x_{nk},x_n) = 0;\qquad \lim\limits_n\rho(x_n,x) = 0.
\]
Для фиксированного $n$ имеем $\exists\ k_n\to\infty\colon \rho(x_{n{k_n}},x_n)<\frac1n$. Следовательно, расстояние
\[
  \rho(x_{nk_n},x)\le \rho(x_{nk_n},x_n) + \rho(x_n,x)\te0.
\]
\end{Proof}

То есть некоторая диагональная последовательность стремится к $x$.

\begin{Def}
Последовательность $\{x_n\}$ называется безусловно суммируемой, если для каждой подпоследовательности сходится ряд, то есть $
  \forall\ \{x_{n_k}\}$ сходится ряд $\rY k1x_{n_k}$, то есть последовательность частичных сумм лежит в $\zeta$.
\end{Def}
Для последовательности действительных чисел "--- это абсолютная сходимость ряда из этих чисел. Для комплексных чисел чуть по-другому.

\begin{Def}
  В пространстве сходимости $(E,\zeta)$ выполняется аксиома полноты, если
\[
  \forall\ \{x_n\}\in\zeta\colon \lim x_n = 0\pau \exists\ \{x_{n_k}\}\text{ безусловно суммируемая}.
\]
\end{Def}
Это определение вводится для того, чтобы в последствии доказать полноту сопряжённого пространства.

\begin{Lem}\label{ful2aksful}
  Если метрическое линейное пространство $(E,\rho) $ полно, то в нём выполняется аксиома полноты.
\end{Lem}
\begin{Proof}
Нам задана последовательность, которая стремится к нулю. Здесь нам понадобися квазинорма $\|x\| = \rho(x,0)$. Раз последовательность стремится к нулю, выполняется следующее свойство
\[
  \lim\|x_n\|=0.
\]
Отсюда следует, что существует такая подпоследовательность $\{n_k\}$, для которой $\|x_{n_k}\|<\frac1{2^k}$. 

Давайте докажем теперь, что ряд $\rY k1x_{n_k}$ сходится. В самом деле, берём частичные суммы $S_n = \RY k1nx_{n_k}$ этого ряда и рассматриваем
\[
  \|S_m-S_n\| = \bigg\|\RY k{n+1}mx_{n_k}\bigg\|\le
  \RY k{n+1}m\|x_{n_k}\|<\frac1{2^n}.
\]
Значит, последовательность частичных сумм является последовательностью Коши. А так как пространство полное, то значит, существует предел $\exists\ S_n$. Но нам нужно доказать больше, что всякий подряд тоже сходится. Это доказывается аналогично с помощью того же самого неравенства.
\end{Proof}

Давайте теперь приведём плохой пример. $\mathcal K(\R) = C_0(\R)$ "--- по-разному обозначают множество непрерывных функций $\phi\colon \R\to\R$, определённых на всей прямой, у которых компактный носитель, то есть $\supp(\phi) \spp \R$.
\[
  \supp(\phi):=\overline{\big\{x\in\R\bigm|\phi(x)\ne0\big\}}.
\]
\begin{Def}
  $\phi_n\te\phi$, если
\begin{roItems}
\item $\phi_n \rsh[R]n\phi$;
\item $\exists\ K\spp\R\colon \supp(\phi_n)\subset K$.
\end{roItems}
\end{Def}

Пусть от противного пространство $K(\R)$ метризуемо. Сходимость $\zeta$ мы определили. Определим ещё одну $\zeta' = \big\{\{\phi_n\}_{n=1}^\infty\subset C_0(\R)\bigm| \sup\limits_\R|\phi_n-\phi_m|\xrightarrow{n,m\to\infty}0,\ \exists K\Subset \R\colon \forall\ n\in\N\pau \supp \phi_n\subset K\big\}$.
Мы знаем, что всякая равномерная последовательность Коши является равномерно сходящейся, поэтому $\zeta'$ действительно пространство сходимости, причём $\forall\ \{\phi_n\}_{n=1}^\infty\in\zeta'\pau \lim\limits_n\phi_n \in C_0(\R)$. При этом $\zeta\subset \zeta'$ и соответствующие пределы совпадают.  Таким образом, $\forall\ \{\phi_n\}_{n=1}^\infty\in\zeta\pau \lim\limits_n\phi_n\in C_0$, что означает полноту пространства $C_0(\R)$ относительно предполагаемой метрики. Значит, по лемме \ref{ful2aksful} в~этом пространстве $\mathcal K(\R)$ выполняется аксиома полноты. Но однако, это пространство не является метрическим пространством, поскольку сходимость не является регулярной (противоречие с леммой \ref{MetrReg}). То есть не существует метрики, чтобы сходимость по метрике совпадала с данной.

\begin{Proof}
Рассмотрим функцию
\[
\eta(x) = \begin{cases}
  1-|x|,&|x|\le1;\\
 0,&|x|>1.
\end{cases}
\]
И построим последовательность $\phi_{nk}(x) = \frac1k\eta\left(\frac xn\right)$. Выполнены условия из определения сходимости, то есть $\lim\limits_{k\to\infty}\phi_{nk}(x) = 0$ в $\mathcal K(\R)$. Но если у нас есть последовательность $\{k_n\}$, то $\phi_{n\,k_n}\cancel{\te}0$ в $\mathcal K(\R)$, потому что $\supp\phi_{nk} = [-n,n]$ и когда $n$ растёт, носители расширяются и не находятся на одном компакте. 

Такиим образом, $\mathcal K(\R)$ не является метрическим по лемме \ref{MetrReg}.
\end{Proof}

Этот пример характерный. Мы увидим, что пространство основных функций также не является метрическим.

\begin{Def}
Пусть $(E,\zeta)$ "--- пространство сходимости. Через $(E',\zeta')$ будем называть сопряжённое пространство сходимости. Здесь $E'$ "--- множество всех линейных непрерывных функционалов $f\colon E\to\F$, а сходимость определяется так: $f_n\to f$, если $\forall\ x\in E\pau \lim f_n(x) = f(x)$.
\end{Def}

\begin{Lem}
Пусть задана двойная последовательность комплексных чисел $\{a_{m,n}\}\subset\F$, такая, что 
\begin{roItems}
  \item $\forall\ m\in\N\pau \exists\ \lim\limits_{n}a_{m,n} = b_m$,
  \item $\exists\ \e>0\colon\forall\ m\in\N\pau|b_m|>\e$,
  \item $\forall\ n$ ряд $\rY m1a_{m,n}$ абсолютно сходится.
\end{roItems}
Тогда $\exists\ m_l\to\infty$ и $\exists\ n_k\to\infty$, для которых
\[
   \lim\limits_k\bigg|\rY l1a_{m_l,n_k}\bigg|=\infty.
\]
\end{Lem}
\begin{Proof}
Мы не будем доказывать для комплексных. Это доказательство сводится к случаю действительных чисел. По условию нам уже задано некоторое $\e>0$. Пусть также $a_{m,n}\in\R$.
Найдём из первых двух свойств $n_1\in\N\colon \forall\ n\ge n_1\pau |a_{1,n}|>\e$. $\forall\ k\ge 2$ найдём $n_k>n_{k-1}\colon \forall\ n\ge n_k\pau |a_{k,n}|>\e$. Заметим, что $\forall\ m\le k\pau n_m\ge n_k\imp |a_{m,n_k}|>\e$.
Отсюда
\[
  \rY m1|a_{m,n_k}|\ge \RY m1k|a_{m,n_k}|>k\,\e.
\]
Теперь берём предел
\[
  \lim\limits_k\rY m1|a_{m,n_k}|=\infty\imp \lim\limits_k\rY m1 a_{m,n_k}^+=\infty \text{ или }
\lim\limits_k\rY m1 a_{m,n_k}^-=\infty.
\]
где $a^\pm = \max\{\pm a,0\}$. 
\end{Proof}

\begin{The}
  Если в $(E,\zeta)$ выполнена аксиома полноты, то $(E',\zeta')$ является полным.
\end{The}
\begin{Proof}
  От противного, пусть $f_n\to f$ и $f_n\in E'$, однако $f\not\in E'$. Придём к противоречию при помощи леммы. Функционал линейный, а то, что он не из $E'$, значит, он не является непрерывным, причём во всех точках (в силу линейности), например не является непрерывным в нуле. Это значит, что
\[
  \exists\ \e>0,\ \exists\ x_m\to 0\colon \big|f(x_m)\big|>\e.
\]
Значит, $\{x_m\}$ "--- безусловно суммируемая последовательность. Теперь используем лемму. Пусть $a_{mn}:=f_n(x_m)$. Легко проверить, что ряды по $n$ безусловно сходится, ну и все остальные условия леммы будут выполнены. Поэтому существуют такие подпоследовательности $m_l\to\infty$ и $n_k\to\infty$, такие, что 
\[
  \yo k\infty\bigg|\rY l1 a_{m_ln_k}\bigg|=\infty.
\]
Пусть $x = \rY l1 x_{m_l}$. Так как последовательность $x_n$ безусловно суммируемая, то ряд сходится к элементу $x$. Тогда в силу того, что $f_n$ сходится в каждой точке
\[
  \big|f(x)\big| = \yo k\infty\big|f_{n_k}(x)\big| = 
  \yo k\infty\bigg|\rY l1 f_{n_k}(x_{m_k})\bigg| = \infty.
\]
Вынести сумму смогли, так как $f_{n_k}\in E'$ и в частности непрерывны. Значит, у нас функционал оказался равен бесконечности.
\end{Proof}

Напомню определение полунормы.
\begin{Def}
  $p\colon E\to\R_+$ полунорма, если
\begin{roItems}
  \item $\forall\ \l\in\F,\ \forall\ x\in E\pau  p(\lambda\,x) = |\l|p(x)$.
  \item $\forall\ x,y\in E\pau p(x+y)\le p(x)+p(y)$.
\end{roItems}
\end{Def}

\begin{Def}
Пусть $(E,\mathcal P)$, где $E$ "--- линейное пространство, где $\mathcal P$ "--- система полунорм. Пара называется полинормированным пространством, если из того, что $\forall\ p\in\mathcal P\pau p(x)=0$ следует, что $x=0$.
\end{Def}
\begin{Def}
 $x_n\to x$ в $(E,\mathcal P)$, если $\forall\ p\in\mathcal P\pau \lim\limits_n p(x_n-x)=0$, то есть
\[
  \forall\ p\in\mathcal P,\ \forall\ \e>0\pau \exists N\in\N\colon \forall n\ge N\pau p(x_n-x)<\e.
\]
\end{Def}
\begin{Def}
$\{x_n\}$ "--- последовательность Коши в $(E,\mathcal P)$, если $\forall\ p\in\mathcal P\pau \lim\limits_{m,n}p(x_n-x_m)=0$, то есть
\[
  \forall\ p\in\mathcal P\pau \exists\ N\in\N\colon\forall\ n,m\ge N\pau p(x_n,x_m)<\e.
\]
\end{Def}

Например, сопряжённое пространство $(E',\zeta') $ является полинормированным пространством относительно системы полунорм $p_x(f): = \big|f(x)\big|$. Очевидно, что так как модуль обладает определёнными свойствами, то это будут полунормы. А если все модули равны нулю, то и $f\equiv0$.

\begin{Def}
  Полинормированное пространство $(E,\mathcal P)$ называется счётно нормированным, если система полунорм счётная, задаётся последовательность полунорм $\mathcal P = \{p_n\}$.
\end{Def}

\begin{Lem}
  Пусть $(E,\mathcal P)$, $\mathcal P =\{p_n\}$ "--- счётно нормированное пространство. Тогда сходимость в этом пространстве $(E,\mathcal P)$ равносильна сходимости относительно метрики
\[
  \rho(x,y) := \rY n1 \frac1{2^n}\frac{p_n(x-y)}{1+p_n(x-y)}.
\]
\end{Lem}
Можно, конечно, и другую формулу придумать. Но нам достаточно её, чтобы доказать, что каждое счётно нормированное пространство является метрическим.

\begin{Proof}
  Надо доказать сначала, что это метрика. Мы с вами уже сталкивались с ней, я просто повторю.
\begin{enumerate}
\item $\rho(x,y) = \rho(y,x)$ очевидно;
\item $\rho(x,y)\le\rho(x,z) + \rho(z,y)$, а это уже нужно доказывать.

Имеем $\phi(t) = \frac t{1+t}$ возрастает, $\phi(t+s)\le \phi(t) + \phi(s)$. Отсюда и вытекало у нас неравенство треугольника.

\item $\rho(x,y)=0\imp \forall n\pau p_n(x-y)=0\imp x=y$.
\end{enumerate}
Таким образом, эта формула определяет некоторую метрику. Нужно ещё проверить, что относительно этой метрики операции сложения и умножения на число непрерывны. Я не буду это проверять, это достаточно просто делается.

Значит, $(E,\rho)$ "--- метрическое линейное пространство. Покажем, что сходимости равносильны.

Пусть $x_n\to x$ в $(E,\mathcal P)$. Берём $\e>0$, тогда $\exists\ m\colon \frac\e2 >\frac1{2^m}$. Так как $p_k(x_n-x)\to0$, то 
\[\exists\ n_k\colon \forall n\ge n_k\pau  p_k(x_n-x)<\frac\e{2m}.
\]
Возьмём $N:=\max\limits_{1\le k\le m}\{n_k\}$. Тогда
\[
  \rho(x_n,x) = \rY k1\frac1{2^k}\frac{p_k(x_n-x)}{1+p_k(x_n-x)}\le
  \RY k1mp_k(x_n-x) + \rY k{m+1}\frac1{2^k} < \e.
\]
Разбили сумму на две. В одной дробь больше единицы, в другой "--- меньше.

Теперь обратно нужно доказать. Пусть $\rho(x_n,x)\to0$. Тогда 
$\frac1{2^k}\frac{p_k(x_n-x)}{1+p_k(x_n-x)}\to0$, то есть
\[
  \forall \e>0\ \exists\ N\in\N\colon \forall\ n\ge N\pau
  \frac1{2^k}\frac{p_k(x_n-x)}{1+p_k(x_n-x)}<\e.
\]
Фиксируем число $k$ Тогда $\forall n\ge N\pau p_k(x_n) <\frac{\e\,2^k}{1-\e\,2^k}$. Поэтому последовательность у нас сходится в счётно нормированном пространстве.

\end{Proof}
Можно доказатель, что последовательности Коши относительно счётной системы полунорм и последовательности Коши относительно метрики.

\begin{Def}
Пусть $(E,\mathcal P_E)$ и $(F,\mathcal P_F)$ "--- полинормированные пространства. Линейное отображение $f\colon E\to F$ называется ограниченным, если
\[
  \forall\ p\in\mathcal P_F\pau \exists\ p_1,\dots,p_n\in\mathcal P_E,\ \exists\ c>0\colon 
  p\big(f(x)\big)\le c\big(p_1(x) + \dots + p_n(x)\big).
\]
\end{Def}
Это определение согласуется с определением ограниченных операторах в нормированных пространствах.

\begin{The}
  Пусть $(E,\mathcal P)$ "--- счётно нормированное пространство.  Тогда линейное отображение $f\colon E\to F$ ограничено, если и только если $f$ непрерывно.
\end{The}
\begin{Proof}
  Необходимость очевидная. Если ограничены, то есть выполнено неравенство; в нём если правая часть стремится к нулю, то и левая тоже.

Нужно доказать достаточность.  Пусть отображение непрерывно. Пусть $q_n(x) = \RY k1n p_k(x)$, где $\mathcal P_E = \{p_n\}$ "--- заданная счётная система полунорм. Если $f$ не являетя ограниченным, то существует $p\in\mathcal P_F$ и последовательность $\{x_n\}$, такие, что 
\begin{equation}\label{nerav}
p\big(f(x_n)\big)\ge n\,q_n(x_n).
\end{equation}
Пусть у нас $y_n=\frac1{\sqrt nq_n(x_n)}\cdot x_n$. Рассмотрим такие элементы. В силу неравенства \eqref{nerav} получаем $p\big(y_n)\big)>\sqrt n$. То есть $p_k(y_n)\to0$, но $f(y_n)\cancel{\to}0$, поскольку
\[
  p_k(y_n) = \frac{p_k(x_n)}{\sqrt{n} q_n(x_n)}\le \frac1{\sqrt{n}}\to0.
\]
\end{Proof}

Нужно ещё привести примеры. Функция $\phi\colon \R^m\to\F$ называется бесконечно дифференцируемой, если существуют все частные производные
\[
  \dl^\alpha\phi(x) = \frac{\dl^{|\alpha|}\phi(x)}{\dl^{\alpha_1}x_1\dots\dl^{\alpha_m}x_m},\qquad \alpha = (\alpha_1,\dots,\alpha_m),\ |\alpha| = \RY k1m\alpha_k.
\]
Через $C_0^\infty(X)$ "--- пространство бескончно дифференцируемых функций, у которых $\supp(\phi)\Subset X$. На этом пространстве вводится счётная система полунорм
\[
  p_k(\phi) = \sum\limits_{|\alpha|\le k}
  \sup\limits_{x\in X} \big|\dl^{\alpha}\phi(x)\big|.
\]
Это будет счётно нормированное пространство. Сходимость в этом пространстве будет определяться также метрикой
\[
  \rho(\phi,\psi) = 
  \rY k0\frac1{2^k}\frac{p_k(\phi-\psi)}{ 1 + p_k(\phi-\psi)}.
\]

Полное счётно нормированное пространство называется пространством Фреше. Пример "--- пространство Фреше.
