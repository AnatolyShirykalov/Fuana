\section{Ортонормированные системы}
Сначала мы докажем одно следствие теоремы об ортогональном разложении. Оно опирается на ещё несколько теорем.
\begin{The}[Рисса]
  Пусть $H$ "--- гильбертово пространство, $\alpha\in H^*$. Тогда $\exists!\ y\in H\colon$
\begin{enumerate}
  \item $\forall\ x\in H\pau \alpha(x) = \la x,y\ra$;
  \item $\|\alpha\| = \|y\|$.
\end{enumerate}
\end{The}
\begin{Proof}
	Обозначим через $L$ ядро этого функционала $L = \ker(\alpha) = \big\{x\in H\big|\alpha(x) = 0\big\}$. Так как функционал ограниченный, он непрерывный,  $L\subset H$ замкнутое подпространство.
	Если $L^\perp=0$, то по теореме об ортогональном разложении $L=H\imp\alpha=0$. Тогда можно взять $y=0$.

	Теперь предположим, что $L^\perp\ne0$. Тогда $\exists\ z\in L^\perp\colon \|z\|=1$ (потому что ортогональное дополнение имеет элемент неравный нулю, возьмём его и нормируем). Положим $u = \alpha(x)z - \alpha(z)x$. Очевидно, $\alpha(u) = 0$. Значит, $u\in L$. Поэтому
	\[
		0=\la u,z\ra = \alpha(x)\la z,z,\ra - \alpha(z)\la x,z\ra = \alpha(x) - \la x,y\ra,
	\]
	где элемент $y = \ol{\alpha(z)} z$. Значит, мы нашли элемент $y$, для которого $\forall\ x\in H\pau \alpha(x) = \la x,y\ra$.

Докажем его единственность. Пусть $\forall\ x\in H\pau\la x,y_1\ra = \la x,y_2\ra$. Тогда $\la x,y_1-y_2\ra=0$ и, значит, $y_1-y_2=0$.

Ну и теперь осталось доказать последнее условие теоремы. В силу неравенства Коши"--~Буняковского, имеем
\[
	\big|\alpha(x)\big| = \big|\la x,y\la\big|\le \|x\|\cdot\|y\|\imp\|\alpha\|\le \|y\|,\pau x = \frac{y}{\|y\|}\imp\|\alpha\| = \|y\|.
\]
\end{Proof}

Отсюда вытекает
\begin{Sl}
  $H^*$ изоморфно $H$.
\end{Sl}

Пример 1. $H=l_2$, $x = \pos x, y = \pos y\in l_2$, $\|x\| = \sqrt{\la x,x}$, $\la x,y\ra = \rY n1x_n\ol y_n$. Тогда
\[
	\forall\ \a\in l_2^*\pau \exists\ y\in l_2\colon \forall\ x\in l_2\pau \alpha(x) = \la x,y\ra, \|\a\| = \|y\|_{l_2}.
\]

Такой же пример можно привести и для $L_2(E,\mu)$, $\|f\| = \left(\I E{|f|^2}\right)^{\frac12}$, $\la f,g\ra = \I E{f\ol g}$ для $f,g\in L_2(E,\mu)$. Тогда
\[
  \forall\ \a\in L_2^*(E,\mu)\pau\exists!\ g\in L_2(E,\mu)\colon \alpha(f) = \I E{fg}.
\]
Единственность понимается специальным образом. С точностью до эквивалентности. Сопряжение от $g$ можно было бы поставить, но ведь $\ol g$ тоже лежит в $L_2(E,\mu)$.

Мы даже не будем требовать гильбертовость сейчас.
\begin{Def}
	Система элементов $\pos e\subset E$ евклидова пространства называется ортонормированной, если
	\[
		\la e_n,e_m\ra = 0\pau (n\ne m),\quad \la e_n,e_n\ra=1.
	\]
	Просто ортогональной, если только первое условие.

	Система $\pos e$ называется \textbf{тотальной}, если
	\[
	  \forall n\in\N\pau \la x,e_n\ra = 0\imp x=0.
	\]
	Иногда это называют полнотой системы $\pos e$.
\end{Def}

\begin{Def}
Пусть $x\in E$. Обозначим $с_n = \la x,e_n\ra$. Эти числа $c_n$ называются коэффициентами Фурье. Тогда каждому элементу $x$ соответствует ряд Фурье
\[
	x\sim \rY n1 c_ne_n.
\]
 Обозначим также $S_n = \RY k1nc_ne_n$ "--- частичные суммы ряда Фурье.
\end{Def}
Вообще говоря, этот ряд не сходится к элементу $x$. Для того, чтобы элемент сходился, нужно, чтобы система была полной.

Давайте несколько свойств перечислим.
\begin{Ut}[Неравенство Бесселя]
	$\|x\|^2\ge \rY n1|c_n|^2$.
\end{Ut}
\begin{Proof}
	Доказывается очень просто. Вычисляя скалярный квадрат по свойствам скалярного произведения, мы получим
	\[
		\|x-S_n\|^2 = \la x-S_n,x-S_n\ra = \la x,x\ra - 2\Re\la x,S_n\ra + \la S_n,S_n\ra = 
		\|x\|^2 - \RY k1n|c_n|^\ge0.
	\]
	Отсюда вытекает неравенство Бесселя, если перейти к пределу по $n\to\infty$ в неравенстве.
\end{Proof}
\begin{Ut}[Равенство Парсеваля]
	Равенство в неравенстве Бесселя выполняется тогда и только тогда, когда ряд Фурье сходится к элементу $x$, то есть
	\[
		\|x\|^2 = \rY n1|c_n|^2\iff \|x-S_n\|\searrow0
	\]
	в силу доказанного равенства.
\end{Ut}
\begin{Ut}[Обобщённое равенство Парсеваля]
	Равенство Парсеваля выполняется, если и только если выполняется обобщённое равенство
	\[
		\forall\ x\in E\pau \|x\|^2 = \rY n1|c_n|^2\iff \forall\ x,y\in E\pau \la x,y\ra = \rY n1 c_n\ol d_n,
	\]
	где $c_n = \la x,y\ra$, $d_n = \la y,e_n\ra$.
\end{Ut}
\begin{Proof}
	Берём $\lambda\in \F$, элемент $x +\lambda y\in E$. Раскоем равенство Парсеваля для этого элемента
	\[
		\|x+\lambda y\|^2 = \rY n1\big|\la x+\lambda y,e_n\ra\big|^2 = \rY n1|c_n+\lambda d_n|^2 = 
		\rY n1|c_n|^2 + 2\rY n1\Re (\ol \lambda c_n\ol d_n) + |\lambda|^2\rY n1|d_n|^2.
	\]
	В лекциях у нас на кафедре двоечка пропущена.

	С другой стороны можно раскрыть скалярный квадрат по свойствам скалярного произведения
	\[
		\|x+\lambda y\| = \|x\|^2 + 2\Re\big(\ol\lambda\la x,y\ra\big) + |\lambda|^2\|y\|^2.
	\]
	Опять используя равенство Парсеваля, видим, что $\rY n1|c_n|^2 = \|x\|^2$, а $\|\lambda\|^2\|y\|^2 = \rY n1|d_n|^2$.
	Откуда
	\[
		\Re\big(\ol \lambda\la x,y\ra\big) = \rY n1\Re(\ol \lambda c_n\ol d_n).
	\]
\end{Proof}

\begin{The}[Стеклова]
	Ортонормированная система $\pos e\subset E$ является полной, если и только если выполняется равенство Парсеваля, то есть
	\[
		\forall\ x\in E\pau \|x\|^2 = \rY n1|c_n|^2
	\]
\end{The}
\begin{Proof}
	Необходимость.
	Пусть $\pos e$ полная система.
	Зафиксируем $\e>0$, $x \in E$. Тогда найдётся $y = \RY k1m\lambda_k e_k\in \sP\ar ek1m = :L_=m$, для которого $\|x-y\|<\e$.
	Так как $x-s_m\perp L_m$, для $n\ge m$ имеем
	\[
		\|x-S_n\|\le \|x-S_m\|\le \|x-y\|<\e.
	\]
	Ну а это и означает, что ряд Фурье сходится к элементу $x$. А по свойству 2 равенства Парсеваля, если ряд Фурье сходится, то и равенство Парсеваля верно.

	Докажем достаточность. Пусть выполнено равенство Парсеваля $\|x\|^2 = \rY n1|c_n|^2$. Тогда 
	\[
		\|x-S_n\| = \|x\|^2 - \RY k1n|c_k|^2\te 0.
	\]
	Ну это и означает, что всякий $x$ отклоняется от частичный суммы меньше чем на $\e$. То есть линейная оболочка всюду плотна и, следовательно, система полна.
\end{Proof}

\begin{Sl}
	Ортонормированная система $\pos e\subset H$ гильбертова пространства полна, если и только если $\pos e$ тотальна. Иными словами, в гильбертовом пространстве полнота равносильна тотальности.
\end{Sl}
\begin{Proof}
	Необходимость. Если система полна и $\forall\ n\in\N\pau c_n=0$, то из равенства Парсеваля, следует, что норма $\|x\|^2 = \rY n1|c_n|^2 = 0$. Значит, $x=0$. Это мы доказали тотальность.

	Достаточность. Пусть $L = \sP\pos e$, тогда $L^\perp = 0$. А по следствие из теоремы об ортогональном дополнении, $\ol L = H$. Следовательно, система полна.
\end{Proof}

В евклидовом пространстве это неверно. Мы в конце прошлой лекции построили пример. $M\subset C[0,1]\subset L_2[0,1]$, состоящее из алгебраических многочленом. Его ортогональное дополнение в $C$ в евклидовой норме из $L_2$ всюду плотно, но его ортогональное дополнение не равно нулю.

Следующая теорема у вас была уже в курсе линейной алгебры, но мы её подкорректируем.
\begin{The}[Метод ортогонализации Грамма"--~Шмидта]
	Для всякой счётной системы линейно независимых элементов евклидова пространства существует линейная ортонормированная система , то есть
	\[
		\forall\ \pos x\subset E\pau\exists\ \pos e\subset E,
	\]
	где $\pos e$ ортонормирована, причём элементы $x_n$ выражаются через $e_n$. $\pos x$ полна $\iff \pos e$ полна.
\end{The}
\begin{Proof}
	Положим $y_1 = x_1$, нормируем $e_1 = \frac{y_1}{\|y_1\|}$.
	Берём $y_2 = x_2 - \la x_2,e_1\ra e_1\perp e_1$ и нормируем $e_2 = \frac{y_2}{\|y_2\|}$.
	И так далее
\[
	y_n:=x_n - \RY k1{n-1}\la x_n,e_k\ra e_k\perp e_1,\dots,e_{n-1},\pau e_n = \frac{y_n}{\|y_n\|}.
\]

Получаем систему уравенений
\[
	\begin{cases}
		e_1 = a_{11} x_1; \\
		e_2 = a_{21} x_1 + a22 x_2;\\
		\dotfill\\
		e_n = e_{n1} x_1 +\dots+a_{nn}x_n;\\
		\dotfill
	\end{cases}
\]
Матрица этой системы треугольная.
Причём $a_{nn}$, значит, система определена.
\end{Proof}

\begin{The}[Рисса"--~Фишера]
	Каждое сепарабельное гильбертово пространство $H$ изометрически изоморфно либо пространству $\F^n$, либо пространству $l_2$. Соответственно, если $H$ над $\R$, то $\F^n = \R^n$, а $l_2$ над $\R$, над $\C$ аналогично.
\end{The}
\begin{Proof}
	Так как пространство сепарабельно, по определению в нём существует счётная полная система элементов. Давайте её обозначим $\pos x\subset H$. Вообще говоря, эти элементы могут быть линейно зависимы. По индукции выбрасываем те элементы, которые зависят от предыдущих. Полнота останется у системы, так как линейная оболочка не изменится. Если останется конечное число элементов, то всё ещё было доказано в курсе линейной алгебры. Пусть система бесконечна, $\dim H = \infty$.

	Построим по полученной полной счётной линейно незивисимой системе счётную полную ортонормированную $\pos e\subset H$. Таким образом, может для любого элемента $x\in H$ рассматривать ряд Фурье $ x = \rY n1 c_n e_n$, где $c_n = \la x,e_n\ra$ "--- коэффициенты Фурье. Определим отображение
\[
		F\colon H\to l_2,\pau F(x) = c = \pos c.
	\]
	Почему $c\in l_2$? В силу равенства Парсеваля $\|F\|^2 = \|c\|^2_{l_2} = \rY n1|c_n|^2 = \|x\|^2$. Получается изометричное отображение. Осталось доказать, изоморфность, то есть, что $F$ "--- это «отображение на», то есть $\Im(F) = l_2$.

Возьмём $c = \pos c\in l_2$. Найдём элемент $x\in H$, у которого $c_n$ будут является коэффициентами Фурье. Для этого рассмаотрим $S_n = \RY k1n c_k e_k\in H$. Раскроем скалярный квадрат по свойствам скалярного произведения и используем ортонормированность системы.
\[
	\|S_n-S_m\|^2 = \la S_n-S_m,S_n-S_m\ra = \bigg\la\RY k{m+1}nc_ke_k,\RY k{m+1}nc_ke_k\bigg\ra = \RY k{m+1}n|c_k|^2\to 0\pau (m\to\infty)
\]
Таким образом, $\pos S$ "--- последовательность Коши в $H$. Значит, существует предел $\exists\ \yo n\infty S_n = x\in H$. Осталось доказать, что $c_n$ есть его коэффициенты Фурье.
\[
	\forall\ n\in\N\pau \la x,e_m\ra =\yo n\infty\la S_n,e_m\ra = c_m.
\]
Таким образом, $c_n$ "--- последовательность коэффициентов Фурье. И, следовательно, теорема доказана полностью. Мы даже получили изометрический изоморфизм.
\end{Proof}

Осталось мне привести примеры. Рассмотрим $L_2[0,1]$, мера Лебега обычна. И рассмотрим $e_n = e^{2\pi i nx} = \cos(2\pi nx) + i\sin (2\pi nx)$, $n\in Z$. Покажем, что эта система является полной. Пусть $\e>0$. Тогда по теореме о всюду плотности $C[0,1]$ в $L_p[0,1]$
\[
	\forall\ f\in L_2[0,1]\pau \exists\ g\in C[0,1]\colon \|f-g\|_{L_2}<\frac\e3.
\]
Чтобы применить теорему Вейерштрасса, нам нужна не просто непрерывная функция, но ещё и периодическая. Это довольно легко сделать.
\[
	\exists\ \phi C[0,1]\colon \phi(0) = \phi(1),\ \|g - \phi\|_{L_2}<\frac\e3.
\]
Можно на отрезке $[0,\delta]$ заменить функцию на линейную $g(1) + (g(\delta) - g(1))t$. Если $\delta$ маленькая, получаемая площадь разностей $g-\phi$ будет маленькая.

Далее по теореме Веерштрасса
\[
	\exists\ T(x) = \RY k{-n}nc_ke_k\colon \|\phi - T\|_{L_2}\le \|\phi - T\|_C<\frac\e3.
\]
Построили такие функции, теперь применяем неравенство треугольника.
\[
	\|f-T\|_{L_2}\le \|f-g\|_{L_2} + \|g-\phi\|_{L_2} + \|\phi- T\|_{L_2}<e.
\]
Вот мы и доказали, что линейная оболочка системы $\pos e$ всюду плотна в $L_2[0,1]$. Также $L_2[0,1]$ изометрически изоморфна $l_2$. Точно так же изоморфизм строиться. Для каждой функции берём последовательность коэффициентов Фурье.
