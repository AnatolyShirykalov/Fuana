\section{Преобразование Фурье в пространствах Лебега первого и второго порядков}
Я напомню, что $x=(x_1,\dots,x_m)$, $y=(y_1,\dots,y_m)\in\R^m$, $\la x,y\ra = \RY k1m x_k\,y_k$. Для  $f\in\L_1(\R^m)$ преобразование Фурье
\[
  \hat f(x) :=\k^m\Gint{\R^m}f(y)\,e^{-\la x,y\ra}\,dy = \Fu(f);\qquad
  \Til f(x) :=\k^m\Gint{\R^m}f(y)\,e^{\la x,y\ra}\,dy = \Fu^{-1}(f).
\]
Обычно $\k=\frac1{\sqrt{2\pi}}$. Тогда $\Til f(x) = \hat f(-x)$. 

\begin{Lem}[Римана"--~Лебега]\label{RimLeb}
  Если $f\in\L_1(\R^m)$, то
\begin{roItems}
\item $\hat f\in C(\R^m)$;
\item $\|\hat f\|_C:=\sup\limits_{x\in\R^m}\big|\hat f(x)\big|\le\k^m\|f\|_{\L_1}$;
\item $\yo{\|x\|}\infty\hat f(x)=0$.
\end{roItems}
\end{Lem}
Здесь норма обычная $\|x\| = \sqrt{\RY k1mx_k^2}$.

\begin{Proof}
  Первое свойство
\begin{multline*}
  \big|\hat f(x-a)-\hat f(x)\big| = \big|\tau_a\hat f(x)-\hat f(x)\big| =
  \k^m\bigg|\Gint{\R^m}f(y)\big(e^{-i\la x-a,y\ra} - e^{-i\la x,y\ra}\big)\,dy\bigg|\le\\ \le
  \k^m\Gint{\R^m}\big|f(y)\big|\big| e^{-i\la x,y\ra}-1\big|\,dy = 
  \k^m\Gint{\R^m}\big|f(y)\big| 2\left|\sin\frac{\la a,y\ra}2\right|\,dy\xrightarrow{a\to0}0.
\end{multline*}
Значит, $f$ равномерно непрерывна.

Второе
\[
  \|\hat f\|_C\le \k^m\Gint{\R^m}\big|f(y)\big|dy = \k^m\|f\|_{\L_1}.
\]

Третье посложнее. Положим $a:=\frac{\pi\,x}{\la x,x\ra}$ для $x\ne0$. Тогда $\widehat{\tau_a f} = -\hat f$. Следовательно
\[
  \big|\hat f(x)\big| = \frac12\big|\hat f(x)-\widehat{\tau_af}(x)\big| = 
 \frac{\k^m}2\bigg|\Gint{\R^m}\big(f(y)-\tau_a(t)\big)e^{-i\la x,y\ra}\,dy\bigg|\le
\frac{\k^m}2 \|f-\tau_af\|_{\L_1}\le
\]
Теперь применяем применяем неравенство треугольника для нормы
\[
  \le\frac{\k^m}2\big(\|f-g\|_{\L_1}+\|g-\tau_ag\|_{\L_1}+\|\tau_ag-\tau_af\|_{\L_1}\big).
\]
Функцию $g$ выбираем так, чтобы $\|f-g\|_{\L_1}<\frac{2\,\e}{3\k^m}$, где $\e>0$ и $g\in C_0(\R^m)$ (непрерывная функция с компактным носителем). Если сдвинем, получим то же неравенство
\[
 \|\tau_a f-\tau_ag\|_{\L_1}<\frac{2\,\e}{3\k^m}.
\]
В силу непрерывности на компактном носителе, $g$ равномерно непрерывна. Существует $\delta>0\colon \|a\|<\delta\imp \|g-\tau_ag\|_{\L_1}<\frac{2\,\e}{3\k^m}$ (по норме в $C$ это верно, по норме в $\L_1$ тем более).

Тогда для $\|a\|<\delta$ имеем $\big|\hat f(x)\big|<\e$. А $\|a\| = \frac\pi{\|x\|}<\delta$. Значит, $\|x\|>\frac\pi\delta$. То есть предел в бесконечности равен нулю.
\end{Proof}

Докажем теперь условие Дини, но в одномерном случае. В отличие от преобразования Фурье обобщённой функции, функция может получиться не из $\L_1$. Но можно на исходную функцию наложить ограничение.
\begin{The}[условие обращения Дини]
  Пусть $f\in\L_1(\R)$ и при некотором $\delta>0$ и некотором $x$ имеем $\int\limits_{-\delta}^{\delta}\left|\frac{f(x+t)-f(x)}t\right|\,dt<\infty$. Тогда утверждается, что 
\[
  \yo n\infty\int\limits_{-n}^n\hat f(y)e^{-i\,x\,y}\,dy = f(x).
\]
\end{The}
\begin{Proof}
 Запишем интеграл и применим теорему Фубини 
\[
  \k\int\limits_{-n}^n\hat f(y)\,e^{-i\,x\,y}\,dy =
  \frac1\pi\Gint{\R}f(z)\bigg( \int\limits_{-n}^n e^{-i\,(x-z)\,y}\,dy\bigg)\,dz=
  \frac1\pi\Gint{\R}f(z)\frac{\sin n(x-z)}{(x-z)}\,dz.
\]
Представим подынтегральную функцию через экспоненту по формуле Эйлера и используем, что $\Gint{\R}\frac{\sin nt}t\,dt=\pi$.
\begin{multline*}
  \frac1\pi\Gint{\R}f(z)\frac{\sin n(x-z)}{(x-z)}\,dz - f(x)=
  \frac1\pi\Gint{\R}\frac{f(x-t)-f(x)}t\sin nt\,dt = \\ =
  \underbrace{\frac1\pi\Gint{|t|\le\delta}\frac{f(x-t)-f(x)}t\sin nt\,dt}_{\te0\ \text{(интегрируема)}}+ 
  \underbrace{\frac1\pi\Gint{|t|>\delta}\frac{f(x-t)}t\sin nt\,dt}_{\to0\ (\ref{RimLeb})} +
  \underbrace{\Gint{|t|>\delta n}f(x) \frac{\sin nt}t\,dt}_{\te0}.
\end{multline*}
\end{Proof}

\begin{Ut}[Формула умножения] 
 Пусть $f,g\in\L_1(\R^m)$. Тогда $\Gint{\R^m}\hat f(x) g(x)\,dx = \Gint{\R^n}f(x)\hat g(x)\,dx$.
\end{Ut}
\begin{Proof}
  $\Gint{\R^m}\bigg(\Gint{\R^m}f(y)e^{-i\la x,y\ra}\,dy\bigg)g(x)\,dx = 
  \Gint{\R^m} f(y)\bigg(\Gint{\R^m} g(x)e^{-i\la x,y\ra}\,dx\bigg)\,dy$.
\end{Proof}
\begin{Ut}[Формула обращения]
  Пусть $f,\hat f\in \L_1(\R^m)$. Тогда
\[
  \Til{\hat f}(x) = \Til{\hat f}(x) = f(x)\text{ п.\,в.} x\in \R^m
\]
\end{Ut}
\begin{Proof}
Имеем
\[
  \forall\ \phi \in S(\R^m)\pau \Gint{\R^m}\Til {\hat f}(x)\phi(x)\,dx = 
  \Gint{\R^m} f(x)\hat{\Til\phi }(x)\,dx = 
  \Gint{\R^m} f(x)\,\phi(x)\,dx.
\]
Значит, $\Til {\hat f(x)}-f(x)=0$ почти всюду.
\end{Proof}
\begin{Ut}[Формулы дифференцирования]
  Пусть $f\in\L_1(\R^m)$, $x^\a f(x)\in\L_1(\R^m)$. Тогда $\dl^\a\hat f(x) = \widehat{(-i\,y)^\a f(y)}$.

Если $f\in W_1^k(\R^m)$, то $\forall\ |\a|\le k,\ \forall\ x\in\R^m\pau 
  \widehat{\dl^\a f}(x) = (ix)^\a\hat f(x)$.
\end{Ut}

Раз мы уже показали, что данное преобразование совпадает с обобщённым, то всё уже доказано. Просто равенства выполнены почти всюду, но для элементов из $\L_1$ это неважно.

\begin{Ut}
  Формула свёртки. Пусть $f,g\in\L_1(\R^m)$. Тогда
\[
  f\star g(x) = \Gint{\R^m} f(y)g(x-y)\,dy\in\L_1(\R^m);\quad
  \widehat{f\star g}(x) = \k^{-m} \hat f(x)\cdot \hat g(x).
\]
\end{Ut}
\begin{Proof}
 Рассмотрим невырожденное линейное преобразование (ведь существует обратное) $(x,y)\to(y,x-y)\colon \R^{2m}\to\R^{2m}$. Так как преобразование линейно, оно переводит измеримые в измеримые. Так как $f(x)\cdot g(y)$  измерима, то $f(y)g(x-y)$ тоже измерима в $\R^{2m}$. Более того, $f(y)g(x-y)\in\L_1(\R^{2m})$. И выполняются неравенства
\[
  \|f\star g\|_{\L_1}\le \Gint{\R^m}\,dx\bigg(\Gint{\R^m}\big|f(y)g(x-y)\big|\,dy\bigg) = \|f\|_{\L_1}\|g\|_{\L_1}.
\]
Кроме того
\begin{multline*}
  \k^m\Gint{\R^m}\bigg(\Gint{\R^m}f(z)g(y-z)\,dz\bigg)e^{-i\la x,y\ra}\,dy=\\=
  \k^m\Gint{\R^m} f(z)\bigg(\Gint{\R^m}g(y-z)\,e^{-i\la x,y-z\ra}\,dy\bigg) e^{-i\la x,y\ra}\,dz = \{ y\to y-z\} =\k^{-m}\hat f(x)\hat g(x).
\end{multline*}
\end{Proof}
\subsection{Преобразование Фурье в $\L_2$}
\begin{Def}
 Обозначим $\Delta_n:=\Big\{ x\in\R^m\Bigm|\max\limits_{1\le k\le m}|x_k|<n\Big\}$. Пусть $f\in\L_2(\R)$. Тогда определим преобразование Фурье
\[
  \hat f(x):=\yo n\infty \Gint{\Delta_n} f(y) e^{-i\la x,y\ra}\,dy.
\]
Здесь предел берётся в $\L_2(\R^m)$.
\end{Def}
Как и в признаке Дини приходится брать предел. Обозначим $f_n(x) = f(x)\chi_{\Delta_n}(x)$. Тогда $f_n\in\L_1(\R^m)$ для каждого $n\in\N$. Тогда считаем по определению $\Til f(x) := \hat f(-x)$.

Докажем, что предел существует и оператор сохраняет норму.
\begin{The}[Планшереля]\label{Plansh}
  Если $f\in\L_2(\R^m)$, то $\exists\ \hat f = \lim\limits_{n\to\infty} \hat f_n$ в $\L_2(\R^m)$ и $\|\hat f\|_{\L_2} = \|f\|_{\L_2}$.
\end{The}
\begin{Proof}
 Легко проверить, что если возьмём функцию $ \phi\in S(\R^m)$, то 
\begin{equation}\label{SwartzFourier}
  \|\phi\|_{\L_2}^2 = \Gint{\R^m}\phi(x)\ol{\phi(x)}\,dx = 
  \Gint{\R^m}\Til{\hat\phi}(x)\ol{\phi(x)}\,dx = 
  \Gint{\R^m}\hat \phi(x)\ol{\hat\phi(x)}\,dx = \|\hat\phi\|_{\L_2}^2.
\end{equation}
Таким образом для функций из $S$ мы доказали.

Пусть $f(x)\in\L_2(\R^m)$ и $f(x)=0$ для всех $x\not\in\Delta_r$. Тогда по лемме \ref{odensity} о плотности 
\[
  \exists\ \phi_n\in\D(\Delta_r)\colon \|f-\phi_n\|_{\L_2}\te0.
\]
 Мы это доказывали для $\L_p$. Значит, $\{\phi_n\}$ "--- последовательность Коши в $\L_2(\R^m)$. Следовательно из равенства \eqref{SwartzFourier} и $\{\hat\phi_n\}$ "--- последовательность Коши в $\L_2(\R^m)$. Сходятся в $\L_2$ на ограниченном множестве, значит, сходятся в~$\L_1$, а $\phi_n,f\in\L_1(\Delta_r)$; поэтому из леммы \ref{RimLeb} Римана"--~Лебега получаем $\hat\phi_n\rsh[ ] n\hat f$. Значит, $\hat\phi_n\te \hat f$ в~$\L_2(\R^m)$. Значит,
\begin{equation}\label{L2Ssssss}
  \|f\|_{\L_2} = \yo n\infty\|\phi_n\|_{\L_2} = \yo n\infty \|\hat\phi_n\| = \|\hat f\|_{\L_2}.
\end{equation}

Докажем теперь в общем случае. Пусть $f\in\L_2(\R^m)$. Тогда $f_n = f\cdot\chi_{\Delta_n}\te f$ в $\L_2(\R^m)$. Значит, $\{f_n\}$ "--- последовательность Коши в $\L_2$. Отсюда и преобразование Фурье тоже является последовательностью Коши в~силу последнего равенства \eqref{L2Ssssss}. Отсюда существует предел в $\L_2$, то есть $\exists\ \hat f=\yo n\infty \hat f_n$ в $\L_2(\R^m)$. И осталось написать равенство норм.
\end{Proof}

Давайте сформулируем теперь свойства преобразования Фурье для функций из $\L_2$.
\begin{Ut}
  Формула умножения. $f,g\in\L_2(\R^m)$. Тогда
\[
  \Gint{\R^m}\hat f(x)g(x)\,dx = \Gint{\R^m}f(x)\hat g(x)\,dx.
\]
\end{Ut}
Эта формула получается из теоремы \ref{Plansh} Планшереля и непрерывности скалярного произведения.
\begin{Ut}\label{L2Fobr}
  Формула обращения. Если $f\in\L_2(\R^m)$, то $\Til {\hat f}(x) = \hat {\Til f}(x) = f(x)$ почти всюду на $\R^m$.
\end{Ut}
Доказывается так же, как и в $\L_1$.
\begin{Ut}
Формула свёртки. Пусть $f\in\L_1(\R^m)$, $g\in\L_2(\R^m)$. Тогда
\[
  f\star g\in\L_2(\R^m),\quad \widehat{f\star g}(x) = \k^{-m}\hat f(x)\hat g(x) \text{ почти всюду на }\R^m.
\]
\end{Ut}
\begin{Proof}
  Интегрируемость в квадрате вытекает и обобщённого неравенства Минковского
\[
  \|f\star g\|_{\L_2}\le \|f\|{\L_1}\cdot \|g\|_{\L_2}.
\]
Применя теорему Планшереля, переходя к пределу, получаем формулу.
\end{Proof}

\subsection{Функции Эрмита}
Это вот такие функции $h_n(x):=c_n e^{\frac{x^2}2}\left(\left(\DP{ }x\right)^ne^{-x^2}\right)$. Здесь $c_n$ константа. Если произвести дифференцирование
\[
  h_n(x) = H_n(x) e^{-\frac{x^2}2},\quad H_n(x) = \RY k0n a_k x^k
\]
$H_n(x)$ называются многочленами Эрмита.

\begin{Ut}\label{ErmOrt}
$\big\{h_n(x)\big\}$ ортогональны.
\end{Ut}
\begin{Proof}
  Имеем
\[
  \Gint{\R} h_n(x)h_m(x)\,dx = 
  c_m\Gint{\R} H_n(x)\left(\DP { }x\right)^m e^{-x^2}\,dx =
  c_m(-1)^m\Gint{\R} e^{-x^2}\left(\DP { }x\right)^m H_n(x)\,dx = 0,\ m>n.
\]
Для $n=m$ это равно $c_n^2 2^n n!\sqrt{\pi}$, так как $\Gint{\R}e^{-x^2}\,dx = \sqrt\pi$.
\end{Proof}
Значит, для ортонормированной системы берём $c_n = \frac1{\sqrt{2^n n!\sqrt{\pi}}}$.

\begin{Lem}
  Пусть $a,b>0$, $n\in\Z_+$, функция $\phi$ измерима и удовлетворяет неравенству $0<\big|\phi(x)\big|\le b e^{-a|x|}$. Тогда система функций $\phi_n(x) = x^n\phi(x)$ при $n\in\Z_+$ полна в $\L_2$.
\end{Lem}
\begin{Proof}
Мы из прошлого семестра знаем критерий полноты. Мы им и воспользуемся. Пусть $f\in\L_2(\R)$ и $\forall\ n\in\Z_+\pau f\perp\phi_n$, то есть 
\[
 \forall\ n\in\Z_+\pau \Gint{\R}f(x)\phi_n(x)\,dx = 0.
\]
Рассмотрим функцию комплексного переменного $F(z) = \Gint{\R}f(t)\phi(t) e^{-i\,t\,z}\,dt$, где $z = x+i\,y\in\C\colon |y|<a$. Можно дифференцировать под знаком интеграла, значит, функция получится голоморфной в $|\Im z|<a$. Заметим, что производные в нуле равны нулю, то есть
\[
  F^{(n)}(0) = \Gint{\R}f(t)(-i\,t)^n\phi(t)\,dt = 0
\]
в силу условия ортогональности. Значит, по теореме об аналитическом продолжении функция будет тождественным нулём в полосе $|\Im z|<a$. В частности, она будет равна нулю для всех $x\in\R$. А тогда это с точностью до константы преобразование Фурье ноль, но существует обратное по \ref{L2Fobr}. И обратное обязано быть нулём почти всюду. Значит, $f(t)\phi(t)=0$ почти всюду. Отсюда $f(t)=0$ почти всюду.
\end{Proof}

\begin{The}\label{ErmFull}
  Функции Эрмита $\{h_n\}$ образуют полную ортонормированную систему, для которой \[
  \forall\ n\in\Z_+\pau \hat h_n(x) = (-i)^n h_n(x).
\]
 То есть они являются собственными функциями преобразования Фурье и образуют полную ортонормированную систему.
\end{The}
\begin{Proof}
$\{h_n\}$ являются ортонормированными по утверждению \ref{ErmOrt}. Осталось доказать, что функции Эрмита являются собственными.
\begin{multline*}
  \hat h_n(x) = \k\Gint{\R} h_n(y) e^{-i\,x\,y}\,dy = 
  \k c_n\Gint{\R}e^{\frac{y^2}2 - i\,x\,y}\left(\DP { }y\right)^ne^{-y^2}\,dy =\\
= \k c_n e^{\frac{x^2}2}\Gint{\R} e^{\frac{(y-i\,x)^2}2}\left(\DP{ }y\right)^ne^{-y^2}\,dy = % интегрируем по частям
  \k c_n(-1)^ne^{\frac{x^2}2}\Gint{\R} e^{-y^2}\left(\DP{ }y\right)^ne^{\frac{(y-i\,x)^2}2}\,dy = \\
 = \k c_n(-i)^ne^{\frac{x^2}2}\left(\DP{ }x\right)^n\Gint{\R}e^{-\frac{y^2}2 - i\,x\,y - \frac{x^2}2}\,dy = \\
\cmt{Мы доказывали \eqref{Ermit1}, что $\widehat{e^{-\frac{x^2}2}} = e^{-\frac{x^2}2}$}\\
= c_n(-i)^n e^{\frac{x^2}2}\left(\DP{ }x\right)^ne^{-x^2} = (-i)^n h_n(x).
\end{multline*}
\end{Proof}
