\section{Теореома о гомеоморфизме}

\begin{The}[Банаха об обратном оператора]
  Пусть $E,F$ "--- банаховы пространства, $A\in \L(E,F)$ и $A$ обратим. Тогда $A^{-1}\in\L(F,E)$.
\end{The}
\begin{Proof}
Доказательство совсем простое. Пусть $A\colon E\to F$ непрерывен. Он имеет замкнутый график $\gr(A)\subset E\times F$, то есть
\[
  x_n\to x,\ y_n = A x_n\to y\imp Ax=y.
\]

Поскольку $A$ обратим, то есть осуществляет инъекцию, если $y_n\to y$ и $x_n = A^{-1}y_n\to x$, то $A^{-1}y=0$. Значит, $gr(A^{-1})\subset F\times E$ тоже замкнут. По теореме о замкнутом графике $A^{-1}\L(F,E)$.
\end{Proof}

Для теоремы о гомеоморфизме нужно ввести понятие факторпространства для нормированного пространства.

\begin{Def}
Пусть $E$ "--- нормированное пространство, $M\subset E$ "--- замкнутое подпространство. Тогда фактор-пространством называется
\[
  \hat E = E/M = \big\{\hat x = x+ M\bigm|x\in E\big\}.
\]

На этом пространстве смежных классов вводятся операции
\begin{azItems}
\item $\forall\ x,y\in E\pau \hat x+ \hat y = \hat{x+y}$;
\item $\forall\ x\in E,\ \forall \lambda\in\F\in\{\R,\C\}\pau \lambda\cdot \hat x = \hat{\lambda x}$;
\item $\|\hat x\|:=\inf\limits_{y\in M}\|x+y\|$.
\end{azItems}
\end{Def}

\begin{Ut}
$\|\hat x\|$ есть норма.
\end{Ut}
\begin{Proof}
  $\lambda \hat x\| = \|\hat{\lambda x}\| =\int\limits_{y\in M}\|\lambda x+y\| = |\lambda|\int\limits_{y\in M}\|x+y\| = |\lambda|\cdot\|\hat x\|$.

Неравенство треугольника
\[
  \|\hat x+\hat y\| = \|\hat{x+y}\| = \inf\limits_{z\in M}\|x+y+z\| = \inf\limits_{u,v\in M}\|x+y+u+v\|\le
  \inf\limits_{u\in M}\|x+u\| + \inf\limits_{v\in M}\|y+v\| = \|hat x\| + \|hat y\|.
\]

И последнее свойство нормы. Если $\|\hat x\|=0$, то $\exists\ y_n\in M\colon x+y_n\to 0$. Значит, $x = -\lim y_n\in M$ и $\hat x=0$.
\end{Proof}

\begin{Lem}
  Если $M\subset E$ замкнутое подпространство в банаховом пространстве $E$, то фактор-пространство $\hat E = E/M$ тоже является банаховым пространством.
\end{Lem}
\begin{Proof}
  Мы уже доказали, что факто-пространство является нормированным пространством. Докажем полноту по этой норме. Пусть нам задана последовательность Коши $\{\hat x_n\}$ в $\hat E$. Выеберем последовательность индексов $n_1<n_2<\dots$ так, что $\forall\ n,m\ge n_k\pau \|\hat x_{n} - \hat x_{m}\|<\frac1{2^k}$. По определению последовательности Коши такая последовательность индексов существует.

Пусть $z_{n_k}\in\hat x_{n_k} = x_{n_k}+M$ такие, что $\|z_{n_{k+1}} - z_{n_k}\|<\frac1{2^k}$. И определим элемент
\[
  z:=z_{n_1} + \rY k1(z_{n_{k+1}} - z_{n_k}) \in E.
\]
Частичные суммы будут как раз $z_{n_k}$. Этот ряд сходится по норме, так как $1/2^k$ сходится.

Кроме того для $n\ge n_k$ имеем 
\[
 \|\hat z - \hat x_n\|\le \|\hat z - \hat x_{n_k}\| + \|\hat x_{n_k} - \hat x_n\|\le \|z-z_{n_k}\|  + \frac1{2^k}
 <\rY lk\frac1{2^l} + \frac1{2^k} < \frac3{2^k}.
\]
Отсюда $\lim \hat x_n = \hat z\in\hat E$.
\end{Proof}

\begin{Def}
Линейный оператор $A\colon E\to F$ называется открытым отображением, если он отображает всякое открытое множество в открытое, то есть
\[
  \forall\ U\subset E\text{ откр.}\pau A(U)\subset F\text{ откр.}
\]
\end{Def}

Давайте докажем, что факто-отображение $\pi\colon E\to \hat E$, определённое по формуле $\pi(x):=\hat x$ является открытым отображением.
\begin{Proof}
Во-первый, $\pi$ непрерывно, так как $\big\|\pi(x)\big\|\le \|x\|$. Во-вторых $\pi$ открыто, так как $\pi(U_r) = \hat U_r$, где
\[
  U_r = \big\{x\in E\bigm|\|x\|<r\big\},\pau r>0;\quad
  \hat U_r =\big\{\hat x\in \hat E\bigm|\|\hat x\|<r\big\}.
\]
В общем это очевидно. Ну а раз открытый шар переходит в открытый, то его сдвиг тоже отображается в открытый шар. А всякое открытое множество является объединением шаров.
\end{Proof}

\begin{Def}
  Линейный оператор $A\colon E\to F$ называется гомоморфизмом, если
\begin{roItems}
 \item $A$ непрерывный;
\item  $A$ открытый.
\end{roItems}
\end{Def}
В частности мы доказали, что фактор-отображение является гомоморфизмом линейных пространств.

\begin{The}[о гомоморфизме]
  Если $E,F$ "--- банаховы пространства, $A\in\L(E,F)$ сюрьективный, то $A$ гомоморфизм.
\end{The}
\begin{Proof}
  Пусть у нас $M$ будет ядром этого оператора, то есть $M = \ker(A) = \big\{x\in E\bigm| A(x)=0\big\}$. Ядро ограниченного оператора является замкнутым подпространством, мы с вами это доказывали. Поэтому можем рассмотреть фактор-пространство, которое будет банаховым пространством $\hat E = E/M$. Оно банахово, поскольку $M$ замкнуто, а $E$ банахово. Определим $\hat A(\hat x) = A(x)$ для $x\in E$. Мы ходим, чтобы $\hat A\colon\hat E\to F$. Этот оператор корректно определён, так как если $\hat x = \hat y$, то $x-y\in M$, значит, $A(x) = A(y)$.
 
Кроме того, $\ker(\hat A) = 0$ и $\Im\hat A=F$, поэтому отсюда следует, что $\hat A$ обратимый оператор. Докажем, что $\|\hat A\| = \|A\|$. В самом деле 
\[
  \big\|\hat A(\hat x)\big\| = \big\|A(x)\big\| = \inf\limits_{y\in M}\big\|A(x+y)\big\|\le \|A\|\cdot \inf\limits_{y\in M}\|x+y\| = \| A\|\|\hat x\|\imp \|\hat A\|\le \|A\|.
\]

И теперь обратное неравенство.
\[
  \big\|A(x)\big\| = \big\|\hat A(\hat x)\big\|\le \|\hat A\|\cdot\|\hat x\|\le \|\hat A\|\|x\|.
\]
По теореме Банаха об обратном операторе, этот оператом гомоморфизм. Ну а сам оператор $A$ является композицией или произведением двух операторов $A = \hat A\pi$. Легко проверить, что произведение двух гомеоморфизмов будет гомеоморфизмом.
\end{Proof}

\begin{The}[о тройке]
  Пусть $E,F,G$ "--- банаховы пространства, $A\in\L(E,F)$, $B\in \L(E,G)$, $A$ сюрьективный и $\ker (A)\subset \ker(B)$. Тогда $\exists\ C\in\L(F,G)\colon B = CA$.
\end{The}
\begin{Proof}
  Чтобы это понимать, лучше нарисовать диаграмму.
\[
\begin{diagram}
  \node{E}\arrow[2]{e,t}{A}
  	  \arrow[2]{se,b}{B}
  \node[2]{F}\arrow{s,r}{C} \\
  \node[2]{G}
\end{diagram}
\]
Положим $C(y) = B(x)$, если $y= Ax$. Пусть случилось так, что $y = Ax_1= A x_2$. Тогда $x_1-x_3 = \ker(A)\subset \ker(B)$. Значит, $B x_1 = B x_2$.

Возьмём $\lambda\in \F$. Тогда $C(\lambda y) = C(\lambda Ax) = C\big(A(\lambda x)\big) = B(\lambda x) = \lambda Bx = \lambda C(y)$.

Пусть $y_1 = A x_1$ и $y_2 = A x_2$. Тогда $C(y_1+y_2) = C\big( A(x_1+x_2)\big) = B(x_1 +x_2) = Bx_2 + Bx_2 = C(y_1) + C(y_2)$.

Положим $M = \ker(A)$ и $\hat E = E/M$.  $\hat A\in\L(\hat E,F)$, $\|\hat A\|=\|A\|$.
\[
  \big\|C(y)\big\| = \big\|B(x)\big\| = \inf\limits_{z\in M}\big\|B(x+y)\big\|\le
  \|B\|\inf\limits_{z\in M}\|x+y\| = \|B\| \|\hat x\|=
  \|B\|\|\hat A^{-1}y\|\le \|B\|\|\hat A^{-1}\|\|y\|.
\]
$\|C\|\le \|B\|\cdot \|\hat A^{-1}\|$.
\end{Proof}


Если $V\subset E$ и $W\subset E^*$, то определяются аннулятор 
\[
V^\perp = \big\{f\in E^*\bigm| \forall\ x\in E\pau f(x)=0\big\}\subset E^*;
\]
и аннулятор
\[
  W_\perp = \big\{x\in E\bigm| \forall\ f\in W\pau f(x)=0\big\}\subset E.
\]
Мы с вами доказывали, что это замкнутые подпространства.

\begin{Lem}[о бианнуляторе]
  Если $M\supset E$ замкнутое подпространство, то $(M^\perp)_\perp = M$. 
\end{Lem}
\begin{Proof}
  То, что $M\subset (M^\perp)_\perp$ очевидно.

Существует $x\in (M^\perp)_\perp$, для которого $x\not\in M$. Для него $\|\hat x\|\ne0$. Положим
\[
  L:=\sp{x,M}.
\]
Имеем $f(\lambda x + y) = \lambda$ в $\hat E = E/M$.
\[
  \|f\|_L = \sup\limits_{z\in L\dd 0}\frac{\big|f(z)\big|}{\|z\|} =
  \sup\limits_{y\in M}\frac{|\lambda|}{\|\lambda x + y\|} = 
  \sup\limits_{y\in M}\frac1{\|x+y\|} = 
  \frac1{\inf\limits_{y\in M}\|x+y\|} = \frac1{\|\hat x\|}<\infty.
\]
Значит, $\exists\ g\in E^*\colon \|g\| = \|f\|_L$ по теореме Хана"--~Банаха и $g(x) = f(x) = 1$. Получили противоречие.
\end{Proof}
\begin{The}
  Пусть $E,F$ "--- баназовы пространства, $A\in\L(E,F)$ и его образ $\Im(A)\subset F$ замкнут. Тогда $\Im(A) = \ker(A^*)_\perp$ и $\Im(A^*) = \ker(A)^\perp$.
\end{The}
Очень важная теорема. Мы уже доказывали одно из включений. А сейчас при новом условии замкнутости образа докажем равенства.
\begin{Proof}
  Первое равенство прямо вытекает из леммы. У нас было доказало равенство $\ker(A^*) = \Im(A)^\perp$. Берём аннулятор, получится бианнулятор
\[
  \ker(A^*)_\perp = \big(\Im(A)^\perp\big)_\perp = \Im(A),
\]
так как образ замкнут.

Второе равенство из включения, которое мы уже строили $\Im(A^*)\subset \ker(A)^\perp$ и теоремы о тройке. Пусть у нас $g\in \ker(A)^\perp$. Тогда в частности $\ker(A)\subset \ker(g)$. Поэтому можно применить теорему о тройке: существует $f\in E^*$, для которого $\forall\ x\in E\pau g(x) = f(Ax)$, а это как раз и означает, что $g = A^*f$, то есть $g\in \Im(A^*)$. Ну и значит, мы доказали обратное включение.
\end{Proof}

\begin{Def}
 Линейный оператор $P\colon E\to E$ называется проектором на подпространство $M\subset E$, если $P^2=P$ и $\Im(P) = M$.
\end{Def}
Все элементы образа не изменяются при отображении $P$.
\begin{Ut}
  $P|_M = I|_M$.
\end{Ut}
\begin{Proof}
  Если $y = P(x)$, то $P(y) = P^2(x) = P(x) = y$.
\end{Proof}
\begin{Ut}
  $\ker(I-P) = \Im(P)$ и $\Im(I-P) = \ker(P)$.
\end{Ut}
\begin{Proof}
Докажем одно, второе сами разберёте. Оно совершенно простое. Пусть $x\in \ker(I-P)$. Тогда $(I-P)x =x-P(x) = 0$ и $x\in Im(P)$.

Если $y = P(x)\in\Im(P)$, то $(I-P)y =(I-P)P(x) = (P-P^2)x=0$. Значит, $y\in \ker(I-P)$.
\end{Proof}
\begin{Ut}
 $E = M\oplus L$, $M = \Im(P)$, $L = \ker(P)$.
\end{Ut}
\begin{Proof}
Так как $I = P + (I-P)$, то $E = M+L$. Осталось доказать, что эти пространстве не пересекаются. Если $y\in M\cap L$, то $y = P(x)=x-P(x)$. Отсюда следует, что $x = 2 P(x)$; если к этому равенству применить $P$, получим $P(x) = 2P(x)$, ну и отсюда $y = P(x) = 0$.
\end{Proof}

Теперь введём такое понятие, очень важное.
\begin{Def}
Подпространство $M\subset E$ называется дополняемым в $E$, если 
\begin{roItems}
  \item $M$ замкнуто;
  \item $\exists\ L\subset E$ замкнутое, такое, что $E = M\oplus L$.
\end{roItems}
\end{Def}
Если замкнутость не требовать, то разложение в прямую сумму будет всегда. А с замкнутостью не всегда. Приведу пример позже и без доказательства.
\begin{The}
  Пусть $E$ "--- банахово пространство, и $M\subset E$ "--- замкнутое подпространство. Тогда $M$ дополняемо, если и только если существует ограниченный проектор $P\in\L(E,E)$ на подпространство $M$.
\end{The}
\begin{Proof}
  Необходимость. Пусть $E = M\oplus L$, где $M,L$ "--- замкнутые подпространства. По определению прямой суммы будем строить проектор.
\[
  \forall\ y\in E\pau \exists !\ y\in M,\ z\in L\colon x = y+ z.
\]
Поэтому корректно определён оператор $P(x) = y\in M$. Очевидно, что он (из определения прямой суммы) будет проектором на подпространство $M$. Нам нужно доказать, что он непрерывен. Для этого воспользуемся теоремой о замкнутом графике. Покажем, что $P\colon E\to E$ имеет замкнутый график. Пусть $x_n\to x$ и $y_n = P(x_n)\to y$. Так как $x_n = y_n + z_n$, то $z_n\to z$. Поэтому $x_n = y_n + z_n\to x = y+ z$. В силу единственности этого разложения мы получаем, что $P(x) = y$, то есть график замкнут. И по теореме о замкнутом графике оператор ограничен.

Достаточность. Пусть $P\in \L(E,E)$ "--- ограниченный проектор на подпространство $M$, где $M = \Im(P)$. Мы доказывали, что $M = \ker(I-P)$. И обозначим через $L$ ядро $P$, то есть 
$L = \ker(P) = \Im(I-P)$. Так как $P$ ограничен, $I-P$ как сумма ограниченных тоже ограничен. Тогда у них замкнутые ядра. Отсюда $E = M\oplus L$.
\end{Proof}

\begin{Lem}
  Пусть $E$ "--- банахово пространство, $M\subset E$ "--- замкнутое подпространство. Тогда если
\begin{roItems}
  \item $\dim M<\infty$;
  \item  или $\codim M:=\dim(\hat E)<\infty$,
\end{roItems}
то $M$ является дополняемым подпространством.
\end{Lem}
\begin{Proof}
  Два случая.
\begin{roItems}
\item Пусть $\{e_1,\dots,e_n\}$ "--- базис в $M$, а $\{f_1,\dots, f_n\}\subset E^*$ биорттогональная система. Пусть $x\in E$. Положим
\[
  y: =\RY k1n f_k(x) e_k\in M;\quad z:= x-y\in L =\CAP k1n\ker(f_k).
\]
Имеем $x = y+z$, а $E = \oplus L$.

\item Берём базис $\{\hat e_1,\dots,\hat e_n\}$ в $\hat E = E/M$. Положим $L:=\sp\{e_1,\dots,e_n\}$. Тогда
\[
  \forall\ x\in E\pau \exists\ \lambda_1,\dots,\lambda_n\in\F\colon \hat x = \RY k1n \lambda_k\hat e_k.
\]
Положим $z:=\RY k1n \lambda_ke_k\in L$ и $y = x-z\in M$. Тогда $x = y+z$ и $E = M\oplus L$.
\end{roItems}
\end{Proof}

Пример недополняемого замткнутого пространства. $C(T) = \big\{f\colon T\to \C\bigm|f\text{ "--- непрерывна на }T\big\}$. Ещё положим
\[
  T:=\big\{z\in\C\bigm| |z|=1\big\};\quad
  A(D) = \big\{f\colon D\to \C\bigm| f\text{ "--- голом. в $D$ и непр. в $\ol D$}\big\};\quad
  D = \big\{z\in \C\bigm |z|\le 1\big\}.
\]
Вот это $A(D)\subset C(T)$ недополняемое пространство.
