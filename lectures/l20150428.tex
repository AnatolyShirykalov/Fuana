\section{Эрмитовы операторы}
Я хотел бы обратить ваше внимание на примеры, которые я включил прямо в вопросы.

Определение эрмитовых операторов у нас уже было. Но я сейчас напомню.
В линейной алгебре тоже есть понятие эрмитова оператора, он задаётся эрмитовой матрицей в конечномерном пространстве.

\begin{Def}
Пусть $H$ "--- гильбертово пространство, линейный оператор $A\colon H\to H$ эрмитовый, если для $\forall\ x,y\in H$ выполняется равенство
\[
  \la Ax,y\ra = \la x,Ay\ra.
\]
\end{Def}
То есть это ограниченный самосопряжённый оператор. Мы доказывали теорему, которая утверждает, что всякий эрмитовый оператор ограничен. Пусть $\mathcal H(H)$ "--- множество всех эрмтовых операторов. Тогда $\mathcal H(H)\subset\L(H)$.

Перейдём к свойствам спектра эрмитового оператора.
\begin{Ut}
  Если $A\in\HH(H)$ и $\lambda\in \sigma_p(A)$, то $\lambda\in\R$ и $H_{\lambda_1}\perp H_{\lambda_2}$ для $\lambda_1\ne \lambda_2$, где $H_\lambda = \ker A_\lambda$, $A_\lambda = \lambda I - A$.
\end{Ut}
\begin{Proof}
Это фактически было доказано в курсе линейной алгебры. Мы это тоже докажем. Пусть $Ae = \lambda e$, $\|e\| = 1$. Тогда имеем $\lambda  = \la A e,e\ra = \la e, Ae\ra = \ol\lambda$. То есть доказали, что $\lambda$ число действительное.

Пусть у нас теперь есть два числа $Ae_1 = \lambda e_1$ и $A e_2 = \lambda_2 e_2$.
Тогда получаем такое равенство: $\lambda_1\la e_1,e_2\ra = \la Ae_1,e_2\ra = \la e_1,A e_2\ra = \lambda_2 \la e_1,e_2\ra$. Поскольку $\lambda_1\ne \lambda_2$, получаем, что $\la e_1,e_2\ra=0$, то есть $e_1\perp e_2$.
\end{Proof}

\begin{Ut}[Критерий Вейля]
    Если $A\in\HH(H)$, то $\sigma(A) = \sigma_e(A) = \big\{\lambda\in \C\bigm| \inf\limits_{\|x\|=1}\|A_\lambda x\| = 0\big\}$.
\end{Ut}
Есть операторы, у которых предельный спектр содержит не только собственные значения.
\begin{Proof}
Мы доказывали, что $\sigma_e(A)\subset \sigma(A)$. Нам надо доказать обратное включение. Пусть $\lambda\in \sigma_e(A)$. По определению существует такая $c>0$, для которой
\[
  \forall\ x\in H\pau \|A_\lambda x\|\ge c\|x\|.
\]
Докажем, что образ $\Im A_\lambda$ замкнут. Предположим, что у нас есть последовательность $A_\lambda x_n\to y$. Тогда используя только что полученное неравенство,
\[
  \|x_m - x_n\|\le \| A_\lambda x_m - A_\lambda x_n\|
\]
Прибавим и вычтем $y$ и поделим на $c$
\[
  \frac{\|A_\lambda x_m - y\|}c + \frac{\|A_\lambda x_n - y\|}c \xrightarrow[n,m\to\infty]{ }0.
\]
Значит, $\{x_n\}$ последовательность Коши в гильбертовом пространстве. И существует предел $x = \lim x_n$. Значит, $A_\lambda x = y\in \Im A_\lambda$. Таким образом, замкнутость образа доказана.

Может ли $\Im A_\lambda = H$? Пусть $\exists\ y\ne 0\colon y\perp \Im A_\lambda$. Ну если это так, то
\[
  \forall\ x\in H\pau\la A_\lambda x,y\ra = \la x, A_{\ol\lambda}y\ra = 0.
\]
Ну а следовательно оператор $A_\lambda y = 0$. Мы доказывали, что собственные числа действительные, значит, $A_\lambda y = 0$, что невозможно, поскольку выполняется неравенство $\|A_\lambda x\|\ge c\|x\|$.

Следовательно, $\Im A_\lambda = H$, $\ker A_\lambda=0$ и $\lambda\not\in\sigma(A)$.
\end{Proof}
\begin{Ut}
  Если $A\in \HH(H)$, то его спектр $\sigma(A)\subset\R$.
\end{Ut}
Мы пока доказали, что собственные значения лежат на действительной оси, а теперь покажем, что и весь спектр.
\begin{Proof}
Пусть $\lambda\in\C\colon \lambda = \alpha + i\beta$, $\beta\ne0$. Тогда запишем следующие равенства ($A_\lambda = A_\alpha + i \beta I$)
\[
  \|A_\lambda x\|^2 = \la A_\lambda x,A_\lambda x\ra = 
   \la A_\alpha x, A_\alpha x\ra - i\beta \la A_\alpha x,x\ra + i\beta\la x, A_\alpha x\ra + \beta^2\la x,x\ra.
\]
В силу эрмитовости остаётся только два слагаемых
\[
  \|A_\lambda x\|^2  = \|A_\lambda x\|^2 + \beta^2 \|x\|^2.
\]
Отсюда выполняется такое неравенство $\|A_\lambda x\|\ge |\beta|\cdot \|x\|$, а это означает, что $\lambda$ не принадлежит предельному спектру. По критерию Вейля $\lambda$ не принадлежит и всему спектру.
\end{Proof}
\begin{Ut}
 Если $A\in\HH(H)$, то спектральный радиус совпадает с нормой $r(A) = \|A\|$.
\end{Ut}
\begin{Proof}
Мы доказывали теорему о спектральном радиусе, мы ей воспользуемся сейчас. Имеем $A^2 = A\cdot A$. Тогда
$ \|A^2\|\le \|A\|^2$ и $\|A^{2^n}\|\le \|A\|^{2^n}$.

Применим неравенство Коши"--~Буняковского (здесь $S\subset H$ "--- единичный шар)
\[
\|A^2\| = \sup\limits_{x\in S}\|A^2 x\| = \sup\limits_{x,y,\in S}\la A^2 x,y\ra = \sup\limits_{x,y\in S}\la Ax,Ay\ra\ge
  \sup\limits_{x\in S}\la Ax,Ax\ra = \sup\limits_{x\in A} \|Ax\|^2 = \|A\|^2.
\]
Значит, $\|A^2\|\ge \|A\|^2$, отсюда $\|A^2\| = \|A\|^2$. 

Если вспомнить теорему о спектральном радиусе, существует $r(A) = \lim\limits_{n\to\infty} \sqrt[n] {\|A^n\|} = \lim\limits_{n\to\infty} \sqrt[2^n]{\|A^{2^n}\|} = \|A\|$.
\end{Proof}

\begin{Ut}
  Если $A\in\K(H)\cap \HH(H)$, то $\sigma_p(A)\ne\q$. 
\end{Ut}
Вы это и в линейной алгебре доказывали.
\begin{Proof}
  Будем считать, что $\dim H = \infty$. В конечномерном случае рассматривается характеристический многочлен и по основной теореме алгебры у него есть хотя бы один корень. У нас будет бесконечная размерность. Пусть $\sigma_p(A) = \q$. По теореме Рисса"--~Шаудера в этом случае $\sigma(A) = \{0\}$. Ну и следовательно спектральный радиус $r(A) = \|A\| = 0$. И наш оператор нулевой. А все вектора является собственными для нулевого оператора с собственным значением $\lambda=0$, противоречие.
\end{Proof}
\begin{The}[Гильберта"--~Шмидта]
  Если $A\in\K(H)\cap \HH(H)$, то $\exists\ \{e_n\}\subset H$ полная ортонормированная система собственных векторов оператора $A$.
\end{The}
Стоит отметить, что в курсе линейной алгебры эта теорема доказывалась.
\begin{Proof}
Докажем для $H$ сепарабельного. (Без этого требования нажно выбирать несчётную ортонормированную систему.)
Пусть $\dim H=\infty$. Поскольку $A\in \K(H)$, то его спектр не более чем счётный $\sigma_p(A) = \{\lambda_n\}$ (по теореме Рисса"--~Шаудера), кроме того по свойству 1 эти собственные числа действительные, а по свойству два эти собственные пространства ортогональны:
\[
  \{\lambda_n\}\subset \R,\pau H_{\lambda_1} \perp H_{\lambda_2}\pau (\lambda_1\ne \lambda_2).
\]
Можем выделить счётную ортонормированную систему в каждом из подпространств $H_{\lambda_n}$, а затем объединяем в одно систему, получим ортонормированную систему.

Нам осталось доказать, что замкнутая линейная оболочка этой системы $L = \ol{\sp}\{e_n\}$ совпадает с $H$. Это будет означать, что система полна.

В силу непрерывности $A\colon L\to L$. Всякий элемент из подпространства $L$ записывается в виде ряда Фурье. А в силу эрмитовости $A\colon L^\perp \to L^\perp$ "--- ортогональное дополнение тоже является инвариантным подпространством. По свойству 5 следует, что ортогональное дополнение $L^\perp$ содержит собственный вектор оператора $A$. (Ведь замкнутое подпространство гильбертова пространства само является гильбертовым.) Отсюда противоречие с тем, что все собственные векторы содержатся в $L$. Таким образом, $L^\perp = \{0\}$. Следовательно $H = L\oplus L^\perp = L$.
\end{Proof}

Давайте приведём пример компактного эрмитова оператора. Стандарный пример $A\colon \L_2\big([a,b]\big)\to \L_2\big([a,b]\big)$, который является интегральным оператором
\[
  \forall\ f\in \L_2\big([a,b]\big)\pau  A f(x) = \int\limits_a^b k(x,y) f(y)\,dy.
\]
Здесь на ядро $k(x,y)$ интегрального оператора накладывается условия
\begin{roItems}
\item $k(x,y) = \ol{k(y,x)}$ "--- это условие обеспечивает эрмитовость оператора. Мы с вами на одной из лекций считали сопряжённый к интегральному оператору.
\item $k\in \L_2 \big([a,b]\big)$ "--- это, мы доказывали, обеспечивает компактность.
\end{roItems}

\section{Задача Штурма"--~Лиувилля}
Задача ставится так. Есть дифференциальный оператор второго порядка
\[
  D u(x) = - \big(p(x) u'(x)\big)' + q(x) u(x).
\]
Здесь $x\in[a,b]$. Ставится задача найти собственные функции и доказать, что из них можно выбрать полную ортонормированную систему.
Собственные функции в общем случае найти не удаётся, только в частных. Мы сделаем некоторые предположения.

Пусть $p\in C^1[a,b]$, $p(x)>0$, $q(x)\in C[a,b]$ и $q(x)\in\R$. Оператор $D\colon \L\to \L_2\big([a,b]\big)$, где $L\subset W_2^2\big([a,b]\big)$ (подпространства пространства Соболева). Значит, функция $u\in L$, если $u'\in \AC[a,b]$, $u''\in\L_2\big([a,b]\big)$. И задаём граничные условия
\[
  a_0 u(a) + a_1 u'(a) = 0;\quad
  b_0 u(b) + b_1 u'(b) = 0.
\]
Будем полагать, что граничные условия нетривиальны, то есть $a_0^2 + a_1^2 >0$, $b_0^2 + b_1^2>0$.

Нам нужно найти собственные функции, удовлетворяющие всем этим условиям.

Для начала докажем, что этот оператор является симметрическим.
\begin{Lem}
  $D$ "--- симметрический оператор, то есть
\[
  \forall\ u,v\in\L\pau \la Du,v\ra = \la u, Dv\ra,
\]
и $\L\subset \L_2\big([a,b]\big)$ всюду плотно.
\end{Lem}
\begin{Proof}
Второе утверждение мы доказывали уже. А первое следует из соотношения
\[
  \la Du,v\ra - \la u,Dv\ra = \int\limits_a^b \big((pu')'\ol v - u(p\ol v')'\big)\,dx = 
  p(u'\ol v - u\ol v')|_a^b = 0
\]
в силу того, что выполнены граничные условия. Потому что если выполнены граничные условия, то определители
\[
  \det\begin{vmatrix}
  u(a) & u'(a)\\
  \ol v(a) & \ol v'(a)
\end{vmatrix} = 0,\quad
  \det\begin{vmatrix}
  u(b) & u'(b)\\
  \ol v(b) & \ol v'(b)
\end{vmatrix} = 0.
\]
\end{Proof}
Таким образом, мы доказали, что оператор $D$ является симметрическим. Ну а теперь будем строить функцию Грина.

\begin{Lem}
  Если ядро оператор $\ker D=0$, то существует функция Грина $G(x,y)$, которая удовлетворяет следующим условиям
\begin{azItems}
  \item $G$ "--- действительная, симмеричная и непрерывная на $[a,b]^2$.
  \item При $y\ne x$ функция $G(x,y)$ как функция $y$ имеет вторую непрерывную производную по $y$.
  \item При $y\ne x$ имеем $D_y G(x,y) = 0$ и удовлетворяет граничным условиям.
  \item При $y=x\pau G'_y(x,y)|_{x-0}^{x+0} = G'_y(x,x+0) - G'_y(x,x-0) = -\frac1{p(x)}$.
\end{azItems}
\end{Lem}
\begin{Proof}
  Пусть $D u_1 = 0$, $Du_2 = 0$, $u_1$ удовлетворяет первому граничному условию, $u_2$ удовлетворяет второму граничному условию. Так как у нас ядро равно нулю, то эти два решения линейно независимы (иначе $u_1$, например, будет удовлетворять сразу двум граничным условия), то есть всякое решение имеет вид $u = C_1 u_1 + C_2 u_2$. Ну ещё известно из теории дифференциальных уравнений, что определитель Вронского $u_1 u'_2 - u_1' u_2\ne 0$. Тогда $\Delta = p (u_1  u'_2 - u'_1 u_2)\ne 0$. Можно даже показать, что это константа, ведь
\[
  \Delta' = u_1(o u'_2)' - u_2 (p u'_1) = 0.
\]
Ну почти вот и всё доказательство. Осталось определить эту функцию. Функцию мы определяем следующим образом. Фиксируем переменную $x$, $y$ меняем
\[
  G(x,y) = \begin{cases}
  c_1 u_1(y),& y\le x;\\
  c_2 u_2(y), & y\ge x.
\end{cases}
\]
Так берём $c_1,c_2$, чтобы
\[
  \begin{cases}
    c_1 u_1(x) - c_2u_2(x) = 0;\\
    c_1 u'_1(x) -c_2 u'_2(x) = \frac1{p(x)}.
\end{cases}
\]
Осталось выписать коэффициенты $c_1 = -\frac{u_1(x)}\Delta$, $c_2 = -\frac{u_2(x)}\Delta$.
\[
  G(x,y) = -\frac1\Delta\begin{cases}
  u_1(y) u_2(x),& a\le y\le x\le b;\\
  u_1(x) u_2(y),& a\le x\le y\le b.
\end{cases}
\]
Выполнение всех условий гладкости следует из теоремы существования.
\end{Proof}

\begin{The}
Пусть $\ker D=0$ и $f\in\L_2[a,b]$. Тогда $Du = f$, $u\in \L\iff u = Af$, где
\[
  Af(x) = \int\limits_a^b G(x,y) f(y)\,dy.
\]
\end{The}
Иными словами, этот интегральный оператор является обратным к оператору $D$.
\begin{Proof}
 Докажем, что $\forall\ u\in \L \pau ADu = u$. Пусть $Du = f$. Тогда
\[
  \int\limits_a^b G(x,y)f(y)\,dy = \int\limits_a^b G(x,y) Du(y)\,dy
\]
Будем теперь интегрировать по частям только первое слагаемое оперетора $D$. Я не буду всё подробно писать. Думаю, что то, что я напишу, будет понятно. Вспомним лемму, что $D$ симметрический оператор, но остаются некоторые неинтегральные члены, которые я напишу.
\begin{multline*}
\int\limits_a^b G(x,y) Du(y)\,dy = \int\limits_a^x G(x,y) Du(y)\,dy + \int\limits_x^b G(x,y) Du(y)\,dy = \\
 \int\limits_a^b D_y G(x,y) u(y)\,dy + p(G'_yu - Gu')|_a^{x-0} + p(G'_y u - Gu')|_{x+0}^b.
\end{multline*}
Многие слагаемые обращаются в ноль, остаются только
\[
  p(G'_y u - Gu')|_{x+0}^{x-0} = 1.
\]
В точке $x$ функции $u, u', G$ непрерывны, а $G'$ испытывает скачок.


Теперь докажем, что $Da f= f$ почти всюду на $[a,b]$. Для этого воспользуемся тем, что у нас функция Грина является действительной и симметричной. Значит, оператор $Af$ будет эрмитовым оператором.
\[
  \forall\ v\in\L\pau \la DA f,v\ra = \la f, AD v\ra = \la f,v\ra.
\]
Ведь $\L\subset \L_2[a,b]$ всюду плотно. Получаем, что $DA f=f$ почти всюду на $[a,b]$.
\end{Proof}
\begin{Sl}[единственность решения задачи Штурма"--~Лиувилля]
  Существует полная ортонормированная система $\{e_n\}$ в $\L_2[a,b]$, состоящая из собственных функций оператора Штурма"--~Лиувилля.
\end{Sl}
\begin{Proof}
Для доказательства следствия рассмотрим два случая.
\begin{azItems}
\item $\ker D=0$. В этом случае можем просто применить теорему. Ну поскольку в теореме Гильберта"--~Шмидта существует полная ортонормированная система собственных функций, то и у оператора $D$ существует полная ортонормированная система собственных функций. А опертор функции Грина будет как раз непрерывным и компакнтым.
\item $\ker D\ne 0$. Тогда существует $\lambda_0\in\R$, которое не является собственным числом оператора $D$, то есть $\ker (\lambda I - D) = 0$. ($\L_2$ сепарабельно, значит, собственных чисел счётное множество, значит, есть действительное, не являющееся собственным числом). У оператора $D_0 = \lambda_0 I - D$ и оператора $D$ собственные функции не отличаются. Таким образом у нас существует полная ортонормированная система оператора $D$.
\end{azItems}
\end{Proof}

Давайте простой пример я приведу. Без доказательства, хотя там доказывать нечего. Вы наверняка этот пример разбирали в курсе дифференциальных уравнений. Вот такой оператор $D u = - u''$ и граничные условия $u(0) = u(1) = 0$. Легко видеть, что $x,1-x$ "--- решения уравнения $Du=0$ (одно для одного граничного условия, другое для второго). Вычислим функцию Грина
\[
  G(x,y) = \begin{cases}
  y(1-x),&0\le y\le x\le 1;\\
  x(1-y),&0\le x\le y\le 1.
\end{cases}
\]
Выписав $u'' + \lambda u = 0$, вы вспомните, что решения $\sin$ и $\cos$, но $\cos$ отпадает из-за одного граничного условия, остаётся только $\sin$. Таким образом $\lambda_n = \pi^2 n^2$, $e_n = \sqrt2 \sin\pi nx$.
