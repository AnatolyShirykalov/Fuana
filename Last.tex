\section{Абсолютно непрерывные функции}
Начнём с определения абсолютной функций множества. У нас будет дальше $(X,\Sigma,\mu)$ "--- измеримое пространство. Обозначим через $\Sigma_E=\{A\subset E|A\in\Sigma\}$, $E\in\Sigma$.
\begin{Def}
 Функция $\phi\colon \Sigma_E\to\R$ называется зарядом, если $\phi$ $\sigma$-аддитивна. Заряд называется абсолютно непрерывным $\phi\ll\mu$ относительно меры $\mu$, если 
 \[
  \forall\ \e>0\pau\exists\ \delta>0\colon \forall\ A\in\Sigma_E,\ \mu(A)<\delta\imp\big|\phi(A)\big|<\e.
\]
\end{Def}

\begin{The}[об абсолютной непрерывности интеграла Лебега]
  Если $f\in L(E,\mu)$, то $\phi(A) =\int_A f\,d\mu$, $A\in \Sigma_E$, является абсолютно непрерывным зарядом.
\end{The}

\begin{Proof}
  Что интеграл зяряд, мы доказывали в прошлой лекции. Надо доказать только абсолютную непрерывность. Представим $f=f_+-f_-$. Тогда можно считать, что $f\ge 0$. Рассмотрим $E_n=E(f\le n)$, $E_n\nearrow E$. Можно воспользоваться свойством непрерывности снизу для меры.
\[
  \forall\ \e>0\pau\exists\ n\in\N\colon \phi(E\dd E_n)<\frac\e2.
\]
А ещё $\forall\ A\in \E_E\pau \mu(A)<\delta = \frac\e{2n}$, $\phi(A\cap E_n)=\Gint{A\cap E_n}f\,d\mu\le n\delta=\frac\e2$.
Ну и осталось написать, что $\phi(A)\hm=\phi(A\cap E_n)+\underbrace{\phi(A\dd E_n)}_{\le\mu(E\dd E_n)}<\frac\e2+\frac\e2=\e$, поскольку у нас $\phi$ монотонна (так как $f$ неотрицательна).
\end{Proof}

Следующая теорема в нашем курсе если и будет доказана, то на последней лекции, если время останется. Кто интересуется, может прочесть в книге Колмогоров"--~Фомин.
\begin{The}[Радона"--~Никодима]
  Если заряд $\phi\colon \E_E\to\R$ удовлетворяет условию
\[
  \forall\ A\in\E_E\colon \mu(A)=0\pau \imp\phi(A)=0.
\]
$E$ имеет $\sigma$-конечную меру.

Тогда $\exists!$ (с точностью до эквивалентности) $f\in L(E,\mu)$ такая, что $\phi(A) = \I Af\pau \forall\ A\in\E_E$.
\end{The}
Помните, что мы называли функции эквивалентными, если они совпадают почти всюду.
\begin{Proof}
 Единственность легко доказать. Если интегралы совпадают для всех $A\in\E_E\pau \I Af=\I Ag$, то пусть $\exists\ B\in\E_E\colon \mu(B)>0$, такой, что $f(x)>g(x)\pau \forall\ x\in B$. Следовательно, $\I B{(f-g)}>0$.
\end{Proof}

Следствие обычно называется свойством абсолютной непрерывности. Его можно было бы и независимо доказать, но это заняло бы определённое время. Так что просто выведем из теоремы Радона"--~Никодима.
\begin{Sl}[критерий абсолютной непрерывности]
  $\phi\ll\mu\iff \forall\ A\in\E_E\colon \mu(A)=0\imp \phi(A)=0$.
\end{Sl}
\begin{Proof}
  Необходимость очевидна. Потому что если множесво меры нуль $\forall\ \e>0\big|\phi(A)\big|<\e$, то $\phi(A)=0$. А обратное вытекает из теоремы Радона"--~Никодима.
\end{Proof}

\subsection{Функции точки}
Сначала я вам напомню определение функции ограниченной в вариациях.
\begin{Def}
  $F\in B\vee[a,b]$, если
\[
  \bigvee\limits_a^bar(F):=\sup\limits_\tau\RY k1n\big|F(x_k)-F(x_{k-1})\big|<\infty,\pau \tau:=\{a=x_0<x_1<\dots<x_n=b\}.
\]
Пространство будет линейным, и в нём можно ввести норму $\| F\| = \big|F(a)\big|+\bigvee\limits_a^b(F)$.
\end{Def}

Напомню свойства без доказательства. Это должно быть в курсе математического анализа.
\begin{Ut}
  Если $F\in B\vee[a,b]$ и $a<c<b$, то $\bigvee\limits_a^bar(F) = \bigvee\limits_a^car(F)+\bigvee\limits_c^bar(F)$.
\end{Ut}
\begin{Ut}
  Если $F(c-0) = F(c)$. то $V(x)=\bigvee\limits_a^xar(F)$, $V(c-0)=V(c)$.
\end{Ut}
\begin{Ut}
  Разложение Жордана. Если $F\in B\vee [a,b]$, то $\exists\ \alpha(x)\uparrow$ и $\beta(x)\uparrow$, такая, что
\[
  \alpha(a)=\beta(a)=0,\pau F(x) = F(a)+\alpha(x)-\beta(x),\pau V(x) = \alpha(x)+\beta(x).
\]
\end{Ut}
\begin{Proof}
  $\alpha(x):=\frac12\big\{ \bigvee_a^xar(F)+F(x)-F(a)\big\}$, $\beta(x):=\frac12\big\{\bigvee_a^xar(F)-F(x)+F(a)\big\}$.
\end{Proof}

Ещё одну теорему приведу без доказательства.
\begin{The}[Лебега о производной монотонной функции]
  Если функция $f\colon [a,b]\to\R$ монотонна, $f(x)\le f(y)$, если $x\le y$ (или наоборот), то существует производная $f'(x)$ почти всюду на $[a,b]$.
\end{The}

\subsection{Интеграл Лебега"--~Стилтьеса}
  Пусть $F\in B\vee [a,b]$ непрерывна слева. Тогда по разложению Жордана можем написать $F(x)=F(a)+\alpha(x)-\beta(x)$, где $\alpha,\beta\uparrow$. Можно построить меры Лебега"--~Стилтьеса $\mu_\alpha,\mu_\beta$. И мы можем тогда построить заряд Лебега"--~Стилтьеса
\[
  \phi_F = \mu_\alpha-\mu_\beta.
\]

Заряд определён на $\E_F:=\E_\alpha\cap\E_\beta$, пересечение $\sigma$-алгебр мер $\mu_\alpha$ и $\mu_\beta$. Определение теперь.

\begin{Def}
  Интеграл Лебега"--~Стилтьеса $\int\limits_a^b f\,d\phi_F:=\int\limits_a^bf\,d\mu_\alpha-\int\limits_a^bf\,d\mu_\beta$. Определён на полуинтервале $[a,b)$.
\end{Def}
 И напомню определение.
\begin{Def}
  Интеграл Римана"--~Стилтьеса $\int\limits_a^bf\,d F :=\yo{d(\tau)}0 R_\tau(f,\xi,F)$, где 
\[R_\tau(f,\xi,F):=\RY k1n f(\xi_k)\big(F(x_k)-F(x_{k-1})\big),\]
 $\tau$ "--- разбиение отрезка, то есть $\tau = \{a=x_0<x_1<\dots<x_n=b\}$, $d(\tau) = \max\limits_{1\le k\le n}(x_k-x_{k-1})$, $\xi = \{\xi_k\}$ и $\xi_k\in[x_{k-1}k,x_{k}]$.
\end{Def}

\begin{Lem}
  Если функция $F\in C[a,b]$, то сущетсвует интеграл Римана"--~Стилтьеса.
\end{Lem}
\begin{Proof}
  Достаточно рассмотреть, когда $F$ неубываюшая. Тогда интегральная сумма будет является интегралом Лебега от некоторой простой функции. $f\tau(x) = f(\xi_k)$ на $[x_{k-1},x_k)$. Так как функция непрерывно, я могу вместо отрезка брать полуинтервал. Ещё на отрезке $f_\tau\rsH[]{}f$. По теореме Лебега интеграл существует.
\end{Proof}

Кстати функцию $F$ можно переопределить в счётном числе точек. От этого интеграл не изменится.

Нам  эта лемма в общем-то и не понадобится.
\begin{The}[о сравнении интегралов]
  Если функция $f\colon [a,b]$ ограничена и $\exists\ \int\limits_a^bd\,dF$, то $\exists\ \int\limits_a^b f\,d\phi_F$ и они равны.
\end{The}
\begin{Proof}
  Применяем разложение Жордана. Без ограничения общности считаем $F(x)= \alpha(x)\uparrow$ и $f\ge 0$. Рассмотрим в этом случае интегральные суммы Дарбу"--~Стилтьеса для заданного разбиения
\[
  \ul D_\tau(f,\alpha):=\RY k1n \ul a_k m_\a\big([x_{k-1},x_l]\big),\pau \ol D_\tau(f,\alpha):=\RY k1n \ol a_k m_\a\big([x_{k-1},x_l]\big),
\]
где $\ul a_k = \inf\limits_{[x_k,x_{k-1})]} f(x)$, $\ol a_k = \sup\limits_{[x_k,x_{k-1})]} f(x)$, $\tau = \{a=x_0<x_1<\dots<x_n=b\}$. Тогда
\[
 \ul D_\tau(f,\a)\le \ol D_\tau(f,\a).
\]

Осталось доказать равенство.
\[
  \forall\ \e>0\pau \exists\ \delta>0\colon \forall\ \tau\colon d(\tau)<\delta\pau I-\e\le R_\tau(f,\xi,\a)\le I+\e,
  \pau I = \int\limits_a^bf\,d\a.
\]
Тогда суммы Римана будут находиться между суммами Дарбу
\[
  \forall\ \e>0\pau I-\e\le \ul D_\tau(f,\a)\le R_\tau(f,\xi,\a)\le\ol D_\tau(f,\a) \le I+\e
\]
\end{Proof}

\begin{Def}
  $f\in AC[a,b]$, где $f\colon [a,b]\to\R$ абсолютно непрерывна, если
  \[
   \forall\ \e>0\pau\exists\ \d>0\colon \forall\ \DUN k1n(a_k,b_k)\subset [a,b]\colon \RY k1n(b_k-a_k)<\d\imp\RY k1n\big|f(b_k)-f(a_k)\big|<\e.
\]
Такие функции образуют линейную пространство, где можно ввести норму $\|f\|:=\big|f(a)\big|+\int\limits_a^b\big|f'(t)\big|\,d t$, корректность котороой мы проверим чуть позже.
\end{Def}

\begin{Ut}
  Есди $f\in \Lip[a,b]$, то есть $\exists\ C\>0\colon \big|f(x)-f(y)\big|\le C|x-y|$ $\forall\ x,y\in[a,b]$, то $f\in AC[a,b]$.
\end{Ut}

\begin{Ut}
  Если $f\in AC[a,b]$, то $f\in C\vee [a,b]$.
\end{Ut}

\begin{Proof}
  Берётся разбиение $\tau = \{a=x_0<x_1<\dots<x_n=b\}$, такое что $(x_k-x_{k-1})=\frac\d2 = \frac{(b-a)}n$. Тогдп вариация
  \[
\bigvee\limits_a^bar(f) = RY 1n\bigvee\limits_{x_k-1}^{x_k}ar(f)\le n\e = \frac{2(b-a)}\d\e.
\] 
\end{Proof}
\begin{Ut}
  Если $f\in AC[a,b]$, то в разложении Жордана $F(x) =F(a)+\alpha(x)-\beta(x)$ $\a,\b\in AC[a,b]$.
\end{Ut}
\begin{Proof}
 Нам нужно доказать, что $V(x) = \bigvee\limits_a^xar(f)$ абсолютно непрерывна. Нужно воспользоваться свойством вариации и записать, что
\[
  \RY k1n  \big|V(b_k)-V(a_k)\big| = \RY k1n \bigvee\limits_{a_k}^{b_k}ar(f)\le\e.
\]
Достаточно заметить, что вариация на отрезке $[a_k,b_k]$ это точная верхняя грать сумм Дарбу. Нужно вспомнить определение абсолютно непрерывных функций и всё сразу понятно станет.
\end{Proof}

Ну и последнее свойство.
\begin{Ut}
  Если $f\in AC[a,b]$, то $\exists!\ g\in L[a,b]$ (единственность с точностью до эквивалентности), такая что $f(x) = f(a)+\int\limits_a^x g(t)\,dt$.
\end{Ut}
\begin{Proof}
  Разложим $f$ по формуле Жордана $f(x) = f(a) = \a(x)-\b(x)$, $\a,\b\uparrow$. Затем построим меры Лебега"--~Стилтьеса $\mu_\a,\mu_\b$ по функциям $\a,\b$. Эти меры будут абсолютно непрерывны $\mu_\a,\mu_\e\ll\lambda$ ($\lambda$ "--- мера Лебега), так как $\alpha,\b$ абсолютно непрерывны (у нас было два определения абсолютной непрерывности для разных объектов, тут используются оба).

Отсюда вытекает, что заряд $\phi_F\ll \lambda$. Ну и по теоереме Радона"--~Никодима 
\[
  f(x)-f(a) = \phi_f\big([a,x)\big)=\int\limits_a^xg(t)\,dt
\]
для некоторой функции $g\in L[a,b]$. Эта функция будет единственной с точностью до эквивалентности, как и в~теореме Радона"--~Никодима.
\end{Proof}

\begin{Lem}
  Пусть $F\uparrow$ на $[a,b]$. Тогда $\int\limits_a^b F'(t)\,dt\le F(b)-F(a)$. Но если $F\in \Lip[a,b]$, то выполняется равенство.
\end{Lem}
По теореме Лебега производная монотонной функции интегрируема почти всюду. Равенство же может быть и не выполнено, например, если взять функцию Кантора (лесницу Кантора).
\begin{Proof}
  Давайте мы продолжим нашу функцию за отрезок $F(x)=F(b)$, $x\in[b,b+1]$. Функция останется неубывающе. Ну и возьмём такие функции и применим теорему Лебега
\[
  F_n(t) = \frac{F\left(x+\frac1n\right)-F(x)}{\frac1n}\te F'(t).
\]
Предел есть по теореме Лебега почти всюду на $[a,b]$. Теперь применим теорему Фату
\[
  \int\limits_a^bF'(t)\,dt\le \varliminf\int\limits_a^b F_n(t)\,dt = 
  \varliminf \bigg(b\int\limits_b^{b+\frac1n}F(t)\,dt - n\int\limits_a^{a+\frac1b}F(t)\,dt\bigg)\le F(b)-F(a).
\]
Это в силу того, что функция неубывающая.

Осталось вторую часть доказать. Чтобы её доказать, нужно вспомнить определение условия Липшица. Из этого определения вытекает, что производная ограничена почти всюду $\big|F'(t)\big|\le C$ почти всюду. Ну и тогда вместо леммы Фату можно применить теорему Лебега о предельном переходе под знаком интеграла.
\end{Proof}

\begin{The}[характеристические свойсва абсолютно непрерывных функций]
   $F\in AC[a,b]$, если и только если
\[
  \exists\ F'(t) \text{(п.\,в.) на }[a,b],\ F'\in L[a,b], F(x) = F(a)+\int\limits_a^xF'(t)\,dt \forall\ x\in[a,b].
\]
\end{The}
\begin{Proof}
Достаточность вытекает из абсолютной непрерывности интеграла Лебега. 

Применяя свойство разложение Жордана, можно считать, что $F\uparrow$ на $[a,b]$. Давайте ещё считать, что $F(a)=0$. Тогда по свойству 4 имеем
\[
  F(x) = \int\limits_a^xf(t)\,dt,\ f\in L[a,b].
\]
Поэтому для доказательства необходимости нужно доказать, что $F'(t) = f(t)$ почти всюду на $[a,b]$.

Введём такие функции $f_n(x) = \min\big\{f(t),n\big\}$ "--- срез функции на уровне $n$. $f$ определена почти всюду, её можно считать неотрицательной. Обозначим
$F_n(x)=\int\limits_a^x f_n(t)\,dt$. Запишем разность
\[
  F(x)-F_n(x) = \int\limits_a^x\big(\underbrace{f(t)-f_n(t)}_{\ge0}\big)\,dt \uparrow.
\]
Следовательно $F'(x)\ge F'_n(x)$ почти всюду на $[a,b]$. Производная существует почти всюду по теореме Лебега. Давайте запишем ещё следующее равенство по лемме, используя, что $F_n(x)\in\Lip[a,b]$.
\[
  F_n(x) = \int\limits_a^xF'_n(t)\,dt = \int\limits_a^x f_n(t)\,dt,
\]
$F'_n(t) = f_n(t)$ почти всюду на $[a,b]$.
\[
  F'(x)\ge F'_n(x) = f_n(x)\pau \text{п.\,в.}
\]
переходя к пределу, получаем $F'(x)\ge f(x)$ почти всюду на $[a,b]$. Тогда
\[
  \int\limits_a^b\big(F'(t)-f(t)\big)\,dt\ge 0.
\]
А по лемме этот же интеграл будет оцениваться нулём и в другую сторону
\[
  \int\limits_a^b F'(t)\,dt\le F(b)-F(a) = \int\limits_a^bf(t)\,dt\le 0.
\]

Значит, интеграл равен нулю. А поскольку функция неотрицательна, то она равна нулю почти всюду и $F'(t)= f(t)$ почти всюду.
\end{Proof}
